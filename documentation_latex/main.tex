\documentclass[a4paper,12pt]{report}

% Pacchetti necessari
\usepackage[utf8]{inputenc}
\usepackage{newunicodechar}
\newunicodechar{│}{$\vert$}
\newunicodechar{─}{-}
\newunicodechar{└}{$\llcorner$}
\newunicodechar{├}{$\vdash$}

\usepackage[T1]{fontenc}
\usepackage[italian]{babel}
\usepackage{graphicx}
\usepackage{listings}
\usepackage{xcolor}
\usepackage{pgf-umlsd}
\usepackage{float}
\usepackage{booktabs}
\usepackage{tabularx}
\usepackage{geometry}
\usepackage{textcomp}
\usepackage{lmodern}
\usepackage{enumitem}
\usepackage{caption}
\usepackage{dirtree}
\usepackage{forest}
\usepackage{longtable}
\usepackage{colortbl}
\usepackage{microtype}

% Impostazione margini
\geometry{margin=2.5cm}

% Configurazione per diagrammi ASCII
\usepackage{fancyvrb}
\DefineVerbatimEnvironment{asciiart}{Verbatim}
{fontsize=\small, frame=single, commandchars=\\\{\}, fontfamily=cmtt, 
baselinestretch=0.85, xleftmargin=1em}

% Definire un ambiente per JSON
\lstdefinestyle{json}{
    language={},
    basicstyle=\ttfamily\small,
    breaklines=true,
    frame=single,
    showstringspaces=false,
    keywordstyle=\color{blue},
    commentstyle=\color{green!60!black},
    stringstyle=\color{red},
    morekeywords={true, false, null},
    sensitive=true,
    morestring=[b]",
    morecomment=[l]{//},
    morecomment=[s]{/*}{*/}
}

% Definire un ambiente per TypeScript Interface
\lstdefinestyle{typescript}{
    language=Java,
    basicstyle=\ttfamily\small,
    breaklines=true,
    frame=single,
    showstringspaces=false,
    keywordstyle=\color{blue},
    commentstyle=\color{green!60!black},
    stringstyle=\color{red},
    morekeywords={interface, extends, string, number, boolean, any, void, null, undefined},
    sensitive=true
}

% Configurazione listati di codice
\lstset{
    basicstyle=\ttfamily\small,
    breaklines=true,
    frame=single,
    showstringspaces=false,
    keywordstyle=\color{blue},
    commentstyle=\color{green!70!black},
    stringstyle=\color{red!70!black},
}

% Configurazione tabelle
\definecolor{tablerowcolor}{rgb}{0.95, 0.95, 0.95}
\newcommand{\altrowcolor}{\rowcolor{tablerowcolor}}

% Configurazione collegamenti (da caricare quasi per ultimo)
\usepackage{hyperref}
\hypersetup{
    colorlinks=true,
    linkcolor=blue,
    filecolor=magenta,
    urlcolor=cyan,
}

% Informazioni del documento
\title{Site War - Documentazione Tecnica}
\author{Documentazione Completa Progetto}
\date{\today}

\begin{document}

\begin{titlepage}
    \centering
    \vspace*{1cm}
    {\huge\bfseries Site War\\}
    \vspace{1.5cm}
    {\Large Documentazione Tecnica Dettagliata\\}
    \vspace{2cm}
    %\includegraphics[width=0.6\textwidth]{placeholder.png}  Sostituire con un logo appropriato
    \vspace{2cm}
    {\large \today\\}
\end{titlepage}

\tableofcontents
\clearpage

% Include i vari capitoli
\chapter{Introduzione}

\section{Overview del Progetto}
Site War è uno strumento avanzato di web testing che analizza in modo comparativo due siti web, presentando la valutazione tecnica come una ``guerra'' tra siti con la proclamazione finale di un vincitore. Questo approccio creativo è ideato per rendere il web testing tecnico più coinvolgente e accessibile.

L'applicazione è progettata per analizzare in modo approfondito due siti web forniti dall'utente tramite URL, offrendo un confronto dettagliato basato su molteplici parametri tecnici. Al termine dell'analisi, oltre ai risultati dettagliati, verrà dichiarato un vincitore tra i due siti in base a un sistema di punteggio ponderato.

Il tester è in grado di validare tutte le componenti dei due siti, avendo una visione completa degli elementi costitutivi attraverso strutture e grafi di supporto. Durante il tempo di elaborazione, un sistema di animazioni coinvolgenti intrattiene l'utente rappresentando visivamente la ``guerra'' tra i siti basata sui risultati in tempo reale.

\section{Obiettivi}
Gli obiettivi principali del progetto Site War sono:

\begin{itemize}
    \item Rendere coinvolgente, giocoso e più accessibile il processo tecnico di web testing attraverso il concetto della ``guerra tra siti''
    \item Fornire uno strumento di analisi comparativa approfondita per molteplici aspetti tecnici dei siti web
    \item Offrire risultati completi e dettagliati, presentati in modo visivamente efficace e comprensibile
    \item Garantire un'esperienza utente fluida e interattiva attraverso animazioni durante il processo di analisi
    \item Fornire una solida base di dati per valutazioni oggettive sulla qualità e l'efficienza di un sito web rispetto alla concorrenza
\end{itemize}

\section{Target di Utenza}
L'applicazione è pensata per rivolgersi a diverse categorie di utenti:

\begin{itemize}
    \item \textbf{Creatori e sviluppatori web} che desiderano confrontare il proprio sito con siti concorrenti per identificare punti di forza e debolezza
    \item \textbf{Utenti curiosi} interessati a ottenere informazioni tecniche riguardo due siti per valutarne l'efficienza, l'efficacia e la sicurezza dal punto di vista tecnico
    \item \textbf{Penetration tester} alla ricerca di vulnerabilità e differenze tecniche o similitudini tra siti
\end{itemize}

\section{Requisiti Funzionali}
I requisiti funzionali principali dell'applicazione includono:

\begin{itemize}
    \item Inserimento degli URL dei due siti da confrontare
    \item Utilizzo dell'AI per valutare se i due URL appartengono a siti dei quali ha senso eseguire il confronto
    \item \textbf{Analisi generica} (linguaggi frontend, framework, CMS, versioni, IP, ecc.)
    \item \textbf{Analisi dei fattori SEO}
    \item \textbf{Analisi delle vulnerabilità} di sicurezza
    \item \textbf{Analisi delle prestazioni}
    \item Confronto basato sulle analisi per decretare il vincitore della ``guerra tra siti''
\end{itemize}

\section{Requisiti Non Funzionali}

\subsection{Performance e UI}
Durante il tempo di risposta per l'esecuzione di ogni test, l'utente viene intrattenuto tramite animazioni che illustrano in modo creativo la ``guerra tra i siti'' basata sui risultati real-time delle analisi e dei relativi confronti. Ogni test completo non dovrebbe superare i 25 secondi di tempo di risposta.

\subsection{Scalabilità}
La maggior parte delle analisi e dei confronti avviene tramite scraping e algoritmi lato client, garantendo una buona scalabilità del sistema.

\subsection{Sicurezza}
Tutte le comunicazioni devono avvenire tramite protocolli sicuri come HTTPS. Gli algoritmi lato client devono rimanere il più possibile protetti e non iniettabili, mentre le API key utilizzate per i servizi esterni devono essere mantenute sicure.

\subsection{Usabilità}
L'interfaccia utente è progettata per essere intuitiva e accessibile, con layout responsive che garantiscono una buona esperienza sia su desktop che su dispositivi mobile. La UI è concepita per rendere il processo divertente pur mostrando tutti i dettagli tecnici delle analisi e dei confronti.

\subsection{Manutenibilità}
Il codice è strutturato in maniera modulare seguendo le linee guida dell'ingegneria del software, implementando i relativi design pattern e con una documentazione completa.

\subsection{Compatibilità}
L'applicativo deve funzionare correttamente sui principali browser (Chrome, Firefox, Safari, Edge) e su diverse piattaforme/ambienti operativi.

\section{Vincoli Progettuali}
Lo sviluppo del progetto Site War è soggetto ai seguenti vincoli tecnici:

\begin{itemize}
    \item \textbf{Lato server}: Utilizzo esclusivo di PHP
    \item \textbf{Lato client}: HTML5, CSS3, JavaScript
    \item \textbf{Framework}: Non è consentito l'uso di framework architetturali (es. TypeScript, Angular, React, Node.js, Laravel) ad eccezione di jQuery
    \item È consentito l'uso di framework con licenza gratuita per scopi non commerciali per l'implementazione di particolari componenti grafiche, come Bootstrap
\end{itemize}

\section{Struttura del Documento}
Questa documentazione tecnica è organizzata seguendo un approccio top-down, partendo dalla visione architetturale complessiva fino ad arrivare ai dettagli implementativi dei singoli componenti. La documentazione è strutturata nei seguenti capitoli:

\begin{itemize}
    \item \textbf{Architettura del Sistema}: Visione d'insieme dell'architettura, componenti principali e loro interazioni
    \item \textbf{Componente Frontend}: Descrizione dettagliata dell'interfaccia utente, delle animazioni e della visualizzazione dei risultati
    \item \textbf{Componente Backend}: Dettagli sul funzionamento del backend, delle API e dell'integrazione con servizi esterni
    \item \textbf{Componente di Analisi}: Spiegazione dei diversi analizzatori e del sistema di confronto e punteggio
    \item \textbf{Modello dei Dati}: Struttura dei dati utilizzati, metriche e loro elaborazione
    \item \textbf{Casi d'Uso}: Descrizione dettagliata dei casi d'uso principali del sistema
    \item \textbf{Sviluppo e Deployment}: Linee guida per lo sviluppo, testing e deployment
    \item \textbf{Sicurezza}: Considerazioni sulla sicurezza dell'applicazione
    \item \textbf{Conclusioni}: Riepilogo e considerazioni finali
\end{itemize}
\chapter{Architettura del Sistema}

\section{Visione d'Insieme dell'Architettura}
Site War implementa un'architettura client-server con elaborazione distribuita, dove la maggior parte dell'analisi viene eseguita lato client per ottimizzare le prestazioni e ridurre il carico sul server, rispettando il vincolo di completare le analisi entro 25 secondi.

L'architettura è progettata per bilanciare il carico di lavoro tra client e server, assegnando al browser client le analisi che possono essere eseguite direttamente sul DOM e sull'interfaccia utente, mentre il server gestisce le analisi più complesse, l'integrazione con API esterne e la gestione delle chiavi API in modo sicuro.

```mermaid
flowchart TD
    subgraph CLIENT["CLIENT BROWSER"]
        HTML_CSS_JS["- HTML/CSS/JS\n- jQuery\n- Bootstrap\n- Anime.js\n- Chart.js\n- Client Analyzers"]
    end
    
    subgraph SERVER["PHP SERVER"]
        API_Controller["- API Controller\n- Proxy Service\n- AI Validator\n- Advanced Analyzers\n- Result Generator"]
    end
    
    subgraph EXTERNAL["EXTERNAL SERVICES"]
        Google["Google PageSpeed Insights"]
        Moz["Moz API"]
        Security["Security Headers"]
        WHOIS["WHOIS API"]
        HTML["HTML Validator"]
        CSS["CSS Validator"]
        OpenAI["OpenAI API"]
    end
    
    CLIENT <--> SERVER
    CLIENT <--> EXTERNAL
    SERVER <--> EXTERNAL
```

\section{Componenti del Sistema}

\subsection{Componenti Principali}
Il sistema Site War è strutturato in tre componenti principali:

\begin{itemize}
    \item \textbf{Frontend Component}: Gestisce l'interfaccia utente, le animazioni e l'interazione con l'utente. Implementa anche l'analisi lato client e la visualizzazione dei risultati.
    
    \item \textbf{Backend Component}: Gestisce le richieste API, coordina le analisi avanzate e integra i servizi esterni. Include la logica di business lato server, la validazione dei dati e la sicurezza.
    
    \item \textbf{Analysis Component}: Implementa gli algoritmi di analisi, confronto e punteggio. Include componenti sia lato client che lato server, coordinati per fornire un'analisi completa e affidabile.
\end{itemize}

\subsection{Componenti di Supporto}
Oltre ai componenti principali, il sistema include i seguenti componenti di supporto:

\begin{itemize}
    \item \textbf{UI Layer Component}: Gestisce specificamente la rappresentazione visiva dell'interfaccia utente e le animazioni della ``guerra'' tra siti.
    
    \item \textbf{API Layer Component}: Gestisce l'interfaccia tra frontend e backend, l'autenticazione e la validazione delle richieste.
    
    \item \textbf{Data Layer Component}: Gestisce l'elaborazione, l'aggregazione e la formattazione dei dati di analisi.
\end{itemize}

```mermaid
flowchart TD
    subgraph SiteWar["Site War System"]
        Frontend["Frontend\nComponent"] <--> Backend["Backend\nComponent"] <--> Analysis["Analysis\nComponent"]
        Frontend --> UILayer["UI Layer\nComponent"]
        Backend --> APILayer["API Layer\nComponent"]
        Analysis --> DataLayer["Data Layer\nComponent"]
    end
```

\section{Interazioni tra Componenti}
I componenti del sistema interagiscono tra loro seguendo un modello di comunicazione ben definito:

\begin{enumerate}
    \item \textbf{Utente $\rightarrow$ Frontend}: L'utente inserisce gli URL dei due siti da confrontare nell'interfaccia utente.
    
    \item \textbf{Frontend $\rightarrow$ Backend}: Il frontend invia una richiesta di validazione degli URL al backend.
    
    \item \textbf{Backend $\rightarrow$ AI Validator}: Il backend utilizza un servizio AI per verificare la pertinenza del confronto tra i due siti.
    
    \item \textbf{Frontend $\rightarrow$ Analysis (Client-side)}: Il frontend avvia le analisi eseguibili lato browser, visualizzando animazioni durante l'elaborazione.
    
    \item \textbf{Backend $\rightarrow$ Analysis (Server-side)}: Il server esegue parallelamente le analisi più complesse e quelle che richiedono API esterne.
    
    \item \textbf{Analysis $\rightarrow$ Frontend}: I risultati delle analisi vengono progressivamente visualizzati nella UI come ``battaglie'' tra i siti.
    
    \item \textbf{Backend $\rightarrow$ Frontend}: Al termine di tutte le analisi, il sistema calcola il punteggio complessivo e proclama il vincitore.
\end{enumerate}

```mermaid
sequenceDiagram
    participant Browser as Browser Client
    participant Frontend as Frontend Analysis
    participant Backend as Backend Analysis
    participant External as External APIs
    participant Results as Results Processing
    
    Browser->>Frontend: Request
    Note right of Browser: User input
    Frontend-->>Browser: Return
    Frontend->>Backend: Process
    Backend->>External: Request
    External-->>Backend: Return
    Backend->>Results: Process
    Results-->>Browser: Return results
```

\section{Flusso dei Dati}
Il flusso dei dati all'interno del sistema segue un percorso lineare dall'input dell'utente fino alla visualizzazione dei risultati:

```mermaid
flowchart LR
    UserInput["User Input"] --> URLInput["URL Input"] 
    URLInput --> ValidationService["Validation Service"]
    ValidationService --> ClientSideAnalysis["Client-side Analysis"]
    ClientSideAnalysis --> APIIntegration["API Integration"]
    APIIntegration --> ServerSideAnalysis["Server-side Analysis"]
    ServerSideAnalysis --> ResultProcess["Result Process"]
    ResultProcess --> ResultsDisplay["Results Display"]
```

\section{Pattern Architetturali}
L'architettura di Site War implementa diversi pattern di design per garantire modularità, estensibilità e manutenibilità del codice.

\subsection{Pattern Utilizzati}

\begin{itemize}
    \item \textbf{Module Pattern}: Organizzazione del codice JavaScript in moduli autonomi con responsabilità ben definite, migliorando la manutenibilità e prevenendo conflitti di namespace.
    
    \item \textbf{Factory Method}: Creazione di analizzatori specifici per ciascun tipo di test senza accoppiare il codice a classi concrete.
    
    \item \textbf{Observer Pattern}: Notifica dei risultati di analisi completati ai vari componenti dell'UI, permettendo aggiornamenti in tempo reale.
    
    \item \textbf{Strategy Pattern}: Implementazione di diverse strategie di analisi e confronto, permettendo di cambiarle dinamicamente in base alle necessità.
    
    \item \textbf{Adapter Pattern}: Integrazione uniforme con diverse API esterne attraverso un'interfaccia standardizzata.
    
    \item \textbf{Facade Pattern}: Interfaccia semplificata per il complesso sistema di analisi sottostante.
    
    \item \textbf{MVC Pattern Adattato}: Separazione tra dati (Model), logica (Controller) e presentazione (View) anche senza framework MVC.
    
    \item \textbf{Proxy Pattern}: Protezione delle API key e intermediazione con API esterne.
\end{itemize}

```mermaid
flowchart TD
    subgraph SystemArchitecture["System Architecture"]
        Module["Module Pattern"] 
        Observer["Observer Pattern"]
        Factory["Factory Pattern"]
        Facade["Facade Pattern"]
        Strategy["Strategy Pattern"]
        Adapter["Adapter Pattern"]
        MVC["MVC Pattern"] 
        Proxy["Proxy Pattern"]
    end
```

\section{Integrazione con Servizi Esterni}
Site War si integra con diversi servizi API esterni per ottenere dati specializzati che arricchiscono l'analisi dei siti web.

\subsection{API e Servizi Utilizzati}
\begin{itemize}
    \item \textbf{Google PageSpeed Insights}: Analisi delle performance del sito
    \item \textbf{Moz API}: Metriche SEO avanzate
    \item \textbf{Security Headers}: Analisi della sicurezza degli header HTTP
    \item \textbf{WHOIS API}: Informazioni sulla registrazione dei domini
    \item \textbf{HTML Validator W3C}: Validazione degli standard HTML
    \item \textbf{CSS Validator W3C}: Validazione CSS
    \item \textbf{OpenAI API}: Validazione della pertinenza del confronto
\end{itemize}

```mermaid
flowchart TD
    subgraph ExternalAPI["External API Integration"]
        SiteWar["Site War System"] --> ProxyLayer["Proxy Layer"]
        
        ProxyLayer --> PerformanceServices["Performance Services"]
        ProxyLayer --> SEOServices["SEO Services"]
        ProxyLayer --> SecurityServices["Security Services"]
        
        PerformanceServices --> PageSpeed["PageSpeed API"]
        SEOServices --> Moz["Moz API"]
        SecurityServices --> SecurityHeaders["Security Headers"]
    end
```

\subsection{Strategie di Resilienza}
Per garantire robustezza nell'integrazione con API esterne, il sistema implementa diverse strategie di resilienza:

```mermaid
flowchart TD
    PrimaryAPI["Primary API Call"] --> APIResponse["API Response Handler"] --> CacheLayer["Cache Layer"]
    
    PrimaryAPI -- "Failure" --> SecondaryAPI["Secondary API Call"]
    SecondaryAPI --> AlternativeProcessing["Alternative Processing"] --> DegradedExperience["Degraded Experience"]
    
    SecondaryAPI -- "Failure" --> ClientSideAlternative["Client-side Alternative"] --> FallbackResults["Fallback Results"]
```

\section{Sicurezza dell'Architettura}
La sicurezza è un aspetto fondamentale dell'architettura di Site War, implementata a diversi livelli per proteggere sia i dati degli utenti che le risorse del sistema.

```mermaid
flowchart TD
    subgraph SecurityArch["Security Architecture"]
        ClientSide["Client-side Security"] 
        APILayer["API Layer Security"] 
        ServerSide["Server-side Security"]
        
        ClientSide --> InputValidation["Input Valid. XSS Prevention CSP Implement. Obfuscation"]
        APILayer --> RequestValidation["Request Valid. CSRF Protection Rate Limiting Authentication"]
        ServerSide --> APIKeyProtection["API Key Protect Input Sanitiz. Error Handling HTTPS Enforce."]
    end
```

\section{Distribuzione del Carico}
Per rispettare il vincolo di tempo di 25 secondi per l'analisi completa, il sistema distribuisce strategicamente il carico di lavoro tra client e server:

```mermaid
flowchart TD
    subgraph Workload["Workload Division"]
        ClientSide["Client-side (65%)"] 
        ServerSide["Server-side (25%)"] 
        ExternalAPIs["External APIs (10%)"]
        
        ClientSide --> ClientAnalysis["- DOM Analysis\n- Basic SEO\n- Performance\n- Visual Metrics"]
        ServerSide --> ServerAnalysis["- Security Checks\n- Advanced Tech Detection\n- Result Process"]
        ExternalAPIs --> ExternalAnalysis["- SEO Advanced\n- WHOIS Data\n- SSL Certificate\n- Ext. Validation"]
    end
```

\section{Principi di Progettazione}
L'architettura di Site War è guidata dai seguenti principi di progettazione:

\begin{itemize}
    \item \textbf{Separation of Concerns}: Separazione netta tra diversi aspetti del sistema
    \item \textbf{Loose Coupling}: Accoppiamento debole tra i componenti per facilitare modifiche e manutenzione
    \item \textbf{High Cohesion}: Alta coesione all'interno dei moduli per garantire focalizzazione su responsabilità specifiche
    \item \textbf{Progressive Enhancement}: Funzionalità di base garantite a tutti gli utenti, con miglioramenti progressivi
    \item \textbf{Responsivity}: Design adattivo per diverse dimensioni di schermo e dispositivi
    \item \textbf{Fallback Strategies}: Strategie di fallback per gestire errori e situazioni impreviste
    \item \textbf{Performance by Design}: Ottimizzazione delle performance come principio di progettazione fondamentale
\end{itemize}

\section{Deployment e Ambiente}
L'architettura di deployment del sistema prevede tre ambienti distinti: sviluppo, test e produzione, con strategie specifiche per ciascun ambiente.

```mermaid
flowchart LR
    Development["Development Environment"] --> Testing["Testing Environment"] --> Production["Production Environment"]
    
    Development --> DevTools["- Local Dev\n- Docker Env\n- Mock APIs\n- Live Reload"]
    Testing --> TestTools["- Integration\n- Tests\n- Performance\n- Security"]
    Production --> ProdTools["- Prod Hosting\n- CDN\n- Monitoring\n- Load Balance"]
```

\chapter{Componente Frontend}

\section{Panoramica del Frontend}
Il componente frontend gestisce l'interfaccia utente, orchestrando la visualizzazione dei dati, le animazioni e l'interazione con il backend. Vengono impiegate tecnologie quali HTML, CSS, JavaScript e librerie specifiche per animazioni e grafici.

\section{Architettura del Frontend}

\subsection{Diagramma delle Classi}
Il diagramma seguente rappresenta la struttura delle classi del componente frontend.

```mermaid
classDiagram
    SiteWarApp <|-- FormUI
    SiteWarApp <|-- BattleUI
    SiteWarApp <|-- AnimationEngine
    FormUI <|-- AnalysisUI
    BattleUI <|-- ResultsUI
    AnimationEngine <|-- ChartManager
    
    class SiteWarApp {
        +initialize()
        +handleRouting()
    }
    
    class FormUI {
        +validateURLs()
        +submitForm()
    }
    
    class BattleUI {
        +showBattle()
        +updateProgress()
    }
    
    class AnimationEngine {
        +createAnimation()
        +updateAnimationState()
    }
    
    class AnalysisUI {
        +showProgress()
        +updateAnalysisState()
    }
    
    class ResultsUI {
        +displayResults()
        +showWinner()
    }
    
    class ChartManager {
        +createCharts()
        +updateChartData()
    }
```

\subsection{Moduli Principali}

\subsubsection{SiteWarApplication}
Modulo principale che gestisce l'applicazione, il routing e l'orchestrazione dei componenti. Funge da punto di ingresso per l'applicazione e coordina l'interazione tra i diversi moduli.

\subsubsection{EventBus}
Implementa il pattern Observer per la comunicazione tra i diversi moduli. Consente ai componenti di sottoscriversi a eventi specifici e di ricevere notifiche quando questi eventi si verificano.

\subsubsection{FormUI}
Gestisce il form di inserimento degli URL, inclusa la validazione e la sottomissione. Verifica che gli URL inseriti siano in un formato valido prima di avviare l'analisi.

\subsubsection{BattleUI}
Responsabile delle animazioni che rappresentano la ``guerra'' tra i siti durante l'analisi. Utilizza Anime.js e Particles.js per creare effetti visivi coinvolgenti.

\subsubsection{ResultsUI}
Visualizza i risultati del confronto in modo chiaro e intuitivo, con grafici e tabelle. Utilizza Chart.js per creare visualizzazioni comparative dei dati.

\subsubsection{AnalysisUI}
Mostra l'avanzamento dell'analisi e fornisce feedback in tempo reale all'utente. Aggiorna la barra di progresso e visualizza lo stato corrente dell'analisi.

\subsubsection{AnimationEngine}
Motore per la creazione e gestione delle animazioni, basato su Anime.js e Particles.js. Implementa le diverse fasi dell'animazione della battaglia.

\subsubsection{ChartManager}
Gestisce la creazione e l'aggiornamento dei grafici utilizzati per visualizzare i risultati dell'analisi. Utilizza Chart.js per renderizzare i dati in modo visivamente efficace.

\section{Diagramma di Stato dell'Interfaccia Utente}
L'interfaccia utente di Site War passa attraverso diversi stati durante l'interazione con l'utente. Di seguito è riportato il diagramma di stato:

```mermaid
stateDiagram-v2
    [*] --> Initial
    Initial --> Form: showForm
    Form --> Initial: resetApp
    Form --> Loading: submitForm
    Loading --> Battle: validationComplete
    Battle --> Results: analysisComplete
    Results --> Details: viewCategory
    Details --> Results: backToResults
    Results --> Export: exportResults
    Export --> Initial: resetApp
    Results --> Initial: resetApp
```

\section{Diagramma delle Animazioni}
Le animazioni sono un aspetto fondamentale dell'esperienza utente in Site War. Il diagramma seguente illustra il flusso di stati delle animazioni durante il processo di analisi:

```mermaid
stateDiagram-v2
    [*] --> Idle
    Idle --> Approach: initAnimation
    Approach --> Idle: reset
    Approach --> Clash: progress > 25%
    Clash --> Battle: progress > 50%
    Clash --> Victory: reset
    Battle --> Victory: progress > 90%
    Victory --> Reset: reset
    Reset --> [*]
```

\section{Diagramma di Sequenza - Processo di Analisi Frontend}
Il diagramma di sequenza seguente illustra il flusso di interazioni tra i vari componenti durante il processo di analisi:

```mermaid
sequenceDiagram
    actor User
    participant FormUI
    participant SiteWarApp
    participant BattleUI
    participant Analyzers
    participant Backend
    
    User->>FormUI: Submit Form
    FormUI->>SiteWarApp: Validate URLs
    SiteWarApp->>BattleUI: Show Battle UI
    BattleUI->>BattleUI: Start Animation
    SiteWarApp->>Analyzers: Create Analyzers
    SiteWarApp->>Backend: Request Server Analysis
    Backend-->>SiteWarApp: Response
    Analyzers-->>SiteWarApp: Analysis Progress
    SiteWarApp->>BattleUI: Update Progress
    BattleUI->>BattleUI: Update Animation
    BattleUI->>User: See Animation
    Analyzers-->>SiteWarApp: Analysis Complete
    SiteWarApp->>User: Show Results
```

\section{Struttura dell'Interfaccia Utente}

\subsection{UI Components Hierarchy}
L'interfaccia utente di Site War è organizzata in una gerarchia di componenti che facilita la manutenzione e l'estensibilità. Di seguito la rappresentazione:

```mermaid
flowchart TD
    MainLayout["Main Layout"] --> FormSection["Form Section"]
    MainLayout --> BattleSection["Battle Section"]
    MainLayout --> ResultsSection["Results Section"]
    
    FormSection --> URLInput["URL Input Fields"]
    BattleSection --> BattleArena["Battle Arena"]
    ResultsSection --> WinnerDisplay["Winner Display"]
    ResultsSection --> DetailTabs["Detail Tabs"]
    
    WinnerDisplay --> ComparisonCharts["Comparison Charts"]
    WinnerDisplay --> CategoryData["Category Data"]
```

\subsection{Sezioni Principali}
L'interfaccia utente è suddivisa in tre sezioni principali:
\begin{itemize}
    \item \textbf{Form Section}: contiene il form per l'inserimento degli URL.
    \item \textbf{Battle Section}: visualizza le animazioni della battaglia.
    \item \textbf{Results Section}: mostra i risultati dell'analisi.
\end{itemize}

\section{Implementazione dell'Animazione}

\subsection{Concetto ``Guerra tra Siti''}
Le animazioni rappresentano visivamente la "guerra" tra i siti web, utilizzando effetti che riflettono la qualità tecnica dei siti analizzati.

\subsection{Fasi della Battaglia}
\begin{enumerate}
    \item \textbf{Approccio (0-25\%)}: I siti si avvicinano, con effetti particellari iniziali.
    \item \textbf{Scontro (25-50\%)}: Primo impatto e intensificazione degli effetti.
    \item \textbf{Combattimento (50-90\%)}: Scambio di "colpi" ed effetti intensi.
    \item \textbf{Vittoria (90-100\%)}: Il sito vincitore emerge e si passa alla schermata dei risultati.
\end{enumerate}

\subsection{Tecnologie di Animazione}
\begin{itemize}
    \item Anime.js
    \item Particles.js
    \item CSS Animations
    \item Canvas
\end{itemize}

\section{Interfaccia di Visualizzazione Risultati}

\subsection{Dashboard Principale}
La dashboard principale dei risultati presenta:
\begin{itemize}
    \item Il vincitore della battaglia, con un badge distintivo.
    \item Punteggi complessivi per entrambi i siti.
    \item Grafici comparativi delle principali categorie.
    \item Opzioni per visualizzare dettagli specifici.
\end{itemize}

\subsection{Visualizzazione per Categoria}
Ogni categoria (Performance, SEO, Sicurezza, Aspetti tecnici) ha una visualizzazione dedicata con:
\begin{itemize}
    \item Tabella comparativa dei valori specifici.
    \item Grafici che evidenziano le differenze.
    \item Indicatori visivi del vincitore per ogni metrica.
    \item Suggerimenti per miglioramenti.
\end{itemize}

\subsection{Sistema di Punteggio Visivo}
\begin{itemize}
    \item Utilizzo di colori (verde per buone e rosso per cattive metriche).
    \item Badge per il vincitore di ogni categoria.
    \item Visualizzazione proporzionale dei punteggi.
\end{itemize}

\section{Responsive Design}

\subsection{Principi di Design}
\begin{itemize}
    \item Layout fluido basato su Bootstrap.
    \item Media queries per adattare l'interfaccia a diverse dimensioni di schermo.
    \item Approccio mobile-first per garantire una buona esperienza su dispositivi mobili.
\end{itemize}

\subsection{Ottimizzazioni per Mobile}
\begin{itemize}
    \item Visualizzazione semplificata delle animazioni su dispositivi mobili.
    \item Grafici adattivi che si ridimensionano in base allo schermo.
    \item Navigazione touch-friendly.
    \item Loading ottimizzato per connessioni più lente.
\end{itemize}

\section{Accessibilità}

\subsection{Conformità WCAG 2.1 AA}
\begin{itemize}
    \item Contrasto adeguato per testo ed elementi UI.
    \item Etichette ARIA per componenti interattivi.
    \item Focus visibile per la navigazione da tastiera.
    \item Testo alternativo per elementi visivi.
\end{itemize}

\subsection{Compatibilità con Screen Reader}
\begin{itemize}
    \item Markup semantico.
    \item Annunci dinamici per aggiornamenti UI.
    \item Ordine logico di tabulazione.
    \item Descrizioni per grafici e visualizzazioni.
\end{itemize}

\section{Performance Client-Side}

\subsection{Ottimizzazioni}
\begin{itemize}
    \item Lazy loading per componenti non essenziali.
    \item Minimizzazione e compressione di CSS e JavaScript.
    \item Caricamento asincrono delle librerie.
    \item Utilizzo di sprites CSS per icone.
\end{itemize}

\subsection{Monitoraggio Performance}
\begin{itemize}
    \item Tracciamento dei tempi di rendering.
    \item Ottimizzazione delle animazioni per dispositivi meno potenti.
    \item Rilevamento automatico delle capacità del dispositivo.
\end{itemize}

\section{Comunicazione con il Backend}

\subsection{Modello di Comunicazione}
Il frontend comunica con il backend tramite chiamate AJAX utilizzando jQuery. Le richieste sono strutturate per minimizzare il numero di chiamate e ottimizzare le prestazioni.

\subsection{Gestione degli Errori}
Il frontend implementa una gestione robusta degli errori per garantire una buona esperienza utente anche in caso di problemi di comunicazione:
\begin{itemize}
    \item Timeout per richieste non risposte.
    \item Retry automatici per errori temporanei.
    \item Feedback visivo in caso di errori.
    \item Modalità degradata in caso di indisponibilità del backend.
\end{itemize}

\section{Diagramma di Comunicazione UI-Eventi}
Il diagramma seguente illustra il flusso di comunicazione basato su eventi tra i vari componenti dell'interfaccia utente:

```mermaid
flowchart TD
    FormUI -- "form.submit" --> EventBus
    FormUI -- "form.validated" --> ValidationService
    EventBus --> SiteWarApp
    
    SiteWarApp -- "analysis.start\nanalysis.progress\nanalysis.complete" --> BattleUI
    BattleUI --> AnalysisUI
    AnalysisUI --> ResultsUI
    
    ResultsUI -- "export.request" --> ExportUI
```

\section{Gestione dei Dati sul Client}

\subsection{Modello di Dati Client-Side}
Il frontend gestisce i dati delle analisi tramite un modello strutturato che facilita l'elaborazione e la visualizzazione dei risultati.

\subsection{Persistenza Temporanea}
\begin{itemize}
    \item Utilizzo di \texttt{localStorage} per mantenere i risultati recenti.
    \item Cache in memoria per ottimizzare le performance.
    \item Reset dei dati al lancio di una nuova analisi.
\end{itemize}

\section{Testing del Frontend}

\subsection{Strategie di Testing}
\begin{itemize}
    \item Test unitari per i moduli principali.
    \item Test di integrazione per le interazioni tra moduli.
    \item Test di compatibilità cross-browser.
    \item Test di performance per garantire la fluidità delle animazioni.
    \item Test di accessibilità WCAG.
\end{itemize}

\subsection{Automazione dei Test}
\begin{itemize}
    \item Utilizzo di strumenti come Lighthouse per test di performance.
    \item Test automatizzati di accessibilità.
    \item Test di regressione per garantire la stabilità durante lo sviluppo.
\end{itemize}
\chapter{Componente Backend}

\section{Panoramica del Backend}
Il componente backend di Site War è responsabile dell'elaborazione delle richieste, dell'orchestrazione delle analisi avanzate, della comunicazione con le API esterne e della generazione dei risultati finali. Implementato in PHP puro senza framework architetturali, questo componente segue un'architettura modulare con pattern di design ben definiti per garantire manutenibilità, sicurezza e performance.

Il backend non solo elabora le richieste dal frontend, ma svolge anche un ruolo cruciale nelle analisi che non possono essere eseguite direttamente nel browser, come le verifiche di sicurezza avanzate, l'integrazione con API che richiedono chiavi private e la validazione della pertinenza del confronto tramite AI.

\section{Architettura del Backend}

\subsection{Diagramma delle Classi}
L'organizzazione del codice backend segue i principi della programmazione orientata agli oggetti, con una chiara separazione delle responsabilità tra le diverse classi. Di seguito è riportato il diagramma delle classi che illustra l'architettura del componente backend:

```mermaid
classDiagram
    class APIController {
        -controllers: Map<string, Controller>
        +__construct()
        +processRequest() void
        -sendResponse(statusCode, data) void
    }
    
    class Controller {
        <<interface>>
        +handleRequest(method, params) array
    }
    
    class AnalyzeCtrl {
        -validator
        -aiService
        -resultService
        +__construct()
        +handleRequest(method, params) array
        -analyzeSite() array
    }
    
    class ValidateCtrl {
        -validator
        -aiService
        +__construct()
        +handleRequest(method, params) array
        -validateUrls() bool
    }
    
    class ReportCtrl {
        -cache
        +__construct()
        +handleRequest(method, params) array
        -getProgress() array
    }
    
    class ServiceFactory {
        +createService(type) Service
    }
    
    class BaseAnalyzer {
        #url: string
        #results: array
        +__construct(url)
        +analyze() array
        +getResults() array
        +isComplete() bool
    }
    
    class BaseService {
        #config: array
        #result: mixed
        +__construct(config)
        +execute() mixed
        +getResult() mixed
        +hasError() bool
    }
    
    class SEO {
        +analyze() array
    }
    
    class Security {
        +analyze() array
    }
    
    class Performance {
        +analyze() array
    }
    
    class Proxy {
        +execute() mixed
    }
    
    class AI {
        +execute() mixed
    }
    
    APIController --> Controller : manages
    Controller <|.. AnalyzeCtrl : implements
    Controller <|.. ValidateCtrl : implements
    Controller <|.. ReportCtrl : implements
    
    ServiceFactory --> BaseAnalyzer : creates
    ServiceFactory --> BaseService : creates
    
    BaseAnalyzer <|-- SEO : extends
    BaseAnalyzer <|-- Security : extends
    BaseAnalyzer <|-- Performance : extends
    
    BaseService <|-- Proxy : extends
    BaseService <|-- AI : extends
```

\subsection{Componenti Principali}

\subsubsection{APIController}
Classe principale che gestisce tutte le richieste HTTP in entrata, le indirizza al controller appropriato e formatta le risposte. Funge da punto di ingresso per tutte le richieste API e implementa un meccanismo di routing semplice ma efficace.

\subsubsection{Controller}
Interfaccia che definisce il contratto per tutti i controller specifici. Ogni controller è responsabile della gestione di un tipo specifico di richiesta e dell'implementazione della logica di business corrispondente.

\subsubsection{AnalyzeController}
Gestisce le richieste di analisi dei siti web, coordinando il processo di analisi e aggregando i risultati. Si occupa di creare i vari analizzatori tramite il ServiceFactory, eseguire le analisi e processare i risultati per il confronto finale.

\subsubsection{ValidateController}
Gestisce la validazione preliminare degli URL inseriti dall'utente e la verifica della pertinenza del confronto tramite l'AIService. Fornisce un feedback immediato sulla validità e rilevanza del confronto richiesto.

\subsubsection{ServiceFactory}
Implementa il pattern Factory Method per creare istanze di servizi e analizzatori. Centralizza la creazione di oggetti e facilita l'estensibilità del sistema.

\subsubsection{BaseAnalyzer}
Classe base astratta per tutti gli analizzatori specifici. Definisce l'interfaccia comune e implementa funzionalità condivise, come la gestione dello stato di completamento e il recupero dei risultati.

\subsubsection{BaseService}
Classe base astratta per tutti i servizi specifici. Simile a BaseAnalyzer, ma focalizzata sui servizi che non eseguono analisi dirette, come il proxy per API esterne.

\subsubsection{ProxyService}
Gestisce le comunicazioni con API esterne in modo sicuro, nascondendo le chiavi API e implementando strategie di cache e rate limiting. Agisce come intermediario tra l'applicazione e i servizi esterni.

\subsubsection{AIService}
Utilizza l'intelligenza artificiale (OpenAI API) per valutare la pertinenza del confronto tra i siti web inseriti dall'utente. Fornisce un livello di validazione semantica che va oltre la semplice validazione sintattica degli URL.

\section{Diagramma di Stato - Richiesta API}
Il seguente diagramma illustra il ciclo di vita di una richiesta API, dai vari stati che attraversa fino alla risposta finale:

```mermaid
stateDiagram-v2
    InitialRequest --> ValidatedRequest: validateRequest
    ValidatedRequest --> InitialRequest: validation failed
    
    InitialRequest --> ErrorResponse: invalid
    ValidatedRequest --> ProcessingRequest: valid
    
    ProcessingRequest --> AnalysisProgress
    AnalysisProgress --> ErrorResponse: error during
    AnalysisProgress --> ResultsProcess: done
    
    ResultsProcess --> FinalResponse
```


\section{Diagramma di Sequenza - Processo di Analisi Backend}
Il diagramma di sequenza seguente illustra il flusso di interazioni tra i vari componenti durante il processo di analisi lato server:

```mermaid
sequenceDiagram
    participant Client
    participant API_Ctrl as API-Ctrl
    participant Analyze
    participant Factory
    participant Analyzer
    participant ExtAPI
    
    Client->>API_Ctrl: Request
    API_Ctrl->>Analyze: Route
    Analyze->>Factory: Create
    Factory->>Analyzer: Create
    Analyzer-->>Factory: Return
    Factory-->>Analyze: Return
    
    Analyze->>Analyzer: Analyze
    Analyzer->>ExtAPI: Request
    ExtAPI-->>Analyzer: Response
    Analyzer-->>Analyze: Results
    Analyze->>Analyze: Process
    Analyze-->>API_Ctrl: Response
    API_Ctrl-->>Client: Response
```

\section{Struttura dei File Server}
Il codice backend è organizzato in una struttura di directory che riflette le responsabilità e le relazioni tra i componenti. Questa organizzazione favorisce la manutenibilità e la scalabilità del codice:

\begin{verbatim}
server/
|
+-- api/
|   +-- index.php                 # Entry point per tutte le richieste API
|   +-- controllers/              # Controller per diversi endpoint
|   |   +-- AnalyzeController.php
|   |   +-- ValidateController.php
|   |   +-- ReportController.php
|   +-- config/                   # Configurazioni API
|       +-- api_keys.php          # Chiavi API (protette)
|       +-- services.php          # Configurazione servizi
|
+-- core/                         # Core del sistema
|   +-- APIController.php         # Controller API principale
|   +-- Controller.php            # Interfaccia controller
|   +-- ServiceFactory.php        # Factory per servizi e analizzatori
|   +-- ConfigManager.php         # Gestione configurazione
|
+-- services/                     # Servizi business logic
|   +-- analyzers/                # Analizzatori specifici
|   |   +-- BaseAnalyzer.php      # Classe base analizzatore
|   |   +-- SEOAnalyzer.php
|   |   +-- SecurityAnalyzer.php
|   |   +-- PerformanceAnalyzer.php
|   |   +-- TechnologyAnalyzer.php
|   +-- AIService.php             # Servizio AI
|   +-- ProxyService.php          # Proxy API
|   +-- ResultService.php         # Elaborazione risultati
|
+-- utils/                        # Utility
|   +-- Cache.php                 # Sistema di cache
|   +-- HttpClient.php            # Client HTTP
|   +-- Logger.php                # Sistema di logging
|   +-- RateLimiter.php           # Limitatore di frequenza
|   +-- Validator.php             # Validazione input
|   +-- Security.php              # Funzioni sicurezza
|
+-- cache/                        # Directory per file di cache
    +-- data/                     # Cache dei dati
    +-- ratelimit/                # Cache per rate limiting
\end{verbatim}

\section{Integrazione con API Esterne}
Il backend di Site War si integra con diverse API esterne per ottenere dati specializzati e completi sui siti web analizzati. La gestione di queste integrazioni è centralizzata attraverso il ProxyService che garantisce sicurezza, efficienza e resilienza.

```mermaid
flowchart TD
    APIConnector --> PageSpeed["PageSpeed API"]
    APIConnector --> MozAPI["Moz API"]
    APIConnector --> SecurityHeaders["Security Headers"]
    APIConnector --> WHOISAPI["WHOIS API"]
    APIConnector --> OpenAIAPI["OpenAI API"]
```

\subsection{API e Servizi Utilizzati}
\begin{table}[H]
\centering
\begin{tabular}{|l|l|l|}
\hline
\textbf{API} & \textbf{Scopo} & \textbf{Endpoint} \\
\hline
Google PageSpeed Insights & Analisi performance & /pagespeedonline/v5/runPagespeed \\
\hline
Moz API & Metriche SEO & /links/api \\
\hline
Security Headers & Analisi sicurezza & /api/v1/analyze \\
\hline
WHOIS API & Informazioni domini & /whoisdata \\
\hline
W3C Validator & Validazione HTML/CSS & /check \\
\hline
\hline
OpenAI API & Validazione pertinenza & /v1/completions \\
\hline
\end{tabular}
\caption{API esterne utilizzate}
\label{table:api-services}
\end{table}

\subsection{Diagramma di Collaborazione - API Integration}
Il seguente diagramma illustra il flusso di collaborazione tra i componenti durante l'integrazione con API esterne:

```mermaid
flowchart TD
    Analyzer -- "1. request API" --> ProxyService
    ProxyService -- "2. return result" --> Analyzer
    
    Analyzer -- "4. process results" --> ResultProcess
    ProxyService -- "3. proxy request" --> ExtAPI["Ext. API"]
    
    ResultProcess -- "5. final result" --> APIResponse
```


\section{Sistema di Cache}
Per ottimizzare le prestazioni e ridurre il carico sulle API esterne, il backend implementa un sistema di cache efficiente che memorizza temporaneamente i risultati delle analisi e delle chiamate API.

\subsection{Implementazione}
Il sistema di cache è implementato nella classe Cache, che fornisce un'interfaccia semplice per memorizzare e recuperare dati in base a chiavi univoche. La cache è basata su file per garantire persistenza anche in caso di riavvio del server.

\subsection{Strategia di Caching}
\begin{itemize}
    \item \textbf{Risultati di analisi}: Cache di 24 ore
    \item \textbf{Verifiche di pertinenza AI}: Cache di 7 giorni
    \item \textbf{Chiamate API esterne}: Tempo variabile in base al servizio (da 1 ora a 24 ore)
    \item \textbf{Pulizia automatica}: I file scaduti vengono eliminati quando vengono richiesti
\end{itemize}

\section{Rate Limiting}
Per prevenire l'abuso delle API esterne e garantire il rispetto delle quote di utilizzo, il backend implementa un sistema di rate limiting che controlla il numero di richieste effettuate a ciascun servizio in un determinato periodo di tempo.

\subsection{Strategia di Rate Limiting}
Il sistema di rate limiting è implementato nella classe RateLimiter, che tiene traccia delle richieste effettuate per ciascun servizio e blocca le richieste eccessive. Le quote sono configurabili e variano in base al servizio.

\begin{table}[H]
\centering
\begin{tabular}{|l|l|l|}
\hline
\textbf{Servizio} & \textbf{Limite} & \textbf{Periodo} \\
\hline
PageSpeed API & 100 & 24 ore \\
\hline
Moz API & 10 & 1 ora \\
\hline
Security Headers & 50 & 1 ora \\
\hline
WHOIS API & 100 & 24 ore \\
\hline
Wappalyzer API & 100 & 24 ore \\
\hline
OpenAI API & 20 & 1 ora \\
\hline
\end{tabular}
\caption{Limiti di rate per servizio}
\label{table:rate-limits}
\end{table}

\section{Sicurezza}
La sicurezza è un aspetto fondamentale del backend di Site War, implementata a diversi livelli per proteggere sia i dati degli utenti che le risorse del sistema.

\subsection{Sanitizzazione Input}
Tutti gli input degli utenti sono sanitizzati prima di essere utilizzati, per prevenire attacchi come SQL Injection, XSS e altri. La classe Security fornisce metodi per sanitizzare diversi tipi di input.

\subsection{Protezione Chiavi API}
Le chiavi API utilizzate per accedere ai servizi esterni sono memorizzate in modo sicuro fuori dalla directory pubblica, per evitare l'accesso non autorizzato. Il ProxyService si occupa di gestire queste chiavi in modo sicuro durante le richieste.

\subsection{Content Security Policy}
L'applicazione implementa una robusta Content Security Policy per prevenire attacchi XSS e altri tipi di attacchi basati sull'iniezione di contenuti.

\subsection{HTTPS}
Tutte le comunicazioni tra client e server avvengono tramite HTTPS per garantire la crittografia dei dati in transito e prevenire attacchi man-in-the-middle.

\section{Logging e Monitoraggio}
Il backend implementa un sistema di logging completo per tenere traccia delle attività del sistema, degli errori e delle prestazioni. Questo facilita il debugging e il monitoraggio del sistema in produzione.

\subsection{Sistema di Logging}
Il sistema di logging è implementato nella classe Logger, che fornisce diversi livelli di logging (debug, info, warning, error) e scrive i log in file organizzati per data.

\subsection{Monitoraggio delle API}
Il sistema implementa un monitoraggio delle API esterne per:
\begin{itemize}
    \item Tracciare i tempi di risposta
    \item Registrare gli errori
    \item Monitorare il rate limiting
    \item Verificare la disponibilità dei servizi
\end{itemize}

\section{Gestione degli Errori}
Il backend implementa una gestione robusta degli errori per garantire che il sistema continui a funzionare anche in presenza di problemi e fornisca feedback utili agli utenti.

```mermaid
flowchart TD
    ClientDetection["Client Detection"] --> ErrorClassify["Error Classify"] --> UserFeedback["User Feedback"]
    ServerDetection["Server Detection"] --> ErrorLogging["Error Logging"] --> FallbackMechanism["Fallback Mechanism"]
```


\subsection{Tipi di Errore Gestiti}
\begin{table}[H]
\centering
\begin{tabular}{|l|l|}
\hline
\textbf{Tipo di Errore} & \textbf{Strategia di Gestione} \\
\hline
URL non valido & Validazione client-side con feedback immediato \\
\hline
API non disponibile & Utilizzo di cache o alternativa client-side \\
\hline
Timeout analisi & Risultati parziali con notifica all'utente \\
\hline
Errore JS client & Catch globale con logging e riavvio modulo \\
\hline
Errore server & Risposta di errore con suggerimento alternativo \\
\hline
Accesso negato API & Fallback a funzionalità limitate senza API \\
\hline
\end{tabular}
\caption{Tipi di errore e strategie di gestione}
\label{table:error-handling}
\end{table}

\section{Diagramma di Attività - Analisi Backend}
Il seguente diagramma illustra il flusso di attività completo durante il processo di analisi lato server:

```mermaid
flowchart TD
    Start --> ValidateRequest
    ValidateRequest --> Valid{"Valid?"}
    
    Valid -- No --> ReturnError
    Valid -- Yes --> CheckRelevance
    
    CheckRelevance --> Relevant{"Relevant?"}
    Relevant -- No --> ReturnWarning
    Relevant -- Yes --> CheckCache
    
    ReturnWarning --> ProcessResults
    
    CheckCache --> Cached{"Cached?"}
    Cached -- Yes --> GetCacheResults
    Cached -- No --> CreateAnalyzers
    
    CreateAnalyzers --> PerformAnalysis
    PerformAnalysis --> ProcessResults
    
    GetCacheResults --> ProcessResults
    
    ProcessResults --> CompareSites
    CompareSites --> DetermineWinner
    DetermineWinner --> CacheResults
    CacheResults --> ReturnResponse
    ReturnResponse --> End
```

\section{Estensibilità}
Il backend di Site War è progettato per essere facilmente estensibile, consentendo l'aggiunta di nuovi analizzatori, servizi e API esterne senza modificare la struttura complessiva del sistema.

\subsection{Aggiunta di Nuove API}
Per aggiungere una nuova API esterna, è sufficiente:
\begin{enumerate}
    \item Aggiungere la configurazione del servizio in \texttt{ProxyService::getServiceConfig()}
    \item Aggiungere la chiave API in \texttt{config/api\_keys.php}
    \item Implementare eventuali logiche di fallback in \texttt{ProxyService::implementFallback()}
    \item Aggiungere il rate limiting in \texttt{RateLimiter::getServiceLimits()}
\end{enumerate}

\subsection{Creazione di Nuovi Analizzatori}
Per aggiungere un nuovo tipo di analisi:
\begin{enumerate}
    \item Creare una nuova classe che estende \texttt{BaseAnalyzer}
    \item Implementare i metodi richiesti (\texttt{analyze}, \texttt{calculateScore})
    \item Aggiungere il nuovo analizzatore in \texttt{AnalyzeController}
    \item Aggiornare il sistema di punteggio in \texttt{ResultService}
\end{enumerate}

\section{Interfacce API}

\subsection{API Endpoints}
\begin{table}[H]
\centering
\begin{tabular}{|l|l|l|l|}
\hline
\textbf{Endpoint} & \textbf{Metodo} & \textbf{Descrizione} & \textbf{Parametri} \\
\hline
/api/analyze & POST & Esegue l'analisi completa & site1, site2 \\
\hline
/api/validate & POST & Valida gli URL e la pertinenza & site1, site2 \\
\hline
/api/progress & GET & Controlla lo stato di avanzamento & session\_id \\
\hline
/api/export & GET & Esporta i risultati in diversi formati & session\_id, format \\
\hline
\end{tabular}
\caption{Endpoint API}
\label{table:api-endpoints}
\end{table}

\subsection{Formato della Risposta}
Tutte le risposte API utilizzano JSON con la seguente struttura base:

\begin{verbatim}
{
  "status": "success|error",
  "data": {
    // Dati specifici della risposta
  },
  "message": "Messaggio informativo (opzionale)"
}
\end{verbatim}

Per le analisi complete, la struttura dei dati è più complessa e include i risultati dettagliati per entrambi i siti, insieme al confronto e alla determinazione del vincitore.
\chapter{Componente di Analisi}

\section{Panoramica del Componente}
Il componente di analisi rappresenta il cuore funzionale di Site War, responsabile dell'esecuzione dei test sui siti web, della raccolta dei dati e della preparazione dei risultati per il confronto. È suddiviso in moduli di analisi client-side e server-side che operano in parallelo per garantire un'analisi completa entro il limite di 25 secondi.

Questo componente implementa una strategia di distribuzione del carico, sfruttando le capacità del browser per le analisi che possono essere eseguite direttamente sul DOM e sul rendering, mentre delega al server le analisi più complesse o quelle che richiedono l'integrazione con API esterne. Il risultato è un sistema ibrido che ottimizza l'uso delle risorse e minimizza i tempi di attesa.

\section{Architettura del Componente}

\subsection{Diagramma delle Classi}
Il componente di analisi è strutturato seguendo i principi della programmazione orientata agli oggetti, con una chiara separazione delle responsabilità e un'architettura estensibile. Il diagramma seguente illustra le principali classi e le loro relazioni:

```mermaid
classDiagram
    class AnalysisManager {
        -site1Url: string
        -site2Url: string
        -analyzers: Map<string, BaseAnalyzer>
        -progress: number
        -results: object
        +constructor(site1Url, site2Url)
        +runAnalysis() Promise<object>
        +trackProgress() number
        +getResults() object
        -initAnalyzers() void
        -analyzeOneSite(url) Promise<object>
        -compareResults(site1Results, site2Results) object
    }
    
    class AnalyzerFactory {
        +createAnalyzer(type, url) BaseAnalyzer
        +getAvailableAnalyzers() string[]
    }
    
    class BaseAnalyzer {
        #url: string
        #results: object
        #progress: number
        #isCompleted: boolean
        +constructor(url)
        +analyze() Promise<object>
        +getResults() object
        +getProgress() number
        +isComplete() boolean
        #calculateScore() number
        #updateProgress(value) void
    }
    
    class DOMAnalyzer {
        -dom: Document
        +analyze() Promise
        -parseDOM() void
        -analyzeStructure()
        -checkSemantics()
    }
    
    class SEOAnalyzer {
        -headElements
        +analyze() Promise
        -checkMetaTags()
        -analyzeHeadings()
        -checkLinks()
    }
    
    class PerformanceAnalyzer {
        -metrics: object
        +analyze() Promise
        -measureFCP()
        -measureLCP()
        -calculateCLS()
    }
    
    class SecurityAnalyzer {
        -headers: object
        +analyze() Promise
        -checkSSL() object
        -checkHeaders() object
        -scanVulnerabilities()
    }
    
    AnalysisManager --> AnalyzerFactory: creates
    AnalyzerFactory --> BaseAnalyzer: creates
    BaseAnalyzer <|-- DOMAnalyzer
    BaseAnalyzer <|-- SEOAnalyzer
    BaseAnalyzer <|-- PerformanceAnalyzer
    BaseAnalyzer <|-- SecurityAnalyzer
```

\subsection{Componenti Principali}

\subsubsection{AnalysisManager}
Classe centrale che coordina l'intero processo di analisi. Si occupa di inizializzare gli analizzatori, eseguire le analisi per entrambi i siti, monitorare lo stato di avanzamento e aggregare i risultati finali.

\subsubsection{AnalyzerFactory}
Implementa il pattern Factory Method per creare istanze di analizzatori specifici in base al tipo richiesto. Centralizza la creazione di oggetti e facilita l'estensibilità del sistema.

\subsubsection{BaseAnalyzer}
Classe base astratta per tutti gli analizzatori specifici. Definisce l'interfaccia comune e implementa funzionalità condivise, come la gestione dello stato di completamento, il monitoraggio del progresso e la restituzione dei risultati.

\subsubsection{Analizzatori Concreti}
Classi specializzate che estendono BaseAnalyzer e implementano analisi specifiche:
\begin{itemize}
    \item \textbf{DOMAnalyzer}: Analizza la struttura del DOM, la gerarchia degli elementi e la qualità del markup.
    \item \textbf{SEOAnalyzer}: Valuta l'ottimizzazione per i motori di ricerca, analizzando meta tag, struttura URL, alt text per immagini, ecc.
    \item \textbf{PerformanceAnalyzer}: Misura le performance del sito, inclusi tempi di caricamento, reattività e fluidità.
    \item \textbf{SecurityAnalyzer}: Verifica la sicurezza del sito, analizzando configurazione SSL, header di sicurezza, vulnerabilità, ecc.
    \item \textbf{TechnologyAnalyzer}: Identifica le tecnologie utilizzate dal sito, come framework, librerie e CMS.
\end{itemize}

\subsubsection{APIConnector}
Gestisce le comunicazioni con le API esterne, fornendo un'interfaccia uniforme per le richieste HTTP e la gestione delle risposte.

\subsubsection{ResultComparator}
Si occupa di confrontare i risultati delle analisi per i due siti e di determinare il vincitore in base ai punteggi ottenuti nelle diverse categorie.

\section{Diagramma di Stato - Processo di Analisi}
Il processo di analisi attraversa diversi stati, dall'inizializzazione fino al completamento. Il diagramma seguente illustra questa transizione di stati:

```mermaid
stateDiagram-v2
    [*] --> Idle
    Idle --> Initiated: initAnalysis
    Initiated --> Idle: reset
    Initiated --> Analyzing: startAnalysis
    Analyzing --> Timeout: setTimeout
    Analyzing --> Completed: analysisComplete
    Timeout --> Completed: reset
    Completed --> Reset: reset
    Reset --> [*]
```

\section{Diagramma di Sequenza - Processo di Analisi}
Il diagramma di sequenza seguente illustra l'interazione tra i vari componenti durante un processo di analisi completo:

```mermaid
sequenceDiagram
    actor User
    participant AnalysisManager
    participant AnalyzerFactory
    participant BaseAnalyzer
    participant APIConnector
    participant ResultComparator
    
    User->>AnalysisManager: StartAnalysis
    AnalysisManager->>AnalyzerFactory: Create Analyzers
    AnalyzerFactory->>BaseAnalyzer: Create Analyzers
    BaseAnalyzer-->>AnalyzerFactory: Return Analyzers
    AnalyzerFactory-->>AnalysisManager: Return Analyzers
    AnalysisManager->>BaseAnalyzer: Run Analysis
    BaseAnalyzer->>APIConnector: API Request
    APIConnector-->>BaseAnalyzer: API Response
    BaseAnalyzer->>ResultComparator: Process Data
    ResultComparator-->>BaseAnalyzer: Return Results
    AnalysisManager->>User: Progress Update
    AnalysisManager->>User: Analysis Complete
```

\section{Diagramma dei Casi d'Uso - Analisi}
Il componente di analisi implementa diversi casi d'uso specifici relativi all'analisi e al confronto dei siti web:

```mermaid
flowchart TD
    User(["User"])
    
    User --> InitiateSiteComparison["Initiate Site Comparison"]
    
    InitiateSiteComparison --> AnalyzeSite1["Analyze Site 1"]
    InitiateSiteComparison --> AnalyzeSite2["Analyze Site 2"]
    
    AnalyzeSite1 --> PerformDOM["Perform DOM Analysis"]
    AnalyzeSite1 --> PerformSEO["Perform SEO Analysis"]
    AnalyzeSite2 --> PerformDOM
    AnalyzeSite2 --> PerformSEO
    
    PerformDOM --> PerformSecurity["Perform Security Analysis"]
    PerformSEO --> PerformPerformance["Perform Performance Analysis"]
    
    PerformSecurity --> CompareResults["Compare Results"]
    PerformPerformance --> CompareResults
    
    CompareResults --> DetermineWinner["Determine Winner"]
```

\section{Moduli di Analisi}
Il componente di analisi è suddiviso in diversi moduli specializzati, ciascuno responsabile di un aspetto specifico dell'analisi di un sito web.

\subsection{Analisi DOM e Struttura (Client-side)}
\begin{itemize}
    \item \textbf{Responsabilità}: Analizzare la struttura del documento HTML, la gerarchia degli elementi e la qualità del markup
    \item \textbf{Metriche chiave}:
    \begin{itemize}
        \item Struttura dei heading (H1-H6)
        \item Rapporto testo/codice
        \item Uso corretto di elementi semantici
        \item Accessibilità della struttura
    \end{itemize}
\end{itemize}

\subsection{Analisi SEO (Client+Server)}
\begin{itemize}
    \item \textbf{Responsabilità}: Valutare l'ottimizzazione per i motori di ricerca
    \item \textbf{Metriche chiave}:
    \begin{itemize}
        \item Meta tag (title, description)
        \item URL structure
        \item Alt text per immagini
        \item Schema markup
        \item Sitemap e robots.txt
        \item Canonical URLs
    \end{itemize}
\end{itemize}

\subsection{Analisi Performance (Client-side)}
\begin{itemize}
    \item \textbf{Responsabilità}: Misurare la velocità e l'efficienza del caricamento
    \item \textbf{Metriche chiave}:
    \begin{itemize}
        \item First Contentful Paint (FCP)
        \item Largest Contentful Paint (LCP)
        \item Time to Interactive (TTI)
        \item Cumulative Layout Shift (CLS)
        \item First Input Delay (FID)
        \item Resource loading times
    \end{itemize}
\end{itemize}

\subsection{Analisi Sicurezza (Server-side)}
\begin{itemize}
    \item \textbf{Responsabilità}: Verificare la sicurezza del sito web
    \item \textbf{Metriche chiave}:
    \begin{itemize}
        \item HTTP security headers
        \item SSL/TLS configuration
        \item Content Security Policy
        \item Cross-site scripting vulnerabilities
        \item HSTS implementation
        \item Cookie security
    \end{itemize}
\end{itemize}

\subsection{Analisi Tecnologica (Client+Server)}
\begin{itemize}
    \item \textbf{Responsabilità}: Identificare tecnologie, framework e librerie utilizzate
    \item \textbf{Metriche chiave}:
    \begin{itemize}
        \item Framework frontend
        \item CMS in uso
        \item Server-side technologies
        \item JavaScript libraries
        \item Versioni del software
        \item API utilizzate
    \end{itemize}
\end{itemize}

\section{Sistema di Punteggio}

\subsection{Diagramma del Calcolo del Punteggio}
Il sistema di punteggio è una componente essenziale per la determinazione del vincitore. Il diagramma seguente illustra il processo di calcolo del punteggio:

```mermaid
flowchart LR
    RawData["Raw Data"] --> Validation["Validation"]
    Validation --> TypeConversion["Type Conversion"]
    TypeConversion --> ScaleAdjustment["Scale Adjustment"]
    ScaleAdjustment --> FinalProcessing["Final Processing"]
    FinalProcessing --> NormalizedData["Normalized Data"]
```

\subsection{Formula di Punteggio}
Il punteggio finale viene calcolato come una media ponderata dei punteggi nelle singole categorie:

\textbf{Punteggio Categoria}:
\begin{verbatim}
CategoryScore = (Metric1 * Weight1) + (Metric2 * Weight2) + ... + (MetricN * WeightN)
\end{verbatim}

\textbf{Punteggio Finale}:
\begin{verbatim}
FinalScore = (PerformanceScore * 0.3) + (SEOScore * 0.25) + 
             (SecurityScore * 0.25) + (TechnicalScore * 0.2)
\end{verbatim}

\subsection{Pesi delle Categorie}
\begin{table}[H]
\centering
\begin{tabular}{|l|c|l|}
\hline
\textbf{Categoria} & \textbf{Peso} & \textbf{Descrizione} \\
\hline
Performance & 30\% & Velocità e reattività del sito \\
\hline
SEO & 25\% & Ottimizzazione per motori di ricerca \\
\hline
Sicurezza & 25\% & Protezione e conformità \\
\hline
Aspetti Tecnici & 20\% & Qualità del codice e tecnologie \\
\hline
\end{tabular}
\caption{Pesi delle categorie nel calcolo del punteggio finale}
\label{table:category-weights}
\end{table}

\subsection{Normalizzazione delle Metriche}
Per garantire una valutazione equa e significativa, le metriche raw vengono normalizzate in un punteggio da 0 a 100 utilizzando scale e soglie specifiche per ciascuna metrica.

\begin{table}[H]
\centering
\begin{tabular}{|l|l|l|c|}
\hline
\textbf{Metrica} & \textbf{Valore Raw} & \textbf{Valutazione} & \textbf{Punteggio} \\
\hline
FCP & < 1000ms & Eccellente & 90-100 \\
\hline
FCP & 1000-2500ms & Buono & 60-89 \\
\hline
FCP & 2500-4000ms & Migliorabile & 30-59 \\
\hline
FCP & > 4000ms & Scarso & 0-29 \\
\hline
LCP & < 2500ms & Eccellente & 90-100 \\
\hline
LCP & 2500-4000ms & Buono & 60-89 \\
\hline
LCP & 4000-6000ms & Migliorabile & 30-59 \\
\hline
LCP & > 6000ms & Scarso & 0-29 \\
\hline
CLS & < 0.1 & Eccellente & 90-100 \\
\hline
CLS & 0.1-0.25 & Buono & 60-89 \\
\hline
CLS & 0.25-0.4 & Migliorabile & 30-59 \\
\hline
CLS & > 0.4 & Scarso & 0-29 \\
\hline
SSL Grade & A+ & Eccellente & 90-100 \\
\hline
SSL Grade & A & Molto Buono & 80-89 \\
\hline
SSL Grade & B & Buono & 70-79 \\
\hline
SSL Grade & C & Migliorabile & 50-69 \\
\hline
SSL Grade & F & Scarso & 0-49 \\
\hline
\end{tabular}
\caption{Normalizzazione delle metriche principali}
\label{table:metric-normalization}
\end{table}

\section{Integrazione con API Esterne}
Il componente di analisi si integra con diverse API esterne per ottenere dati specializzati sui siti web analizzati.

\subsection{Diagramma di Integrazione API}

```mermaid
flowchart TD
    APIConnector[("APIConnector")]
    
    APIConnector --> PageSpeed["PageSpeed API"]
    APIConnector --> MozSEO["Moz SEO API"]
    APIConnector --> SecurityHeaders["Security Headers"]
    APIConnector --> WHOIS["WHOIS API"]
    APIConnector --> W3CValidator["W3C Validator"]
    APIConnector --> OpenAI["OpenAI API"]
```

\subsection{API Utilizzate}
\begin{table}[H]
\centering
\begin{tabular}{|l|l|}
\hline
\textbf{API} & \textbf{Scopo} \\
\hline
Google PageSpeed Insights & Metriche di performance \\
\hline
Moz API & Metriche SEO \\
\hline
Security Headers & Analisi header di sicurezza \\
\hline
WHOIS API & Informazioni sul dominio \\
\hline
W3C Validator & Validazione HTML/CSS \\
\hline
\hline
OpenAI API & Validazione pertinenza \\
\hline
\end{tabular}
\caption{API esterne utilizzate}
\label{table:external-apis}
\end{table}

\section{Gestione del Parallelismo}

\subsection{Diagramma di Esecuzione Parallela}
Per ottimizzare i tempi di esecuzione, il componente di analisi implementa diverse strategie di parallelizzazione:

```mermaid
flowchart TD
    StartAnalysis["Start Analysis"] --> Site1Analysis["Site 1 Analysis"] & Site2Analysis["Site 2 Analysis"]
    
    Site1Analysis & Site2Analysis --> ParallelAnalysis["Parallel Analysis Execution"]
    
    ParallelAnalysis --> DOM["DOM"] & SEO["SEO"] & Perf["Perf"] & Sec["Sec"]
    
    DOM & SEO & Perf & Sec --> ResultsAggregation["Results Aggregation"]
    
    ResultsAggregation --> Comparison["Comparison"]
```

\subsection{Strategie di Parallelizzazione}
\begin{enumerate}
    \item \textbf{Parallelizzazione Inter-sito}: Analisi parallela dei due siti web
    \item \textbf{Parallelizzazione Intra-sito}: Esecuzione parallela dei diversi analizzatori per lo stesso sito
    \item \textbf{Esecuzione Asincrona API}: Richieste API asincrone per minimizzare i tempi di attesa
    \item \textbf{Prioritizzazione Task}: Priorità alle analisi con minor tempo di esecuzione
    \item \textbf{Pooling Risorse}: Limitazione del numero di richieste parallele per evitare sovraccarichi
\end{enumerate}

\section{Gestione degli Errori}

\subsection{Diagramma di Gestione Errori}

```mermaid
flowchart TD
    AnalysisRequest["Analysis Request"] --> APIRequest["API Request"]
    
    APIRequest -- Success --> DataProcessing["Data Processing"]
    APIRequest -- Error --> ErrorDetection["Error Detection"]
    
    DataProcessing --> SuccessResult["Success Result"]
    ErrorDetection --> ErrorClassification["Error Classification"]
    
    SuccessResult --> ReturnFullResult["Return Full Result"]
    ErrorClassification --> ErrorRecovery["Error Recovery"]
    
    ErrorRecovery --> ReturnPartialResult["Return Partial Result"]
```

\subsection{Tipi di Errore e Strategie di Recupero}
\begin{table}[H]
\centering
\begin{tabular}{|l|l|}
\hline
\textbf{Tipo di Errore} & \textbf{Strategia di Recupero} \\
\hline
Timeout API & Utilizzare valori predefiniti con penalità \\
\hline
API non disponibile & Usare analisi client-side alternativa \\
\hline
Dati parziali & Continuare con i dati disponibili \\
\hline
Errore di parsing & Utilizzare una struttura di pagina predefinita \\
\hline
Errore di rete & Ritentare con backoff esponenziale \\
\hline
Errore di autenticazione API & Passare a modalità limitata \\
\hline
\end{tabular}
\caption{Tipi di errore e strategie di recupero}
\label{table:error-recovery}
\end{table}

\section{Estensibilità del Sistema}

\subsection{Diagramma di Estensibilità}

```mermaid
flowchart TD
    BaseAnalyzerInterface["Base Analyzer Interface"] --> CustomAnalyzer["Custom Analyzer"]
    CustomAnalyzer --> AnalyzerRegistry["Analyzer Registry"]
    AnalyzerRegistry --> AnalysisManager["Analysis Manager"]
```    

\subsection{Passi per Aggiungere un Nuovo Analizzatore}
\begin{enumerate}
    \item Creare una nuova classe che estende \texttt{BaseAnalyzer}
    \item Implementare i metodi richiesti (\texttt{analyze}, \texttt{getResults}, \texttt{isComplete})
    \item Registrare il nuovo analizzatore nel \texttt{AnalyzerFactory}
    \item Aggiornare il sistema di punteggio per includere i nuovi risultati
\end{enumerate}

\subsection{Interfaccia Plug-in}
Il sistema supporta l'aggiunta di nuovi analizzatori tramite un'interfaccia plug-in ben definita:

\begin{verbatim}
interface AnalyzerPlugin {
  name: string;
  description: string;
  version: string;
  category: string;
  weight: number;
  
  // Metodi richiesti
  initialize(config: object): void;
  analyze(url: string): Promise<object>;
  getResults(): object;
  getScore(): number;
}
\end{verbatim}

\section{Diagramma dei Componenti di Sistema}

```mermaid
flowchart TD
    AnalysisManager["Analysis Manager"] --> AnalysisFactory["Analysis Factory"]
    AnalysisFactory --> |creates| AnalyzerComponent["Analyzer Component"]
    AnalyzerComponent --> AnalysisResults["Analysis Results"]
    AnalysisManager --> AnalysisResults
    AnalyzerComponent --> ExternalAPIAccess["External API Access"]
    AnalysisResults --> ResultComparator["Result Comparator"]
    ExternalAPIAccess --> ResultComparator
```
\chapter{Modello dei Dati}

\section{Panoramica del Modello dei Dati}
Il modello dei dati di Site War è progettato per gestire efficacemente i dati relativi all'analisi e al confronto dei siti web. Questo modello è strutturato per supportare l'acquisizione, l'elaborazione, la comparazione e la visualizzazione dei dati in modo coerente e performante.

Il sistema si basa su un modello orientato agli oggetti che rappresenta i risultati delle analisi attraverso classi ben definite, con una chiara gerarchia e relazioni tra le diverse entità. Questo approccio facilita l'estensibilità del sistema e la manutenzione del codice, consentendo l'aggiunta di nuove metriche e categorie di analisi senza impattare la struttura esistente.

\section{Diagramma delle Classi - Modello dei Dati}

Il diagramma seguente illustra la struttura principale del modello dei dati di Site War, mostrando le classi principali e le loro relazioni:

```mermaid
classDiagram
    class AnalysisResult {
        +site1: SiteAnalysis
        +site2: SiteAnalysis
        +winner: string
        +comparison: ComparisonResult
        +timestamp: number
        +determineWinner() string
        +getWinningCategories(site) string[]
        +toJSON() object
    }
    
    class SiteAnalysis {
        +url: string
        +performance: PerformanceMetrics
        +seo: SEOMetrics
        +security: SecurityMetrics
        +technical: TechnicalMetrics
        +finalScore: number
        +calculateFinalScore() number
        +getHighestScoringCategory() string
        +getLowestScoringCategory() string
    }
    
    class ComparisonResult {
        +performance: string
        +seo: string
        +security: string
        +technical: string
        +overallScore: string
        +scoreGap: number
        +getWinningCategories(site) string[]
        +isCloseDuel() boolean
    }
    
    class BaseMetrics {
        +score: number
        +metrics: object
        +details: object
        +getScore() number
        +getMetrics() object
        +getDetails() object
    }
    
    class PerformanceMetrics {
        +fcp: number
        +lcp: number
        +tti: number
        +cls: number
        +totalSize: number
        +getTiming() obj
    }
    
    class SEOMetrics {
        +title: string
        +meta: object
        +headings: object
        +links: object
        +images: object
        +getSEOIssues()
    }
    
    class SecurityMetrics {
        +ssl: object
        +headers: object
        +vulnerabilities
        +cookiesSecurity
        +csp: object
        +getVulnCount()
    }
    
    class TechnicalMetrics {
        +html: object
        +css: object
        +javascript: obj
        +responsive: bool
        +technologies: []
        +getTechStack()
    }
    
    class DomainInfo {
        +registrar: string
        +creationDate: date
        +expiryDate: date
        +nameservers: []
        +ipAddress: string
        +getAge() number
    }
    
    AnalysisResult --> SiteAnalysis
    AnalysisResult --> ComparisonResult
    BaseMetrics <|-- PerformanceMetrics
    BaseMetrics <|-- SEOMetrics
    BaseMetrics <|-- SecurityMetrics
    BaseMetrics <|-- TechnicalMetrics
```

\section{Struttura Dati - Formato JSON}

\subsection{Formato Completo}
Il sistema utilizza JSON come formato principale per la rappresentazione e lo scambio di dati. Di seguito è riportato un esempio di struttura JSON completa per i risultati di un'analisi:

\begin{asciiart}
{
  "site1": {
    "url": "https://example1.com",
    "performance": {
      "score": 85,
      "metrics": {
        "fcp": 1200,
        "lcp": 2500,
        "tti": 3500,
        "cls": 0.05,
        "totalSize": 1500000
      },
      "details": {
        "resources": {
          "js": 450000,
          "css": 120000,
          "images": 850000,
          "fonts": 80000
        },
        "resourceCount": 32,
        "renderBlocking": 3
      }
    },
    "seo": {
      "score": 78,
      "metrics": {
        "title": "Good",
        "meta": "Average",
        "headings": "Good",
        "images": "Average",
        "links": "Good"
      },
      "details": {
        "titleLength": 55,
        "metaDescription": 140,
        "hasCanonical": true,
        "imgAltTags": 85,
        "headingStructure": "Well structured",
        "linkQuality": "Good"
      }
    },
    "security": {
      "score": 92,
      "metrics": {
        "ssl": "A+",
        "headers": "Good",
        "vulnerabilities": 0,
        "cookies": "Secure",
        "csp": "Implemented"
      },
      "details": {
        "sslDetails": {
          "grade": "A+",
          "protocol": "TLS 1.3",
          "expiry": "2024-05-15"
        },
        "securityHeaders": {
          "strictTransportSecurity": true,
          "contentSecurityPolicy": true,
          "xFrameOptions": true,
          "xContentTypeOptions": true,
          "referrerPolicy": true
        },
        "cookiesDetails": {
          "secure": true,
          "httpOnly": true,
          "sameSite": "Strict"
        }
      }
    },
    "technical": {
      "score": 88,
      "metrics": {
        "html": "Valid",
        "css": "Valid",
        "javascript": "Modern",
        "responsive": true,
        "technologies": ["HTML5", "CSS3", "JavaScript", "jQuery"]
      },
      "details": {
        "htmlVersion": "HTML5",
        "cssVersion": "CSS3",
        "jsFeatures": ["ES6", "Modules"],
        "frameworks": ["jQuery 3.6.0"],
        "libraries": ["Animate.js", "Chart.js"],
        "serverInfo": {
          "server": "Nginx",
          "platform": "Linux"
        }
      }
    },
    "finalScore": 85.8
  },
  "site2": {
    // Struttura simile a site1
  },
  "winner": "site1",
  "comparison": {
    "performance": "site1",
    "seo": "site2",
    "security": "site1",
    "technical": "site1",
    "overallScore": "site1",
    "scoreGap": 7.3
  },
  "timestamp": 1633276850000
}
\end{asciiart}

\subsection{Formato Semplificato per UI}
Per ottimizzare la visualizzazione nell'interfaccia utente, viene utilizzata una versione semplificata del formato JSON:

\begin{asciiart}
{
  "site1": {
    "url": "https://example1.com",
    "scores": {
      "performance": 85,
      "seo": 78,
      "security": 92,
      "technical": 88,
      "final": 85.8
    },
    "highlights": {
      "strengths": ["Fast loading time", "Excellent security", "Modern technologies"],
      "weaknesses": ["Meta descriptions", "Image optimization"]
    }
  },
  "site2": {
    // Struttura simile a site1
  },
  "winner": {
    "site": "site1",
    "name": "example1.com",
    "margin": "Clear win (7.3 points)"
  },
  "categoryWinners": {
    "performance": "site1",
    "seo": "site2",
    "security": "site1",
    "technical": "site1"
  }
}
\end{asciiart}

\section{Diagramma delle Interazioni con i Dati}
Il diagramma seguente illustra il flusso di interazioni con i dati durante il processo di analisi:

```mermaid
flowchart LR
    Analyzers["Analyzers"] --> RawData["Raw Data"]
    RawData --> NormalizedData["Normalized Data"]
    NormalizedData --> CategoryScores["Category Scores"]
    CategoryScores --> ResultsProcessor["Results Processor"]
    ResultsProcessor --> UIDisplay["UI Display"]
```


\section{Diagramma ER per Risultati dell'Analisi}
Il diagramma Entity-Relationship (ER) seguente illustra le relazioni tra le entità del modello dei dati:

```mermaid
erDiagram
    AnalysisResult ||--|| ComparisonResult : has
    AnalysisResult ||--|{ SiteAnalysis : contains
    SiteAnalysis ||--|| DomainInfo : has
    DomainInfo ||--|{ CategoryMetrics : has
    CategoryMetrics ||--|| CategoryScore : scores
    CategoryMetrics ||--o{ PerformanceMetrics : "specializes to"
    CategoryMetrics ||--o{ SEOMetrics : "specializes to"
    CategoryMetrics ||--o{ SecurityMetrics : "specializes to"
    CategoryMetrics ||--o{ TechnicalMetrics : "specializes to"
    CategoryMetrics ||--o{ AccessibilityMetrics : "specializes to"
```

\section{Diagramma di Aggregazione dei Dati}
Il processo di aggregazione dei dati grezzi in risultati significativi è illustrato nel seguente diagramma:

```mermaid
flowchart TD
    RawMetrics["Raw Metrics"] -- "aggregated into" --> CategoryMetrics["Category Metrics"]
    CategoryMetrics -- "aggregated into" --> SiteAnalysis["Site Analysis"]
    SiteAnalysis -- "compared to create" --> AnalysisResult["Analysis Result"]
```

\section{Dettaglio delle Entità di Dati Principali}

\subsection{AnalysisResult}
\textbf{Responsabilità}: Rappresenta i risultati completi dell'analisi e del confronto tra due siti web.

\textbf{Attributi}:
\begin{itemize}
    \item \texttt{site1}: Risultati completi dell'analisi del primo sito
    \item \texttt{site2}: Risultati completi dell'analisi del secondo sito
    \item \texttt{winner}: Indicatore del sito vincitore (``site1'', ``site2'' o ``tie'')
    \item \texttt{comparison}: Risultati dettagliati del confronto
    \item \texttt{timestamp}: Data e ora dell'analisi
\end{itemize}

\textbf{Metodi}:
\begin{itemize}
    \item \texttt{determineWinner()}: Determina il vincitore in base ai punteggi finali
    \item \texttt{getWinningCategories(site)}: Ottiene le categorie in cui un sito ha ottenuto i punteggi più alti
    \item \texttt{toJSON()}: Serializza i risultati in formato JSON
\end{itemize}

\subsection{SiteAnalysis}
\textbf{Responsabilità}: Contiene i dati di analisi completi per un singolo sito web.

\textbf{Attributi}:
\begin{itemize}
    \item \texttt{url}: URL del sito analizzato
    \item \texttt{performance}: Metriche di performance
    \item \texttt{seo}: Metriche SEO
    \item \texttt{security}: Metriche di sicurezza
    \item \texttt{technical}: Metriche tecniche
    \item \texttt{finalScore}: Punteggio finale ponderato
\end{itemize}

\textbf{Metodi}:
\begin{itemize}
    \item \texttt{calculateFinalScore()}: Calcola il punteggio finale ponderato
    \item \texttt{getHighestScoringCategory()}: Restituisce la categoria con il punteggio più alto
    \item \texttt{getLowestScoringCategory()}: Restituisce la categoria con il punteggio più basso
\end{itemize}

\subsection{PerformanceMetrics}
\textbf{Responsabilità}: Contiene le metriche di performance di un sito web.

\textbf{Attributi}:
\begin{itemize}
    \item \texttt{score}: Punteggio complessivo di performance (0-100)
    \item \texttt{metrics}: Valori specifici delle metriche
    \begin{itemize}
        \item \texttt{fcp}: First Contentful Paint (ms)
        \item \texttt{lcp}: Largest Contentful Paint (ms)
        \item \texttt{tti}: Time to Interactive (ms)
        \item \texttt{cls}: Cumulative Layout Shift
        \item \texttt{totalSize}: Dimensione totale della pagina (bytes)
    \end{itemize}
    \item \texttt{details}: Dettagli aggiuntivi sulla performance
\end{itemize}

\textbf{Metodi}:
\begin{itemize}
    \item \texttt{getScore()}: Restituisce il punteggio complessivo
    \item \texttt{getMetrics()}: Restituisce le metriche specifiche
    \item \texttt{getTiming()}: Restituisce i tempi di caricamento aggregati
\end{itemize}

\subsection{SEOMetrics}
\textbf{Responsabilità}: Contiene le metriche SEO di un sito web.

\textbf{Attributi}:
\begin{itemize}
    \item \texttt{score}: Punteggio complessivo SEO (0-100)
    \item \texttt{metrics}: Valori specifici delle metriche
    \begin{itemize}
        \item \texttt{title}: Valutazione del titolo
        \item \texttt{meta}: Valutazione dei meta tag
        \item \texttt{headings}: Valutazione della struttura dei titoli
        \item \texttt{images}: Valutazione delle immagini
        \item \texttt{links}: Valutazione dei link
    \end{itemize}
    \item \texttt{details}: Dettagli aggiuntivi SEO
\end{itemize}

\textbf{Metodi}:
\begin{itemize}
    \item \texttt{getScore()}: Restituisce il punteggio complessivo
    \item \texttt{getMetrics()}: Restituisce le metriche specifiche
    \item \texttt{getSEOIssues()}: Restituisce i problemi SEO rilevati
\end{itemize}

\subsection{SecurityMetrics}
\textbf{Responsabilità}: Contiene le metriche di sicurezza di un sito web.

\textbf{Attributi}:
\begin{itemize}
    \item \texttt{score}: Punteggio complessivo di sicurezza (0-100)
    \item \texttt{metrics}: Valori specifici delle metriche
    \begin{itemize}
        \item \texttt{ssl}: Valutazione SSL/TLS
        \item \texttt{headers}: Valutazione degli header di sicurezza
        \item \texttt{vulnerabilities}: Numero di vulnerabilità rilevate
        \item \texttt{cookies}: Valutazione della sicurezza dei cookies
        \item \texttt{csp}: Valutazione della Content Security Policy
    \end{itemize}
    \item \texttt{details}: Dettagli aggiuntivi sulla sicurezza
\end{itemize}

\textbf{Metodi}:
\begin{itemize}
    \item \texttt{getScore()}: Restituisce il punteggio complessivo
    \item \texttt{getMetrics()}: Restituisce le metriche specifiche
    \item \texttt{getVulnCount()}: Restituisce il numero di vulnerabilità
\end{itemize}

\subsection{TechnicalMetrics}
\textbf{Responsabilità}: Contiene le metriche tecniche di un sito web.

\textbf{Attributi}:
\begin{itemize}
    \item \texttt{score}: Punteggio complessivo tecnico (0-100)
    \item \texttt{metrics}: Valori specifici delle metriche
    \begin{itemize}
        \item \texttt{html}: Valutazione HTML
        \item \texttt{css}: Valutazione CSS
        \item \texttt{javascript}: Valutazione JavaScript
        \item \texttt{responsive}: Flag di responsività
        \item \texttt{technologies}: Array di tecnologie rilevate
    \end{itemize}
    \item \texttt{details}: Dettagli aggiuntivi tecnici
\end{itemize}

\textbf{Metodi}:
\begin{itemize}
    \item \texttt{getScore()}: Restituisce il punteggio complessivo
    \item \texttt{getMetrics()}: Restituisce le metriche specifiche
    \item \texttt{getTechStack()}: Restituisce lo stack tecnologico completo
\end{itemize}

\subsection{ComparisonResult}
\textbf{Responsabilità}: Contiene i risultati del confronto tra i due siti web.

\textbf{Attributi}:
\begin{itemize}
    \item \texttt{performance}: Vincitore per la categoria performance
    \item \texttt{seo}: Vincitore per la categoria SEO
    \item \texttt{security}: Vincitore per la categoria sicurezza
    \item \texttt{technical}: Vincitore per la categoria tecnica
    \item \texttt{overallScore}: Vincitore complessivo
    \item \texttt{scoreGap}: Differenza tra i punteggi finali
\end{itemize}

\textbf{Metodi}:
\begin{itemize}
    \item \texttt{getWinningCategories(site)}: Restituisce le categorie vinte da un sito
    \item \texttt{isCloseDuel()}: Determina se il confronto è stato equilibrato
\end{itemize}

\section{Diagramma di Ereditarietà - Metriche}
Il sistema utilizza l'ereditarietà per organizzare le diverse metriche in una gerarchia coerente:

```mermaid
classDiagram
    BaseMetrics <|-- PerformanceMetrics
    BaseMetrics <|-- SEOMetrics
    BaseMetrics <|-- SecurityMetrics
    BaseMetrics <|-- TechnicalMetrics
    BaseMetrics <|-- AccessibilityMetrics
    
    class BaseMetrics {
        +score: number
        +metrics: object
        +details: object
        +getScore() number
        +getMetrics() object
        +getDetails() object
    }
```

\section{Diagramma dei Punteggi}
Il calcolo dei punteggi finali si basa su una ponderazione delle diverse categorie di analisi:

```mermaid
flowchart TD
    Performance["Performance Score"] -- "30%" --> FinalScore["Final Score"]
    SEO["SEO Score"] -- "25%" --> FinalScore
    Security["Security Score"] -- "25%" --> FinalScore
    Technical["Technical Score"] -- "20%" --> FinalScore
    Additional["Additional Metrics"] --> FinalScore
```

\section{Diagramma di Persistenza}
I dati generati dal sistema possono essere persistiti in diversi formati:

```mermaid
flowchart TD
    AnalysisResults["Analysis Results"] -- "stored as" --> JSONFiles["JSON Files"]
    JSONFiles <-- "cached in" --> MemoryCache["Memory Cache"]
    JSONFiles -- "exported as" --> CSVReports["CSV Reports"]
    JSONFiles -- "exported as" --> PDFReports["PDF Reports"]
```

\section{Specifica delle Interfacce dei Dati}

\subsection{Interfaccia AnalysisResult}
\begin{asciiart}
interface AnalysisResult {
  site1: SiteAnalysis;
  site2: SiteAnalysis;
  winner: 'site1' | 'site2' | 'tie';
  comparison: ComparisonResult;
  timestamp: number;
}
\end{asciiart}

\subsection{Interfaccia SiteAnalysis}
\begin{asciiart}
interface SiteAnalysis {
  url: string;
  performance: PerformanceMetrics;
  seo: SEOMetrics;
  security: SecurityMetrics;
  technical: TechnicalMetrics;
  finalScore: number;
}
\end{asciiart}

\subsection{Interfaccia BaseMetrics}
\begin{asciiart}
interface BaseMetrics {
  score: number;
  metrics: Record<string, any>;
  details: Record<string, any>;
}
\end{asciiart}

\subsection{Interfaccia PerformanceMetrics}
\begin{asciiart}
interface PerformanceMetrics extends BaseMetrics {
  metrics: {
    fcp: number;
    lcp: number;
    tti: number;
    cls: number;
    totalSize: number;
  };
  details: {
    resources: {
      js: number;
      css: number;
      images: number;
      fonts: number;
    };
    resourceCount: number;
    renderBlocking: number;
  };
}
\end{asciiart}

\subsection{Interfaccia SEOMetrics}
\begin{asciiart}
interface SEOMetrics extends BaseMetrics {
  metrics: {
    title: string;
    meta: string;
    headings: string;
    images: string;
    links: string;
  };
  details: {
    titleLength: number;
    metaDescription: number;
    hasCanonical: boolean;
    imgAltTags: number;
    headingStructure: string;
    linkQuality: string;
  };
}
\end{asciiart}

\subsection{Interfaccia SecurityMetrics}
\begin{asciiart}
interface SecurityMetrics extends BaseMetrics {
  metrics: {
    ssl: string;
    headers: string;
    vulnerabilities: number;
    cookies: string;
    csp: string;
  };
  details: {
    sslDetails: {
      grade: string;
      protocol: string;
      expiry: string;
    };
    securityHeaders: {
      strictTransportSecurity: boolean;
      contentSecurityPolicy: boolean;
      xFrameOptions: boolean;
      xContentTypeOptions: boolean;
      referrerPolicy: boolean;
    };
    cookiesDetails: {
      secure: boolean;
      httpOnly: boolean;
      sameSite: string;
    };
  };
}
\end{asciiart}

\subsection{Interfaccia TechnicalMetrics}
\begin{asciiart}
interface TechnicalMetrics extends BaseMetrics {
  metrics: {
    html: string;
    css: string;
    javascript: string;
    responsive: boolean;
    technologies: string[];
  };
  details: {
    htmlVersion: string;
    cssVersion: string;
    jsFeatures: string[];
    frameworks: string[];
    libraries: string[];
    serverInfo: {
      server: string;
      platform: string;
    };
  };
}
\end{asciiart}

\subsection{Interfaccia ComparisonResult}
\begin{asciiart}
interface ComparisonResult {
  performance: 'site1' | 'site2' | 'tie';
  seo: 'site1' | 'site2' | 'tie';
  security: 'site1' | 'site2' | 'tie';
  technical: 'site1' | 'site2' | 'tie';
  overallScore: 'site1' | 'site2' | 'tie';
  scoreGap: number;
}
\end{asciiart}

\section{Normalizzazione dei Dati}

\subsection{Diagramma di Normalizzazione}

```mermaid
flowchart LR
    RawData["Raw Data"] --> Validation["Validation"]
    Validation --> TypeConversion["Type Conversion"]
    TypeConversion --> ScaleAdjustment["Scale Adjustment"]
    ScaleAdjustment --> FinalProcessing["Final Processing"]
    FinalProcessing --> NormalizedData["Normalized Data"]
```

\subsection{Regole di Normalizzazione per Metriche Chiave}
\begin{table}[H]
\centering
\begin{tabular}{|l|l|l|c|}
\hline
\textbf{Metrica} & \textbf{Valore Raw} & \textbf{Valutazione} & \textbf{Punteggio} \\
\hline
FCP & < 1000ms & Eccellente & 90-100 \\
\hline
FCP & 1000-2500ms & Buono & 60-89 \\
\hline
FCP & 2500-4000ms & Migliorabile & 30-59 \\
\hline
FCP & > 4000ms & Scarso & 0-29 \\
\hline
LCP & < 2500ms & Eccellente & 90-100 \\
\hline
LCP & 2500-4000ms & Buono & 60-89 \\
\hline
LCP & 4000-6000ms & Migliorabile & 30-59 \\
\hline
LCP & > 6000ms & Scarso & 0-29 \\
\hline
CLS & < 0.1 & Eccellente & 90-100 \\
\hline
CLS & 0.1-0.25 & Buono & 60-89 \\
\hline
CLS & 0.25-0.4 & Migliorabile & 30-59 \\
\hline
CLS & > 0.4 & Scarso & 0-29 \\
\hline
SSL Grade & A+ & Eccellente & 90-100 \\
\hline
SSL Grade & A & Molto Buono & 80-89 \\
\hline
SSL Grade & B & Buono & 70-79 \\
\hline
SSL Grade & C & Migliorabile & 50-69 \\
\hline
SSL Grade & F & Scarso & 0-49 \\
\hline
\end{tabular}
\caption{Normalizzazione delle metriche principali}
\label{table:normalization-rules}
\end{table}

\section{Diagramma di Relazioni tra Metriche}

```mermaid
flowchart TD
    Performance["Performance"] -- "affects" --> UserExperience["User Experience"]
    Technical["Technical"] --> UserExperience
    Accessibility["Accessibility"] --> UserExperience
    
    UserExperience -- "contributes to" --> OverallQuality["Overall Quality"]
    Security["Security"] --> OverallQuality
    SEO["SEO"] --> OverallQuality
```

\section{Diagramma della Rappresentazione Visiva dei Dati}

```mermaid
flowchart TD
    AnalysisResults["Analysis Results"] -- "format for charts" --> ChartData["Chart Data"]
    ChartData -- "render as" --> RadarChart["Radar Chart"]
    
    AnalysisResults -- "format for tables" --> TableData["Table Data"]
    TableData -- "render as" --> ComparisonTable["Comparison Table"]
    TableData -- "render as" --> ScoreGauges["Score Gauges"]
    
    ComparisonTable --> DetailedMetricsView["Detailed Metrics View"]
    ScoreGauges --> DetailedMetricsView
    
    AnalysisResults -- "format for details" --> DetailedData["Detailed Data"]
    DetailedData --> DetailedMetricsView
```
\chapter{Casi d'Uso}

\section{Panoramica dei Casi d'Uso}
In questo capitolo vengono descritti in dettaglio i casi d'uso principali del sistema Site War, illustrando le interazioni tra gli utenti e il sistema per le varie funzionalità offerte dall'applicazione. Attraverso questi casi d'uso, si ottiene una visione completa e dettagliata del funzionamento del sistema dal punto di vista dell'utente.

\section{Attori del Sistema}
Il sistema Site War interagisce con diversi tipi di utenti, ciascuno con esigenze e obiettivi specifici:

\begin{itemize}
    \item \textbf{Utente Generico}: Qualsiasi visitatore del sito Web War che desidera confrontare due siti web per curiosità o per scopi informativi.
    
    \item \textbf{Sviluppatore Web}: Utente che utilizza il sistema per confrontare il proprio sito con siti concorrenti, con l'obiettivo di identificare punti di forza e debolezza e migliorare il proprio lavoro.
    
    \item \textbf{Analista SEO/Performance}: Utente specializzato che utilizza il sistema per ottenere dati tecnici comparativi dettagliati su aspetti specifici dei siti web.
    
    \item \textbf{Penetration Tester}: Utente interessato agli aspetti di sicurezza dei siti web, che utilizza il sistema per identificare vulnerabilità e problemi di sicurezza.
\end{itemize}

\section{Diagramma dei Casi d'Uso}
Il diagramma seguente illustra i principali casi d'uso del sistema Site War e le loro relazioni:

```mermaid
flowchart TD
    UC1["UC1: Inserire URL"] --> UC2["UC2: Validare URL"]
    UC2 --> UC3["UC3: Avviare Analisi"]
    UC3 --> UC4["UC4: Visualizzare Animazioni"]
    UC4 --> UC5["UC5: Visualizzare Risultati"]
    UC5 --> UC6["UC6: Esplorare Dettagli"]
    UC6 --> UC7["UC7: Esportare Risultati"]
    UC5 --> UC8["UC8: Analizzare Vincitore"]
```

\section{Descrizioni Dettagliate dei Casi d'Uso}

\subsection{UC1: Inserire URL}

\subsubsection{Attori Principali}
Tutti gli utenti

\subsubsection{Pre-condizioni}
L'utente ha aperto il sito Site War

\subsubsection{Trigger}
L'utente desidera confrontare due siti web

\subsubsection{Scenario Principale di Successo}
\begin{enumerate}
    \item L'utente accede alla homepage di Site War
    \item Il sistema presenta un form con due campi per l'inserimento degli URL
    \item L'utente inserisce l'URL del primo sito nel campo ``Sito 1''
    \item L'utente inserisce l'URL del secondo sito nel campo ``Sito 2''
    \item L'utente fa clic sul pulsante ``Inizia la battaglia!''
    \item Il sistema procede con la validazione degli URL (UC2)
\end{enumerate}

\subsubsection{Estensioni}
\begin{itemize}
    \item 3-4a. L'utente inserisce un URL in formato non valido
    \begin{enumerate}
        \item Il sistema evidenzia il campo con errore
        \item Il sistema mostra un messaggio di errore specifico
        \item L'utente corregge l'input e riprova
    \end{enumerate}
    
    \item 5a. L'utente decide di cambiare uno degli URL inseriti
    \begin{enumerate}
        \item L'utente modifica uno o entrambi i campi URL
        \item L'utente fa clic sul pulsante ``Inizia la battaglia!''
    \end{enumerate}
\end{itemize}

\subsubsection{Requisiti Speciali}
\begin{itemize}
    \item Tempo di risposta: la validazione del formato URL deve essere immediata (client-side)
    \item I campi del form devono essere accessibili da tastiera
    \item Il form deve essere responsive per diversi dispositivi
\end{itemize}

\subsubsection{Frequenza}
Molto frequente (ogni analisi inizia da qui)

\subsection{UC2: Validare URL}

\subsubsection{Attori Principali}
Sistema

\subsubsection{Pre-condizioni}
L'utente ha inserito due URL e inviato il form

\subsubsection{Trigger}
Invio del form di input URL

\subsubsection{Scenario Principale di Successo}
\begin{enumerate}
    \item Il sistema verifica che entrambi gli URL siano in formato valido
    \item Il sistema verifica che entrambi i siti siano accessibili
    \item Il sistema utilizza l'AI per valutare la pertinenza del confronto tra i due siti
    \item L'AI conferma che i siti sono confrontabili
    \item Il sistema procede con l'avvio dell'analisi (UC3)
\end{enumerate}

\subsubsection{Estensioni}
\begin{itemize}
    \item 2a. Uno o entrambi i siti non sono accessibili
    \begin{enumerate}
        \item Il sistema mostra un messaggio di errore specifico
        \item L'utente viene invitato a verificare gli URL e riprovare
    \end{enumerate}
    
    \item 4a. L'AI determina che i siti non sono confrontabili
    \begin{enumerate}
        \item Il sistema mostra un messaggio che spiega perché i siti non sono confrontabili
        \item L'utente può scegliere di procedere comunque con l'analisi o modificare gli URL
    \end{enumerate}
    
    \item 4b. Il servizio AI non è disponibile
    \begin{enumerate}
        \item Il sistema procede con l'analisi senza la validazione di pertinenza
        \item Il sistema mostra un avviso che la validazione di pertinenza non è stata eseguita
    \end{enumerate}
\end{itemize}

\subsubsection{Requisiti Speciali}
\begin{itemize}
    \item Timeout: la validazione completa non deve superare i 5 secondi
    \item In caso di problemi con l'API AI, il sistema deve degradare in modo elegante
\end{itemize}

\subsubsection{Frequenza}
Molto frequente (ogni analisi richiede validazione)

\subsection{UC3: Avviare Analisi}

\subsubsection{Attori Principali}
Sistema

\subsubsection{Pre-condizioni}
Gli URL sono stati validati e sono confrontabili

\subsubsection{Trigger}
Completamento della validazione URL

\subsubsection{Scenario Principale di Successo}
\begin{enumerate}
    \item Il sistema mostra l'interfaccia di ``battaglia'' con i due siti
    \item Il sistema avvia le analisi lato client (DOM, Performance base, SEO base)
    \item Il sistema invia richieste al backend per le analisi più avanzate
    \item Il sistema aggiorna la visualizzazione dell'avanzamento dell'analisi
    \item Il backend elabora le richieste e comunica con API esterne quando necessario
    \item Il sistema riceve progressivamente i risultati e aggiorna l'interfaccia
    \item Al completamento di tutte le analisi, il sistema procede alla visualizzazione dei risultati (UC5)
\end{enumerate}

\subsubsection{Estensioni}
\begin{itemize}
    \item 3a. Errore di comunicazione con il backend
    \begin{enumerate}
        \item Il sistema mostra un messaggio di errore
        \item L'utente può riprovare l'analisi
    \end{enumerate}
    
    \item 5a. Errore di comunicazione con API esterne
    \begin{enumerate}
        \item Il sistema utilizza dati parziali o strategie di fallback
        \item Il sistema continua con analisi alternative disponibili
        \item I risultati saranno marcati come parziali
    \end{enumerate}
\end{itemize}

\subsubsection{Requisiti Speciali}
\begin{itemize}
    \item Performance: l'analisi completa deve essere completata entro 25 secondi
    \item Il sistema deve mostrare un indicatore di avanzamento accurato
    \item Le analisi devono avvenire in parallelo quando possibile
\end{itemize}

\subsubsection{Frequenza}
Molto frequente (ogni analisi passa per questa fase)

\subsection{UC4: Visualizzare Animazioni}

\subsubsection{Attori Principali}
Utente

\subsubsection{Pre-condizioni}
L'analisi è stata avviata

\subsubsection{Trigger}
Avanzamento dell'analisi

\subsubsection{Scenario Principale di Successo}
\begin{enumerate}
    \item Il sistema visualizza un'animazione iniziale che rappresenta i due siti come ``guerrieri''
    \item Man mano che l'analisi procede, il sistema aggiorna l'animazione con effetti visivi
    \item Le prestazioni relative dei siti influenzano l'animazione (es. il sito più veloce sembra ``attaccare'' l'altro)
    \item Al raggiungimento di soglie di progresso (25\%, 50\%, 75\%), l'animazione cambia fase
    \item Quando l'analisi è completa, l'animazione mostra l'effetto finale di ``vittoria''
    \item Il sistema passa alla visualizzazione dei risultati (UC5)
\end{enumerate}

\subsubsection{Estensioni}
\begin{itemize}
    \item 2a. L'utente usa un dispositivo a basse prestazioni
    \begin{enumerate}
        \item Il sistema rileva le capacità del dispositivo
        \item Il sistema mostra animazioni semplificate per garantire performance adeguate
    \end{enumerate}
\end{itemize}

\subsubsection{Requisiti Speciali}
\begin{itemize}
    \item Le animazioni devono essere fluide (60 fps)
    \item Il sistema deve adattare le animazioni in base alle capacità del dispositivo
    \item L'accessibilità deve essere garantita anche con animazioni attive
\end{itemize}

\subsubsection{Frequenza}
Molto frequente (ogni analisi include animazioni)

\subsection{UC5: Visualizzare Risultati}

\subsubsection{Attori Principali}
Utente

\subsubsection{Pre-condizioni}
L'analisi è stata completata

\subsubsection{Trigger}
Completamento dell'analisi

\subsubsection{Scenario Principale di Successo}
\begin{enumerate}
    \item Il sistema mostra una dashboard con la proclamazione del vincitore
    \item Il sistema visualizza i punteggi complessivi per entrambi i siti
    \item Il sistema mostra un confronto visivo delle principali categorie di analisi
    \item L'utente può visualizzare i dettagli di ogni categoria tramite tab
    \item L'utente può decidere di esplorare i dettagli specifici (UC6)
    \item L'utente può esportare i risultati (UC7)
\end{enumerate}

\subsubsection{Estensioni}
\begin{itemize}
    \item 1a. L'analisi ha prodotto risultati parziali
    \begin{enumerate}
        \item Il sistema mostra un avviso che alcuni dati potrebbero essere incompleti
        \item Il sistema indica quali analisi sono state completate con successo
    \end{enumerate}
    
    \item 4a. L'utente desidera confrontare una metrica specifica
    \begin{enumerate}
        \item L'utente seleziona la categoria desiderata
        \item Il sistema mostra una visualizzazione dettagliata per quella categoria
    \end{enumerate}
\end{itemize}

\subsubsection{Requisiti Speciali}
\begin{itemize}
    \item I risultati devono essere visualizzati in modo chiaro e intuitivo
    \item I grafici comparativi devono essere accessibili e comprensibili
    \item La dashboard deve essere responsive per diversi dispositivi
\end{itemize}

\subsubsection{Frequenza}
Molto frequente (ogni analisi completata)

\subsection{UC6: Esplorare Dettagli}

\subsubsection{Attori Principali}
Utente

\subsubsection{Pre-condizioni}
I risultati dell'analisi sono stati visualizzati

\subsubsection{Trigger}
L'utente desidera esplorare dettagli specifici

\subsubsection{Scenario Principale di Successo}
\begin{enumerate}
    \item L'utente fa clic su una categoria specifica (Performance, SEO, Sicurezza, Tecnica)
    \item Il sistema mostra una vista dettagliata con metriche specifiche per quella categoria
    \item Il sistema visualizza grafici comparativi per le metriche della categoria
    \item Il sistema evidenzia i punti di forza e debolezza di ciascun sito
    \item L'utente può navigare tra le diverse categorie utilizzando i tab
    \item L'utente può tornare alla vista generale dei risultati
\end{enumerate}

\subsubsection{Estensioni}
\begin{itemize}
    \item 2a. Dati insufficienti per la categoria selezionata
    \begin{enumerate}
        \item Il sistema mostra un messaggio che indica la mancanza di dati sufficienti
        \item Il sistema offre suggerimenti per analisi alternative
    \end{enumerate}
\end{itemize}

\subsubsection{Requisiti Speciali}
\begin{itemize}
    \item La navigazione tra categorie deve essere intuitiva
    \item I dettagli tecnici devono essere presentati in modo comprensibile
    \item Devono essere forniti suggerimenti per il miglioramento
\end{itemize}

\subsubsection{Frequenza}
Frequente (la maggior parte degli utenti esplora i dettagli)

\subsection{UC7: Esportare Risultati}

\subsubsection{Attori Principali}
Utente

\subsubsection{Pre-condizioni}
I risultati dell'analisi sono stati visualizzati

\subsubsection{Trigger}
L'utente desidera salvare o condividere i risultati

\subsubsection{Scenario Principale di Successo}
\begin{enumerate}
    \item L'utente fa clic sul pulsante ``Esporta risultati''
    \item Il sistema mostra un menu con opzioni di esportazione (CSV, PDF, Stampa)
    \item L'utente seleziona il formato desiderato
    \item Il sistema genera il file nel formato scelto
    \item Il browser avvia il download del file o apre l'anteprima di stampa
\end{enumerate}

\subsubsection{Estensioni}
\begin{itemize}
    \item 3a. L'utente sceglie l'opzione ``Stampa''
    \begin{enumerate}
        \item Il sistema prepara una versione ottimizzata per la stampa
        \item Il browser apre l'anteprima di stampa
    \end{enumerate}
    
    \item 4a. Errore nella generazione del file
    \begin{enumerate}
        \item Il sistema mostra un messaggio di errore
        \item L'utente può riprovare o scegliere un formato alternativo
    \end{enumerate}
\end{itemize}

\subsubsection{Requisiti Speciali}
\begin{itemize}
    \item I file esportati devono includere tutti i dati rilevanti
    \item I formati di esportazione devono essere standard e compatibili
    \item La versione stampabile deve essere ottimizzata per la carta
\end{itemize}

\subsubsection{Frequenza}
Occasionale (alcuni utenti esportano i risultati)

\subsection{UC8: Analizzare Vincitore}

\subsubsection{Attori Principali}
Utente

\subsubsection{Pre-condizioni}
I risultati dell'analisi sono stati visualizzati

\subsubsection{Trigger}
L'utente desidera comprendere i fattori che hanno determinato il vincitore

\subsubsection{Scenario Principale di Successo}
\begin{enumerate}
    \item L'utente fa clic sul badge del vincitore o su un pulsante ``Perché ha vinto?''
    \item Il sistema mostra una spiegazione dettagliata dei fattori chiave che hanno contribuito alla vittoria
    \item Il sistema evidenzia le principali differenze tra i due siti
    \item Il sistema fornisce suggerimenti su come il sito perdente potrebbe migliorare
    \item L'utente può navigare tra diverse aree di confronto
    \item L'utente può tornare alla vista principale dei risultati
\end{enumerate}

\subsubsection{Estensioni}
\begin{itemize}
    \item 2a. Il confronto è stato molto equilibrato
    \begin{enumerate}
        \item Il sistema spiega i fattori di desempate utilizzati
        \item Il sistema mostra quanto è stato ravvicinato il confronto
    \end{enumerate}
\end{itemize}

\subsubsection{Requisiti Speciali}
\begin{itemize}
    \item Le spiegazioni devono essere comprensibili anche per utenti non tecnici
    \item I suggerimenti di miglioramento devono essere pratici e attuabili
    \item La visualizzazione deve evidenziare chiaramente i punti di forza e debolezza
\end{itemize}

\subsubsection{Frequenza}
Frequente (molti utenti vogliono capire il risultato)

\section{Flussi di Interazione Principali}

\subsection{Flusso Base}
\begin{enumerate}
    \item L'utente inserisce gli URL dei due siti (UC1)
    \item Il sistema valida gli URL (UC2)
    \item Il sistema avvia l'analisi (UC3)
    \item L'utente visualizza le animazioni durante l'analisi (UC4)
    \item Il sistema mostra i risultati (UC5)
    \item L'utente esplora i dettagli (UC6)
    \item L'utente esporta i risultati (UC7)
\end{enumerate}

\subsection{Flusso Alternativo - Validazione Fallita}
\begin{enumerate}
    \item L'utente inserisce gli URL dei due siti (UC1)
    \item Il sistema determina che i siti non sono confrontabili (UC2)
    \item L'utente sceglie di procedere comunque
    \item Il sistema avvia l'analisi con avviso (UC3)
    \item Il flusso continua come nel flusso base
\end{enumerate}

\subsection{Flusso Alternativo - Analisi Parziale}
\begin{enumerate}
    \item L'utente inserisce gli URL dei due siti (UC1)
    \item Il sistema valida gli URL (UC2)
    \item Il sistema avvia l'analisi (UC3)
    \item Alcune analisi esterne falliscono
    \item Il sistema mostra risultati parziali con avviso (UC5)
    \item L'utente può esplorare i dati disponibili (UC6)
\end{enumerate}

\section{Requisiti Non Funzionali dei Casi d'Uso}

\subsection{Performance}
\begin{itemize}
    \item UC3 (Avviare Analisi): Completamento entro 25 secondi per l'analisi completa
    \item UC4 (Visualizzare Animazioni): 60 fps per le animazioni, degradando su dispositivi meno potenti
    \item UC2 (Validare URL): Validazione completa entro 5 secondi
\end{itemize}

\subsection{Usabilità}
\begin{itemize}
    \item UC1 (Inserire URL): Form semplice e intuitivo, accessibile da tastiera
    \item UC5 (Visualizzare Risultati): Visualizzazione chiara e comprensibile dei dati tecnici
    \item UC6 (Esplorare Dettagli): Navigazione intuitiva tra le categorie
\end{itemize}

\subsection{Sicurezza}
\begin{itemize}
    \item UC2 (Validare URL): Sanitizzazione degli input per prevenire attacchi
    \item UC3 (Avviare Analisi): Protezione delle chiavi API dai client
    \item UC7 (Esportare Risultati): Prevenzione di data leakage nei file esportati
\end{itemize}

\subsection{Scalabilità}
\begin{itemize}
    \item UC3 (Avviare Analisi): Gestione parallela di multiple richieste di analisi
    \item UC2 (Validare URL): Cache dei risultati di validazione per URL frequenti
\end{itemize}

\section{Diagramma di Stato del Processo di Analisi}
Il diagramma seguente illustra gli stati attraverso cui passa il sistema durante il processo di analisi:

```mermaid
stateDiagram-v2
    [*] --> Iniziale
    Iniziale --> URLInseriti: inserimento
    URLInseriti --> Validazione: invio
    Validazione --> AnalisiInCorso: validazione riuscita
    AnalisiInCorso --> RisultatiVisualizzati: analisi completata
    RisultatiVisualizzati --> NuovaAnalisi: reset
    NuovaAnalisi --> Iniziale: nuova operazione
    Validazione --> URLInseriti: errore validazione
```

Il diagramma di stato mostra come il sistema passa attraverso diversi stati durante il processo di analisi:

\begin{enumerate}
    \item \textbf{Iniziale}: Lo stato di partenza quando l'utente accede al sistema
    \item \textbf{URL Inseriti}: L'utente ha inserito gli URL ma non ha ancora avviato l'analisi
    \item \textbf{Validazione}: Il sistema sta verificando la validità e la pertinenza degli URL
    \item \textbf{Analisi In Corso}: Il sistema sta eseguendo le analisi sui siti web
    \item \textbf{Risultati Visualizzati}: Il sistema mostra i risultati all'utente
    \item \textbf{Nuova Analisi}: L'utente decide di avviare una nuova analisi
\end{enumerate}

\section{Diagramma di Attività - Analisi Completa}
Il diagramma seguente illustra il flusso di attività complete durante un processo di analisi:

```mermaid
flowchart TD
    InserireURL["Inserire URL"] --> ValidareURL["Validare URL"]
    ValidareURL --> VerificarePertinenza["Verificare Pertinenza"]
    VerificarePertinenza --> AvviareAnalisi["Avviare Analisi"]
    
    VerificarePertinenza --> EPertinente{"È Pertinente?"}
    EPertinente -- "No" --> MostrareAvviso["Mostrare Avviso"]
    EPertinente -- "Sì" --> EseguireAnalisiClient["Eseguire Analisi Lato Client"]
    
    MostrareAvviso --> ContinuaComunque["Continua Comunque"]
    ContinuaComunque --> EseguireAnalisiClient
    
    EseguireAnalisiClient --> EseguireAnalisiServer["Eseguire Analisi Lato Server"]
    EseguireAnalisiServer --> CombinareRisultati["Combinare Risultati"]
    CombinareRisultati --> DeterminareVincitore["Determinare Vincitore"]
    DeterminareVincitore --> MostrareRisultati["Mostrare Risultati"]
```

\section{Matrice Casi d'Uso-Requisiti}
La tabella seguente mappa i casi d'uso ai requisiti funzionali e non funzionali del sistema, mostrando quali requisiti sono soddisfatti da ciascun caso d'uso:

\begin{table}[H]
\centering
\begin{tabular}{|l|c|c|c|c|c|c|}
\hline
\textbf{Caso d'Uso} & \textbf{Performance} & \textbf{Usabilità} & \textbf{Accessibilità} & \textbf{Sicurezza} & \textbf{Accuratezza} & \textbf{Responsività} \\
\hline
UC1: Inserire URL & & \checkmark & \checkmark & \checkmark & & \checkmark \\
\hline
UC2: Validare URL & \checkmark & & & \checkmark & \checkmark & \\
\hline
UC3: Avviare Analisi & \checkmark & & & \checkmark & \checkmark & \\
\hline
UC4: Visualizzare Animazioni & \checkmark & \checkmark & \checkmark & & & \checkmark \\
\hline
UC5: Visualizzare Risultati & & \checkmark & \checkmark & & \checkmark & \checkmark \\
\hline
UC6: Esplorare Dettagli & & \checkmark & \checkmark & & \checkmark & \checkmark \\
\hline
UC7: Esportare Risultati & & \checkmark & & \checkmark & \checkmark & \\
\hline
UC8: Analizzare Vincitore & & \checkmark & \checkmark & & \checkmark & \\
\hline
\end{tabular}
\caption{Matrice Casi d'Uso-Requisiti}
\label{table:use-case-requirements}
\end{table}

\section{Analisi dei Rischi per i Casi d'Uso}
L'analisi dei rischi identifica i potenziali problemi che potrebbero verificarsi durante l'esecuzione dei casi d'uso e le strategie per mitigarli:

\begin{table}[H]
\centering
\begin{tabular}{|l|l|l|l|}
\hline
\textbf{Caso d'Uso} & \textbf{Rischio} & \textbf{Impatto} & \textbf{Strategia di Mitigazione} \\
\hline
UC2: Validare URL & API AI non disponibile & Medio & Procedere senza validazione di pertinenza \\
\hline
UC3: Avviare Analisi & Timeout API esterne & Alto & Implementare strategie di fallback e cache \\
\hline
UC3: Avviare Analisi & Rate limiting API & Alto & Implementare code e prioritizzazione \\
\hline
UC4: Visualizzare Animazioni & Performance browser limitata & Medio & Degrado graceful delle animazioni \\
\hline
UC5: Visualizzare Risultati & Dati incompleti & Alto & Mostrare avvisi e risultati parziali \\
\hline
UC7: Esportare Risultati & Errore generazione file & Basso & Offrire formati alternativi e retry \\
\hline
\end{tabular}
\caption{Analisi dei Rischi per i Casi d'Uso}
\label{table:use-case-risks}
\end{table}

\section{Considerazioni sull'Esperienza Utente}
I casi d'uso sono stati progettati per garantire un'esperienza utente ottimale, considerando diversi aspetti:

\begin{itemize}
    \item \textbf{Feedback Continuo}: Durante il processo di analisi, l'utente riceve feedback costante attraverso animazioni e indicatori di avanzamento.
    
    \item \textbf{Degradazione Elegante}: In caso di errori o limitazioni, il sistema offre alternative e continua a funzionare con capacità ridotte invece di fallire completamente.
    
    \item \textbf{Chiarezza dei Risultati}: I risultati sono presentati in modo chiaro e comprensibile, con diversi livelli di dettaglio per soddisfare utenti con diverse esigenze.
    
    \item \textbf{Accessibilità}: Tutti i casi d'uso considerano le esigenze di accessibilità, garantendo che l'applicazione sia utilizzabile da utenti con diverse capacità.
    
    \item \textbf{Apprendimento Progressivo}: L'interfaccia è progettata per consentire un apprendimento progressivo, con informazioni di base immediatamente comprensibili e dettagli tecnici accessibili su richiesta.
\end{itemize}
\chapter{Sviluppo e Deployment}

\section{Ambiente di Sviluppo}
L'ambiente di sviluppo di Site War è configurato per garantire efficienza, coerenza e qualità durante tutto il processo di sviluppo. Questo capitolo descrive in dettaglio gli strumenti, le configurazioni e le pratiche da seguire durante lo sviluppo del progetto.

\subsection{Requisiti Software}
Per lo sviluppo di Site War sono necessari i seguenti strumenti:

\begin{itemize}
    \item Editor di testo o IDE (es. Visual Studio Code, Sublime Text)
    \item Browser moderni (Chrome, Firefox, Safari, Edge)
    \item PHP 7.4 o superiore
    \item Server web (es. Apache, Nginx)
    \item MySQL o altro database compatibile
    \item Node.js e npm per la gestione delle dipendenze frontend
    \item Strumenti di versionamento (es. Git)
\end{itemize}

\subsection{Configurazione dell'Ambiente}
Di seguito sono riportate le istruzioni per configurare l'ambiente di sviluppo:

\begin{enumerate}
    \item \textbf{Installazione del server locale}:
    \begin{itemize}
        \item Installare XAMPP/WAMP/MAMP con PHP 7.4 o superiore
        \item Configurare un virtual host per il progetto
        \item Impostare i permessi corretti sulle directory
    \end{itemize}
    
    \item \textbf{Configurazione del repository Git}:
    \begin{itemize}
        \item Clonare il repository da GitHub
        \item Configurare i git hooks per lint pre-commit
        \item Impostare .gitignore per escludere file di configurazione locali e cache
    \end{itemize}
    
    \item \textbf{Installazione delle dipendenze}:
    \begin{itemize}
        \item Eseguire \texttt{npm install} per le dipendenze frontend
        \item Eseguire \texttt{composer install} per eventuali dipendenze PHP
    \end{itemize}
    
    \item \textbf{Configurazione delle API Key}:
    \begin{itemize}
        \item Copiare il file \texttt{config/api\_keys.sample.php} in \texttt{config/api\_keys.php}
        \item Ottenere le chiavi API per i servizi richiesti e inserirle nel file
    \end{itemize}
\end{enumerate}

%\subsection{Struttura di Directory}
%Il progetto segue la seguente struttura di directory:
%
%\begin{verbatim}
%site-war/
%│
%├── assets/                   # Risorse statiche
%│   ├── css/                  # Fogli di stile
%│   │   ├── main.css          # Stile principale
%│   │   ├── animations.css    # Animazioni della "guerra"
%│   │   ├── components/       # Stili per componenti specifici
%│   │   └── vendors/          # CSS di terze parti (Bootstrap)
%│   │
%│   ├── js/                   # JavaScript
%│   │   ├── main.js           # Entry point
%│   │   ├── modules/          # Moduli funzionali
%│   │   │   ├── analyzers/    # Moduli di analisi
%│   │   │   ├── ui/           # Componenti UI
%│   │   │   ├── core/         # Funzionalità core
%│   │   │   └── comparison/   # Logica di confronto
%│   │   │
%│   │   └── vendors/          # Librerie di terze parti
%│   │
%│   └── images/               # Immagini e icone
%│
%├── server/                   # Backend PHP
%│   ├── api/                  # Endpoint API
%│   ├── config/               # Configurazioni
%│   ├── core/                 # Funzionalità core
%│   ├── services/             # Servizi di business logic
%│   └── utils/                # Utility functions
%│
%├── templates/                # Template HTML modulari
%│   ├── components/           # Componenti riutilizzabili
%│   └── pages/                # Template pagine
%│
%├── tests/                    # Test unitari e funzionali
%│
%├── index.php                 # Entry point applicazione
%├── .htaccess                 # Configurazione Apache
%└── README.md                 # Documentazione
%\end{verbatim}

\section{Linee Guida per lo Sviluppo}

\subsection{Standard di Codifica}
Per garantire la manutenibilità e la leggibilità del codice, tutti gli sviluppatori devono seguire questi standard:

\subsubsection{PHP}
\begin{itemize}
    \item Seguire le PSR-1 e PSR-12 per lo stile del codice
    \item Utilizzare la documentazione PHPDoc per tutte le classi e i metodi
    \item Evitare l'uso di variabili globali
    \item Limitare l'uso di costrutti basati su stringhe come \texttt{eval()}
    \item Usare costanti per valori fissi
\end{itemize}

\subsubsection{JavaScript}
\begin{itemize}
    \item Seguire lo standard Airbnb JavaScript Style Guide
    \item Usare il Module Pattern per organizzare il codice
    \item Documentare le funzioni con JSDoc
    \item Preferire l'uso di funzioni anonime con arrow functions
    \item Usare sempre \texttt{'use strict'}
\end{itemize}

\subsubsection{HTML/CSS}
\begin{itemize}
    \item Seguire le linee guida HTML5
    \item Usare la convenzione BEM per i nomi delle classi CSS
    \item Strutturare i fogli di stile in modo modulare
    \item Garantire la validazione W3C
    \item Mantenere la separazione tra struttura (HTML), presentazione (CSS) e comportamento (JS)
\end{itemize}

\subsection{Best Practices}
\begin{itemize}
    \item \textbf{Security by Design}: Implementare misure di sicurezza fin dall'inizio
    \item \textbf{Defensive Programming}: Validare tutti gli input e gestire tutti i possibili errori
    \item \textbf{DRY (Don't Repeat Yourself)}: Evitare la duplicazione del codice
    \item \textbf{KISS (Keep It Simple, Stupid)}: Preferire soluzioni semplici e chiare
    \item \textbf{Progressive Enhancement}: Garantire funzionalità di base per tutti gli utenti
\end{itemize}

\subsection{Processo di Revisione del Codice}
Tutto il codice deve passare attraverso un processo di revisione prima di essere integrato nel branch principale:

\begin{enumerate}
    \item Lo sviluppatore crea un branch feature/fix
    \item Lo sviluppatore implementa le modifiche e esegue i test locali
    \item Viene creata una Pull Request
    \item Almeno un altro sviluppatore revisiona il codice
    \item I test automatici vengono eseguiti sulla Pull Request
    \item Dopo l'approvazione e il passaggio dei test, il codice viene integrato
\end{enumerate}

\section{Testing}

\subsection{Tipi di Test}
Il progetto prevede diversi livelli di testing per garantire la qualità del software:

\begin{itemize}
    \item \textbf{Test Unitari}: Verifica delle singole unità di codice
    \item \textbf{Test di Integrazione}: Verifica dell'interazione tra diversi moduli
    \item \textbf{Test Funzionali}: Verifica del comportamento dell'applicazione dal punto di vista dell'utente
    \item \textbf{Test di Performance}: Verifica delle prestazioni del sistema
    \item \textbf{Test di Compatibilità}: Verifica del funzionamento sui diversi browser e dispositivi
    \item \textbf{Test di Accessibilità}: Verifica della conformità agli standard WCAG 2.1 AA
    \item \textbf{Test di Sicurezza}: Verifica della resistenza a vulnerabilità comuni
\end{itemize}

\subsection{Strumenti di Testing}
\begin{itemize}
    \item \textbf{PHPUnit}: Per i test unitari PHP
    \item \textbf{Jest}: Per i test unitari JavaScript
    \item \textbf{Cypress}: Per i test funzionali end-to-end
    \item \textbf{Lighthouse}: Per i test di performance
    \item \textbf{BrowserStack}: Per i test di compatibilità cross-browser
    \item \textbf{WAVE}: Per i test di accessibilità
    \item \textbf{OWASP ZAP}: Per i test di sicurezza automatizzati
\end{itemize}

\subsection{Strategia di Testing}
\begin{itemize}
    \item \textbf{Test-Driven Development (TDD)}: Per componenti critici
    \item \textbf{Continuous Testing}: Test automatici eseguiti ad ogni commit
    \item \textbf{Test Coverage}: Mirare a una copertura di test del 80\% per il codice core
    \item \textbf{Smoke Testing}: Verifiche rapide sui principali flussi utente
    \item \textbf{Regression Testing}: Verifica che le nuove modifiche non rompano funzionalità esistenti
\end{itemize}

\section{Continuous Integration e Continuous Deployment}

\subsection{Workflow CI/CD}
Il progetto utilizza un workflow CI/CD completo per automatizzare il processo di integrazione e deployment:

```mermaid
flowchart LR
    DeveloperChange["Developer Change"] --> GitCommit["Git Commit"]
    GitCommit --> AutomatedTests["Automated Tests"]
    AutomatedTests --> CodeReview["Code Review"]
    CodeReview --> MergeToMainBranch["Merge to Main Branch"]
    MergeToMainBranch --> StagingDeployment["Staging Deployment"]
    StagingDeployment --> LiveTesting["Live Testing"]
    LiveTesting --> ProductionDeployment["Production Deployment"]
```

\subsection{Configurazione CI/CD}
\begin{itemize}
    \item \textbf{Jenkins}: Per l'automazione dei processi CI/CD
    \item \textbf{GitHub Actions}: Per l'integrazione con il repository
    \item \textbf{Docker}: Per la creazione di ambienti di test e deployment consistenti
    \item \textbf{Automated Testing}: Esecuzione automatica di tutti i test ad ogni commit
    \item \textbf{Deployment Automatico}: Deployment automatico agli ambienti di staging e produzione dopo il passaggio dei test
\end{itemize}

\section{Gestione delle Release}

\subsection{Versionamento}
Il progetto segue il versionamento semantico (SemVer):

\begin{itemize}
    \item \textbf{Major}: Cambiamenti incompatibili con le versioni precedenti
    \item \textbf{Minor}: Nuove funzionalità compatibili con le versioni precedenti
    \item \textbf{Patch}: Correzioni di bug compatibili con le versioni precedenti
\end{itemize}

\subsection{Processo di Release}
\begin{enumerate}
    \item Creazione di un branch \texttt{release/X.Y.Z}
    \item Esecuzione di test completi sul branch di release
    \item Aggiornamento di changelog e documentazione
    \item Merge nel branch \texttt{main} e tagging con la versione
    \item Deployment in produzione
    \item Comunicazione della release agli stakeholder
\end{enumerate}

\subsection{Hotfix}
Per correzioni urgenti in produzione:
\begin{enumerate}
    \item Creazione di un branch \texttt{hotfix/X.Y.Z+1}
    \item Implementazione e test della correzione
    \item Merge nei branch \texttt{main} e \texttt{develop}
    \item Deployment in produzione
\end{enumerate}

\section{Ambienti di Deployment}

\subsection{Ambienti Disponibili}
\begin{itemize}
    \item \textbf{Development}: Ambiente di sviluppo locale per ogni sviluppatore
    \item \textbf{Integration}: Ambiente condiviso per l'integrazione delle modifiche
    \item \textbf{Staging}: Ambiente di pre-produzione per test finali
    \item \textbf{Production}: Ambiente di produzione visibile agli utenti finali
\end{itemize}

\subsection{Requisiti Infrastrutturali}
\begin{table}[H]
\centering
\begin{tabular}{|l|l|l|l|}
\hline
\textbf{Risorsa} & \textbf{Sviluppo} & \textbf{Staging} & \textbf{Produzione} \\
\hline
Server Web & Apache/Nginx & Nginx & Nginx \\
\hline
PHP & 7.4+ & 7.4+ & 7.4+ \\
\hline
CPU & 2 core & 4 core & 8+ core \\
\hline
RAM & 4 GB & 8 GB & 16+ GB \\
\hline
Storage & 20 GB SSD & 50 GB SSD & 100+ GB SSD \\
\hline
Caching & Opzionale & Redis & Redis \\
\hline
CDN & No & Opzionale & Sì \\
\hline
SSL & Opzionale & Sì & Sì \\
\hline
\end{tabular}
\caption{Requisiti infrastrutturali per ambiente}
\label{table:infrastructure-requirements}
\end{table}

\subsection{Configurazione di Deployment}

```mermaid
flowchart TD
    subgraph ProductionDeployment["Production Deployment"]
        Frontend["Frontend (Nginx)"] <--> Backend["Backend (PHP-FPM)"]
        Backend <--> RedisCache["Redis Cache"]
        
        Frontend --> CDNAssets["CDN (Assets)"]
        Backend --> ExternalAPIs["External APIs"]
    end
```

\subsection{Gestione delle Configurazioni}
\begin{itemize}
    \item \textbf{Environment Variables}: Configurazioni specifiche per ambiente
    \item \textbf{Configuration Files}: File di configurazione per impostazioni non sensibili
    \item \textbf{Secret Management}: Gestione sicura di chiavi API e credenziali
    \item \textbf{Feature Flags}: Abilita/disabilita funzionalità per ambiente
\end{itemize}

\section{Monitoraggio e Logging}

\subsection{Sistema di Logging}
\begin{itemize}
    \item \textbf{Application Logs}: Log delle attività dell'applicazione
    \item \textbf{Error Logs}: Log degli errori e delle eccezioni
    \item \textbf{Access Logs}: Log degli accessi al sistema
    \item \textbf{Performance Logs}: Log delle metriche di performance
    \item \textbf{API Logs}: Log delle chiamate alle API esterne
\end{itemize}

\subsection{Monitoraggio}
\begin{itemize}
    \item \textbf{Uptime Monitoring}: Controllo della disponibilità del servizio
    \item \textbf{Performance Monitoring}: Monitoraggio delle prestazioni
    \item \textbf{Error Tracking}: Rilevamento e tracciamento degli errori
    \item \textbf{Resource Monitoring}: Monitoraggio dell'utilizzo delle risorse
    \item \textbf{User Analytics}: Analisi del comportamento degli utenti
\end{itemize}

\subsection{Alerting}
\begin{itemize}
    \item \textbf{Threshold Alerts}: Avvisi basati su soglie predefinite
    \item \textbf{Anomaly Detection}: Rilevamento di comportamenti anomali
    \item \textbf{Incident Management}: Processo di gestione degli incidenti
    \item \textbf{On-Call Rotation}: Rotazione del personale di supporto
\end{itemize}

\section{Documentazione}

\subsection{Tipi di Documentazione}
\begin{itemize}
    \item \textbf{Documentazione Tecnica}: Descrizione dettagliata dell'architettura e dei componenti
    \item \textbf{API Documentation}: Documentazione delle API esposte
    \item \textbf{Code Documentation}: Documentazione del codice sorgente
    \item \textbf{User Documentation}: Guide per gli utenti finali
    \item \textbf{Operations Documentation}: Procedure operative e di manutenzione
\end{itemize}

\subsection{Gestione della Documentazione}
\begin{itemize}
    \item \textbf{Documentation as Code}: La documentazione è trattata come codice e versionata
    \item \textbf{Automated Generation}: Generazione automatica della documentazione dall'API e dal codice
    \item \textbf{Review Process}: Processo di revisione per la documentazione
    \item \textbf{Update on Release}: Aggiornamento della documentazione ad ogni release
\end{itemize}

\section{Backup e Disaster Recovery}

\subsection{Strategia di Backup}
\begin{itemize}
    \item \textbf{Code Repository}: Il codice è versionato e disponibile nel repository Git
    \item \textbf{Configuration Backup}: Backup regolari delle configurazioni
    \item \textbf{Data Backup}: Backup dei dati generati dall'applicazione
    \item \textbf{Log Backup}: Archiviazione dei log per analisi future
\end{itemize}

\subsection{Disaster Recovery}
\begin{itemize}
    \item \textbf{Recovery Plan}: Piano dettagliato per il ripristino del servizio
    \item \textbf{Recovery Time Objective (RTO)}: Tempo massimo accettabile per il ripristino
    \item \textbf{Recovery Point Objective (RPO)}: Massima perdita di dati accettabile
    \item \textbf{Failover Mechanism}: Meccanismi di failover per garantire la continuità del servizio
    \item \textbf{Regular Testing}: Test regolari del piano di disaster recovery
\end{itemize}

\section{Manutenzione e Supporto}

\subsection{Manutenzione Regolare}
\begin{itemize}
    \item \textbf{Security Updates}: Aggiornamenti regolari per questioni di sicurezza
    \item \textbf{Dependency Updates}: Aggiornamento delle dipendenze
    \item \textbf{Performance Optimization}: Ottimizzazione continua delle performance
    \item \textbf{Code Refactoring}: Refactoring del codice per migliorarne la qualità
\end{itemize}

\subsection{Supporto}
\begin{itemize}
    \item \textbf{Issue Tracking}: Sistema di tracciamento dei problemi
    \item \textbf{Support Tiers}: Livelli di supporto in base alla criticità
    \item \textbf{Response Time}: Tempi di risposta garantiti per ciascun livello
    \item \textbf{User Feedback}: Raccolta e analisi del feedback degli utenti
\end{itemize}
\chapter{Sicurezza}

\section{Panoramica della Sicurezza}
La sicurezza è un aspetto fondamentale del sistema Site War, implementata a diversi livelli per proteggere sia i dati degli utenti che le risorse del sistema. Questo capitolo descrive in dettaglio le misure di sicurezza adottate, le potenziali vulnerabilità e le strategie di mitigazione implementate.

L'approccio alla sicurezza del sistema segue il principio di "Security by Design", integrando considerazioni sulla sicurezza fin dalle prime fasi di progettazione dell'architettura. Questo approccio proattivo permette di identificare e mitigare potenziali rischi prima che possano essere sfruttati.

\section{Modello di Sicurezza}
Il modello di sicurezza di Site War è strutturato in diversi livelli, ciascuno con specifiche misure di protezione:

```mermaid
flowchart TD
    subgraph SecurityArchitecture["Security Architecture"]
        ClientSideSecurity["Client-side Security"]
        APILayerSecurity["API Layer Security"]
        ServerSideSecurity["Server-side Security"]
    end
    
    ClientSideSecurity --> InputValidation["Input Valid.\nXSS Prevention\nCSP Implement.\nObfuscation"]
    APILayerSecurity --> RequestValidation["Request Valid.\nCSRF Protection\nRate Limiting\nAuthentication"]
    ServerSideSecurity --> APIKeyProtection["API Key Protect\nInput Sanitiz.\nError Handling\nHTTPS Enforce."]
```

\subsection{Principi di Sicurezza}
Il sistema Site War si basa sui seguenti principi di sicurezza:

\begin{itemize}
    \item \textbf{Defense in Depth}: Implementazione di multiple barriere di protezione
    \item \textbf{Principle of Least Privilege}: Concessione dei minimi privilegi necessari
    \item \textbf{Secure by Default}: Configurazioni sicure predefinite
    \item \textbf{Fail Securely}: In caso di errore, il sistema deve fallire in modo sicuro
    \item \textbf{Security Through Transparency}: Trasparenza nelle pratiche di sicurezza
\end{itemize}

\section{Sicurezza Client-Side}

\subsection{Validazione degli Input}
Tutti gli input dell'utente sono validati sia lato client che lato server:

\begin{itemize}
    \item \textbf{Validazione Sintattica}: Verifica del formato e della lunghezza
    \item \textbf{Validazione Semantica}: Verifica della coerenza e della logica
    \item \textbf{Validazione Contestuale}: Verifica basata sul contesto di utilizzo
    \item \textbf{Sanitizzazione}: Rimozione o escape di caratteri pericolosi
\end{itemize}

\subsection{Protezione da Cross-Site Scripting (XSS)}
\begin{itemize}
    \item \textbf{Output Encoding}: Codifica dell'output per prevenire l'esecuzione di script
    \item \textbf{Content Security Policy (CSP)}: Restrizione delle fonti di contenuto eseguibile
    \item \textbf{HTTP-only Cookies}: Prevenzione dell'accesso ai cookie tramite JavaScript
    \item \textbf{X-XSS-Protection}: Header HTTP per abilitare la protezione XSS dei browser
\end{itemize}

\subsection{Obfuscation del Codice JavaScript}
\begin{itemize}
    \item \textbf{Minificazione}: Rimozione di spazi, commenti e altri elementi non necessari
    \item \textbf{Compressione}: Riduzione delle dimensioni del codice
    \item \textbf{Offuscamento}: Trasformazione del codice per renderlo meno leggibile
    \item \textbf{Protezione Algoritmi}: Nascondere la logica di business critica
\end{itemize}

\section{Sicurezza API Layer}

\subsection{Autenticazione e Autorizzazione}
\begin{itemize}
    \item \textbf{Validazione della Richiesta}: Verifica dell'integrità e della validità della richiesta
    \item \textbf{Origin Validation}: Verifica dell'origine della richiesta
    \item \textbf{Rate Limiting}: Limitazione del numero di richieste per prevenire abusi
    \item \textbf{API Key Validation}: Verifica delle chiavi API per le richieste autenticate
\end{itemize}

\subsection{Protezione da Cross-Site Request Forgery (CSRF)}
\begin{itemize}
    \item \textbf{CSRF Tokens}: Utilizzo di token anti-CSRF per le richieste sensibili
    \item \textbf{Same-Site Cookies}: Utilizzo di cookie same-site per limitare le richieste cross-site
    \item \textbf{Referrer Policy}: Controllo delle informazioni di referrer inviate nelle richieste
\end{itemize}

\subsection{Sicurezza delle Comunicazioni}
\begin{itemize}
    \item \textbf{HTTPS}: Utilizzo di HTTPS per tutte le comunicazioni
    \item \textbf{HTTP Strict Transport Security (HSTS)}: Forzatura dell'uso di HTTPS
    \item \textbf{Certificate Pinning}: Verifica del certificato SSL/TLS
    \item \textbf{TLS 1.3}: Utilizzo della versione più recente e sicura di TLS
\end{itemize}

\section{Sicurezza Server-Side}

\subsection{Protezione delle Chiavi API}
Le chiavi API utilizzate per accedere ai servizi esterni sono protette con le seguenti misure:

\begin{itemize}
    \item \textbf{Storage Sicuro}: Le chiavi sono memorizzate in file di configurazione protetti
    \item \textbf{Path Traversal Protection}: Prevenzione dell'accesso a file di configurazione
    \item \textbf{Proxy Service}: Le richieste API vengono inoltrate tramite un servizio proxy
    \item \textbf{Key Rotation}: Rotazione periodica delle chiavi API
\end{itemize}

\subsection{Protezione da Injection}
\begin{itemize}
    \item \textbf{Prepared Statements}: Utilizzo di statement preparati per le query al database
    \item \textbf{Input Sanitization}: Sanitizzazione degli input per prevenire injection
    \item \textbf{Parameter Validation}: Validazione dei parametri delle richieste
    \item \textbf{Safe API}: Utilizzo di API sicure che prevengono l'injection
\end{itemize}

\subsection{Sicurezza dei File}
\begin{itemize}
    \item \textbf{File Type Validation}: Validazione del tipo di file
    \item \textbf{File Size Limits}: Limitazione della dimensione dei file
    \item \textbf{File Storage Isolation}: Isolamento dello storage dei file
    \item \textbf{File Access Control}: Controllo dell'accesso ai file
\end{itemize}

\section{Protezione della Privacy dei Dati}

\subsection{Minimizzazione dei Dati}
\begin{itemize}
    \item \textbf{Raccolta Minimale}: Raccolta solo dei dati necessari
    \item \textbf{Eliminazione Tempestiva}: Eliminazione dei dati non più necessari
    \item \textbf{Storage Temporaneo}: Utilizzo di storage temporaneo per dati transitori
\end{itemize}

\subsection{Conformità GDPR}
\begin{itemize}
    \item \textbf{Consenso Esplicito}: Ottenere il consenso esplicito dell'utente
    \item \textbf{Diritto all'Oblio}: Possibilità di cancellare i propri dati
    \item \textbf{Portabilità dei Dati}: Possibilità di esportare i propri dati
    \item \textbf{Privacy by Design}: Privacy integrata nella progettazione
\end{itemize}

\section{Analisi delle Minacce}

\subsection{Modello STRIDE}
L'analisi delle minacce è stata condotta utilizzando il modello STRIDE:

\begin{table}[H]
\centering
\begin{tabular}{|l|l|l|}
\hline
\textbf{Tipo di Minaccia} & \textbf{Descrizione} & \textbf{Contromisure} \\
\hline
Spoofing & Impersonare un'altra entità & Validazione delle richieste \\
\hline
Tampering & Modificare dati & Controlli di integrità, HTTPS \\
\hline
Repudiation & Negare un'azione & Logging, audit trail \\
\hline
Information Disclosure & Accesso non autorizzato a dati & Encryption, controllo accessi \\
\hline
Denial of Service & Interruzione del servizio & Rate limiting, failover \\
\hline
Elevation of Privilege & Ottenere privilegi non autorizzati & Principle of least privilege \\
\hline
\end{tabular}
\caption{Analisi delle minacce con modello STRIDE}
\label{table:stride-threats}
\end{table}

\subsection{Attack Vectors}
\begin{itemize}
    \item \textbf{Client-Side Attacks}: XSS, CSRF, Clickjacking
    \item \textbf{Network Attacks}: Man-in-the-Middle, DNS Spoofing
    \item \textbf{Server-Side Attacks}: Injection, Path Traversal
    \item \textbf{API Abuse}: Rate Limiting Bypass, API Key Theft
\end{itemize}

\section{Security Headers}
La seguente configurazione di security headers è implementata per migliorare la sicurezza del browser:

\begin{table}[H]
\centering
\begin{tabular}{|l|l|}
\hline
\textbf{Header} & \textbf{Valore/Scopo} \\
\hline
Content-Security-Policy & Definisce le fonti affidabili per l'esecuzione di script, stili, ecc. \\
\hline
X-Content-Type-Options & nosniff; Previene il MIME type sniffing \\
\hline
X-Frame-Options & DENY; Previene il clickjacking \\
\hline
X-XSS-Protection & 1; mode=block; Abilita la protezione XSS del browser \\
\hline
Strict-Transport-Security & max-age=31536000; Forza l'uso di HTTPS \\
\hline
Referrer-Policy & strict-origin-when-cross-origin; Controlla le informazioni di referrer \\
\hline
Feature-Policy & Limita l'accesso a funzionalità del browser \\
\hline
\end{tabular}
\caption{Security Headers}
\label{table:security-headers}
\end{table}

\section{Rate Limiting e Protezione da DoS}

\subsection{Strategia di Rate Limiting}
Per prevenire abusi e attacchi di tipo Denial of Service, il sistema implementa una strategia di rate limiting:

\begin{itemize}
    \item \textbf{IP-based Rate Limiting}: Limitazione del numero di richieste per IP
    \item \textbf{API-based Rate Limiting}: Limitazione del numero di richieste per API
    \item \textbf{Graduated Rate Limiting}: Limitazione crescente in base all'utilizzo
    \item \textbf{Token Bucket Algorithm}: Algoritmo flessibile per il rate limiting
\end{itemize}

\subsection{Protezione da DoS/DDoS}
\begin{itemize}
    \item \textbf{Traffic Filtering}: Filtraggio del traffico sospetto
    \item \textbf{Resource Allocation}: Allocazione delle risorse in base alla priorità
    \item \textbf{Connection Throttling}: Limitazione delle connessioni simultanee
    \item \textbf{CDN}: Utilizzo di CDN per assorbire il traffico
\end{itemize}

\section{Secure Coding Practices}

\subsection{Linee Guida per la Sicurezza del Codice}
\begin{itemize}
    \item \textbf{Input Validation}: Validare tutti gli input dell'utente
    \item \textbf{Output Encoding}: Codificare tutto l'output in base al contesto
    \item \textbf{Error Handling}: Gestire gli errori in modo sicuro
    \item \textbf{Authentication}: Implementare meccanismi di autenticazione robusti
    \item \textbf{Session Management}: Gestire le sessioni in modo sicuro
    \item \textbf{Access Control}: Implementare controlli di accesso adeguati
\end{itemize}

\subsection{Code Review di Sicurezza}
\begin{itemize}
    \item \textbf{Security Code Review}: Revisione del codice specifica per la sicurezza
    \item \textbf{Static Analysis}: Analisi statica del codice
    \item \textbf{Dynamic Analysis}: Analisi dinamica dell'applicazione
    \item \textbf{Peer Review}: Revisione del codice da parte di altri sviluppatori
\end{itemize}

\section{Incident Response}

\subsection{Piano di Risposta agli Incidenti}
In caso di incidente di sicurezza, il sistema prevede un piano di risposta strutturato:

\begin{enumerate}
    \item \textbf{Rilevamento}: Identificazione dell'incidente
    \item \textbf{Contenimento}: Limitazione dell'impatto dell'incidente
    \item \textbf{Eradicazione}: Rimozione della causa dell'incidente
    \item \textbf{Ripristino}: Ripristino del normale funzionamento
    \item \textbf{Analisi Post-Incidente}: Analisi delle cause e delle lezioni apprese
    \item \textbf{Miglioramento}: Implementazione di miglioramenti per prevenire incidenti simili
\end{enumerate}

\subsection{Reporting e Comunicazione}
\begin{itemize}
    \item \textbf{Internal Reporting}: Reportistica interna sugli incidenti
    \item \textbf{External Reporting}: Comunicazione con gli utenti e le autorità
    \item \textbf{Disclosure Policy}: Politica di divulgazione degli incidenti
    \item \textbf{Communication Templates}: Template per la comunicazione in caso di incidente
\end{itemize}

\section{Testing di Sicurezza}

\subsection{Metodologie di Test}
\begin{itemize}
    \item \textbf{SAST (Static Application Security Testing)}: Analisi statica del codice
    \item \textbf{DAST (Dynamic Application Security Testing)}: Analisi dinamica dell'applicazione
    \item \textbf{Penetration Testing}: Test di penetrazione manuali
    \item \textbf{Vulnerability Scanning}: Scansione automatica delle vulnerabilità
    \item \textbf{Security Code Review}: Revisione manuale del codice per problemi di sicurezza
\end{itemize}

\subsection{Strumenti di Test}
\begin{itemize}
    \item \textbf{OWASP ZAP}: Per il test automatico delle vulnerabilità web
    \item \textbf{Burp Suite}: Per il test manuale delle vulnerabilità web
    \item \textbf{SonarQube}: Per l'analisi statica del codice
    \item \textbf{Nmap}: Per la scansione delle porte e dei servizi
    \item \textbf{Metasploit}: Per il test di penetrazione avanzato
\end{itemize}

\section{Gestione delle Dipendenze}

\subsection{Sicurezza delle Dipendenze}
\begin{itemize}
    \item \textbf{Dependency Scanning}: Scansione delle dipendenze per vulnerabilità note
    \item \textbf{Version Pinning}: Fissare le versioni delle dipendenze
    \item \textbf{Regular Updates}: Aggiornamenti regolari delle dipendenze
    \item \textbf{Minimal Dependencies}: Utilizzo del minimo numero di dipendenze necessarie
\end{itemize}

\subsection{Supply Chain Security}
\begin{itemize}
    \item \textbf{Vendor Verification}: Verifica dei fornitori di software
    \item \textbf{Integrity Verification}: Verifica dell'integrità dei pacchetti
    \item \textbf{Provenance Tracking}: Tracciamento della provenienza del software
    \item \textbf{Build Reproducibility}: Riproducibilità del processo di build
\end{itemize}

\section{Monitoraggio e Logging di Sicurezza}

\subsection{Event Logging}
\begin{itemize}
    \item \textbf{Security Events}: Registrazione degli eventi di sicurezza
    \item \textbf{Access Logging}: Registrazione degli accessi
    \item \textbf{Change Logging}: Registrazione delle modifiche
    \item \textbf{Error Logging}: Registrazione degli errori
\end{itemize}

\subsection{Log Management}
\begin{itemize}
    \item \textbf{Log Rotation}: Rotazione dei log per gestire lo spazio
    \item \textbf{Log Retention}: Conservazione dei log per il periodo necessario
    \item \textbf{Log Protection}: Protezione dei log da manipolazioni
    \item \textbf{Log Analysis}: Analisi dei log per identificare pattern sospetti
\end{itemize}

\section{Compliance e Standard di Sicurezza}

\subsection{Compliance Framework}
\begin{itemize}
    \item \textbf{GDPR}: Regolamento generale sulla protezione dei dati
    \item \textbf{PCI DSS}: Standard di sicurezza per i dati delle carte di pagamento
    \item \textbf{ISO 27001}: Standard per la gestione della sicurezza delle informazioni
    \item \textbf{OWASP Top 10}: Le 10 vulnerabilità web più critiche
\end{itemize}

\subsection{Security Assessment}
\begin{itemize}
    \item \textbf{Regular Audits}: Audit regolari della sicurezza
    \item \textbf{Compliance Checks}: Verifiche di conformità
    \item \textbf{Risk Assessment}: Valutazione dei rischi
    \item \textbf{Gap Analysis}: Analisi delle lacune di sicurezza
\end{itemize}
\chapter{Conclusioni}

\section{Riepilogo del Progetto}
Il progetto Site War rappresenta uno strumento innovativo nel campo del web testing, capace di analizzare e confrontare due siti web in modo approfondito, presentando il risultato come una "guerra" con la proclamazione di un vincitore. Questa documentazione ha fornito una visione completa e dettagliata dell'architettura, dei componenti e delle funzionalità del sistema, seguendo un approccio top-down che ha permesso di esplorare tutti gli aspetti del progetto, dalle considerazioni ad alto livello fino ai dettagli implementativi.

L'utilizzo di un approccio creativo per rendere il web testing più accessibile e coinvolgente distingue Site War da altri strumenti analitici tradizionali. La combinazione di analisi tecniche approfondite con una presentazione visivamente coinvolgente permette di attrarre un pubblico più ampio, mantenendo al contempo il rigore tecnico necessario per fornire informazioni accurate e utili.

\section{Punti di Forza del Sistema}

\subsection{Architettura Distribuita}
L'architettura client-server con elaborazione distribuita rappresenta uno dei principali punti di forza del sistema. Sfruttando le capacità di elaborazione del browser client per le analisi che possono essere eseguite lato client e delegando al server le analisi più complesse o quelle che richiedono l'integrazione con API esterne, il sistema ottimizza l'uso delle risorse e minimizza i tempi di attesa, rispettando il vincolo di completare le analisi entro 25 secondi.

\subsection{Modularità e Estensibilità}
Il sistema è progettato con una struttura altamente modulare che favorisce l'estensibilità. L'utilizzo di pattern di design come Factory Method, Strategy e Observer permette di aggiungere facilmente nuove funzionalità e adattare il sistema a nuove esigenze senza modificare la struttura esistente. Questo approccio modulare facilita anche la manutenzione e l'evoluzione del sistema nel tempo.

\subsection{Esperienza Utente Coinvolgente}
L'interfaccia utente di Site War, con le sue animazioni e visualizzazioni interattive, trasforma un processo tecnico in un'esperienza coinvolgente. Le animazioni che rappresentano la "guerra" tra i siti durante l'analisi mantengono l'attenzione dell'utente durante il processo di elaborazione, mentre i risultati finali sono presentati in modo chiaro e intuitivo, con diversi livelli di dettaglio per soddisfare utenti con diverse esigenze.

\subsection{Approccio alla Sicurezza}
La sicurezza è una priorità nel design del sistema, con un approccio "Security by Design" che integra considerazioni sulla sicurezza fin dalle prime fasi di progettazione. Le misure di sicurezza implementate a diversi livelli (client-side, API layer, server-side) garantiscono la protezione dei dati e delle risorse del sistema, mentre l'analisi delle minacce e il piano di risposta agli incidenti assicurano una gestione efficace dei rischi.

\section{Sfide e Soluzioni}

\subsection{Performance e Tempo di Risposta}
Una delle principali sfide del progetto è stata quella di completare le analisi entro il limite di 25 secondi. Questa sfida è stata affrontata attraverso diverse strategie:

\begin{itemize}
    \item Parallelizzazione delle analisi
    \item Distribuzione del carico tra client e server
    \item Implementazione di un sistema di cache
    \item Ottimizzazione delle chiamate API
    \item Prioritizzazione delle analisi più rapide
\end{itemize}

\subsection{Integrazione con API Esterne}
L'integrazione con diverse API esterne ha rappresentato un'altra sfida significativa, considerando la variabilità dei tempi di risposta e la possibilità di indisponibilità. Questa sfida è stata affrontata attraverso:

\begin{itemize}
    \item Implementazione di un servizio proxy
    \item Strategie di fallback in caso di indisponibilità
    \item Sistema di cache per ridurre le chiamate ripetute
    \item Gestione delle rate limits
\end{itemize}

\subsection{Visualizzazione dei Dati Tecnici}
La presentazione dei dati tecnici in modo comprensibile per utenti con diversi livelli di competenza ha richiesto un'attenta progettazione dell'interfaccia utente. Le soluzioni adottate includono:

\begin{itemize}
    \item Utilizzo di grafici e visualizzazioni intuitive
    \item Presentazione a livelli, con dettagli tecnici accessibili su richiesta
    \item Uso di colori e icone per indicare lo stato e la qualità
    \item Spiegazioni contestuali per le metriche tecniche
\end{itemize}

\subsection{Compatibilità Cross-Browser}
Garantire un'esperienza coerente su diversi browser e dispositivi ha richiesto un'attenzione particolare alla compatibilità. Questo è stato ottenuto attraverso:

\begin{itemize}
    \item Test approfonditi su diversi browser e dispositivi
    \item Implementazione di fallback per funzionalità non supportate
    \item Utilizzo di librerie cross-browser
    \item Design responsive per adattarsi a diverse dimensioni di schermo
\end{itemize}

\section{Opportunità Future}

\subsection{Espansione delle Analisi}
Il sistema potrebbe essere esteso per includere nuovi tipi di analisi, come:

\begin{itemize}
    \item Analisi dell'accessibilità
    \item Analisi delle best practices per il mobile
    \item Analisi della compatibilità con le normative (es. GDPR, CCPA)
    \item Analisi dell'impatto ambientale (es. carbon footprint)
\end{itemize}

\subsection{Miglioramento dell'AI}
L'integrazione dell'AI potrebbe essere migliorata per fornire:

\begin{itemize}
    \item Analisi semantica più approfondita del contenuto
    \item Suggerimenti personalizzati per il miglioramento
    \item Previsione dei trend e delle performance future
    \item Identificazione automatica di pattern e anomalie
\end{itemize}

\subsection{Arricchimento della Visualizzazione}
Le visualizzazioni e le animazioni potrebbero essere ulteriormente arricchite con:

\begin{itemize}
    \item Implementazione di realtà aumentata/virtuale
    \item Visualizzazioni 3D interattive
    \item Storytelling basato sui dati
    \item Personalizzazione dell'esperienza visiva
\end{itemize}

\subsection{Funzionalità Social}
Potrebbero essere introdotte funzionalità social per arricchire l'esperienza utente:

\begin{itemize}
    \item Condivisione dei risultati sui social media
    \item Classifica dei siti più performanti
    \item Commenti e feedback sulla "battaglia"
    \item Sfide tra comunità di sviluppatori
\end{itemize}

\section{Considerazioni Finali}
Il progetto Site War rappresenta un esempio di come la tecnologia possa essere utilizzata non solo per fornire informazioni tecniche, ma anche per creare esperienze coinvolgenti e accessibili. La combinazione di rigore tecnico e creatività nella presentazione ha il potenziale per attrarre un pubblico più ampio agli aspetti tecnici del web development, contribuendo alla diffusione di conoscenze e best practices nel settore.

Il design modulare e l'architettura scalabile del sistema forniscono una solida base per l'evoluzione futura, permettendo l'adattamento a nuove tecnologie e esigenze. L'attenzione alla sicurezza, alla performance e all'esperienza utente garantisce un prodotto robusto e di qualità.

In definitiva, Site War non è solo uno strumento di analisi tecnica, ma un'innovativa piattaforma che trasforma il web testing in un'esperienza coinvolgente e formativa, rendendo la tecnologia più accessibile e divertente per un pubblico più ampio.

\end{document}
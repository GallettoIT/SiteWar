\chapter{Sviluppo e Deployment}

\section{Ambiente di Sviluppo}
L'ambiente di sviluppo di Site War è configurato per garantire efficienza, coerenza e qualità durante tutto il processo di sviluppo. Questo capitolo descrive in dettaglio gli strumenti, le configurazioni e le pratiche da seguire durante lo sviluppo del progetto.

\subsection{Requisiti Software}
Per lo sviluppo di Site War sono necessari i seguenti strumenti:

\begin{itemize}
    \item Editor di testo o IDE (es. Visual Studio Code, Sublime Text)
    \item Browser moderni (Chrome, Firefox, Safari, Edge)
    \item PHP 7.4 o superiore
    \item Server web (es. Apache, Nginx)
    \item MySQL o altro database compatibile
    \item Node.js e npm per la gestione delle dipendenze frontend
    \item Strumenti di versionamento (es. Git)
\end{itemize}

\subsection{Configurazione dell'Ambiente}
Di seguito sono riportate le istruzioni per configurare l'ambiente di sviluppo:

\begin{enumerate}
    \item \textbf{Installazione del server locale}:
    \begin{itemize}
        \item Installare XAMPP/WAMP/MAMP con PHP 7.4 o superiore
        \item Configurare un virtual host per il progetto
        \item Impostare i permessi corretti sulle directory
    \end{itemize}
    
    \item \textbf{Configurazione del repository Git}:
    \begin{itemize}
        \item Clonare il repository da GitHub
        \item Configurare i git hooks per lint pre-commit
        \item Impostare .gitignore per escludere file di configurazione locali e cache
    \end{itemize}
    
    \item \textbf{Installazione delle dipendenze}:
    \begin{itemize}
        \item Eseguire \texttt{npm install} per le dipendenze frontend
        \item Eseguire \texttt{composer install} per eventuali dipendenze PHP
    \end{itemize}
    
    \item \textbf{Configurazione delle API Key}:
    \begin{itemize}
        \item Copiare il file \texttt{config/api\_keys.sample.php} in \texttt{config/api\_keys.php}
        \item Ottenere le chiavi API per i servizi richiesti e inserirle nel file
    \end{itemize}
\end{enumerate}

%\subsection{Struttura di Directory}
%Il progetto segue la seguente struttura di directory:
%
%\begin{verbatim}
%site-war/
%│
%├── assets/                   # Risorse statiche
%│   ├── css/                  # Fogli di stile
%│   │   ├── main.css          # Stile principale
%│   │   ├── animations.css    # Animazioni della "guerra"
%│   │   ├── components/       # Stili per componenti specifici
%│   │   └── vendors/          # CSS di terze parti (Bootstrap)
%│   │
%│   ├── js/                   # JavaScript
%│   │   ├── main.js           # Entry point
%│   │   ├── modules/          # Moduli funzionali
%│   │   │   ├── analyzers/    # Moduli di analisi
%│   │   │   ├── ui/           # Componenti UI
%│   │   │   ├── core/         # Funzionalità core
%│   │   │   └── comparison/   # Logica di confronto
%│   │   │
%│   │   └── vendors/          # Librerie di terze parti
%│   │
%│   └── images/               # Immagini e icone
%│
%├── server/                   # Backend PHP
%│   ├── api/                  # Endpoint API
%│   ├── config/               # Configurazioni
%│   ├── core/                 # Funzionalità core
%│   ├── services/             # Servizi di business logic
%│   └── utils/                # Utility functions
%│
%├── templates/                # Template HTML modulari
%│   ├── components/           # Componenti riutilizzabili
%│   └── pages/                # Template pagine
%│
%├── tests/                    # Test unitari e funzionali
%│
%├── index.php                 # Entry point applicazione
%├── .htaccess                 # Configurazione Apache
%└── README.md                 # Documentazione
%\end{verbatim}

\section{Linee Guida per lo Sviluppo}

\subsection{Standard di Codifica}
Per garantire la manutenibilità e la leggibilità del codice, tutti gli sviluppatori devono seguire questi standard:

\subsubsection{PHP}
\begin{itemize}
    \item Seguire le PSR-1 e PSR-12 per lo stile del codice
    \item Utilizzare la documentazione PHPDoc per tutte le classi e i metodi
    \item Evitare l'uso di variabili globali
    \item Limitare l'uso di costrutti basati su stringhe come \texttt{eval()}
    \item Usare costanti per valori fissi
\end{itemize}

\subsubsection{JavaScript}
\begin{itemize}
    \item Seguire lo standard Airbnb JavaScript Style Guide
    \item Usare il Module Pattern per organizzare il codice
    \item Documentare le funzioni con JSDoc
    \item Preferire l'uso di funzioni anonime con arrow functions
    \item Usare sempre \texttt{'use strict'}
\end{itemize}

\subsubsection{HTML/CSS}
\begin{itemize}
    \item Seguire le linee guida HTML5
    \item Usare la convenzione BEM per i nomi delle classi CSS
    \item Strutturare i fogli di stile in modo modulare
    \item Garantire la validazione W3C
    \item Mantenere la separazione tra struttura (HTML), presentazione (CSS) e comportamento (JS)
\end{itemize}

\subsection{Best Practices}
\begin{itemize}
    \item \textbf{Security by Design}: Implementare misure di sicurezza fin dall'inizio
    \item \textbf{Defensive Programming}: Validare tutti gli input e gestire tutti i possibili errori
    \item \textbf{DRY (Don't Repeat Yourself)}: Evitare la duplicazione del codice
    \item \textbf{KISS (Keep It Simple, Stupid)}: Preferire soluzioni semplici e chiare
    \item \textbf{Progressive Enhancement}: Garantire funzionalità di base per tutti gli utenti
\end{itemize}

\subsection{Processo di Revisione del Codice}
Tutto il codice deve passare attraverso un processo di revisione prima di essere integrato nel branch principale:

\begin{enumerate}
    \item Lo sviluppatore crea un branch feature/fix
    \item Lo sviluppatore implementa le modifiche e esegue i test locali
    \item Viene creata una Pull Request
    \item Almeno un altro sviluppatore revisiona il codice
    \item I test automatici vengono eseguiti sulla Pull Request
    \item Dopo l'approvazione e il passaggio dei test, il codice viene integrato
\end{enumerate}

\section{Testing}

\subsection{Tipi di Test}
Il progetto prevede diversi livelli di testing per garantire la qualità del software:

\begin{itemize}
    \item \textbf{Test Unitari}: Verifica delle singole unità di codice
    \item \textbf{Test di Integrazione}: Verifica dell'interazione tra diversi moduli
    \item \textbf{Test Funzionali}: Verifica del comportamento dell'applicazione dal punto di vista dell'utente
    \item \textbf{Test di Performance}: Verifica delle prestazioni del sistema
    \item \textbf{Test di Compatibilità}: Verifica del funzionamento sui diversi browser e dispositivi
    \item \textbf{Test di Accessibilità}: Verifica della conformità agli standard WCAG 2.1 AA
    \item \textbf{Test di Sicurezza}: Verifica della resistenza a vulnerabilità comuni
\end{itemize}

\subsection{Strumenti di Testing}
\begin{itemize}
    \item \textbf{PHPUnit}: Per i test unitari PHP
    \item \textbf{Jest}: Per i test unitari JavaScript
    \item \textbf{Cypress}: Per i test funzionali end-to-end
    \item \textbf{Lighthouse}: Per i test di performance
    \item \textbf{BrowserStack}: Per i test di compatibilità cross-browser
    \item \textbf{WAVE}: Per i test di accessibilità
    \item \textbf{OWASP ZAP}: Per i test di sicurezza automatizzati
\end{itemize}

\subsection{Strategia di Testing}
\begin{itemize}
    \item \textbf{Test-Driven Development (TDD)}: Per componenti critici
    \item \textbf{Continuous Testing}: Test automatici eseguiti ad ogni commit
    \item \textbf{Test Coverage}: Mirare a una copertura di test del 80\% per il codice core
    \item \textbf{Smoke Testing}: Verifiche rapide sui principali flussi utente
    \item \textbf{Regression Testing}: Verifica che le nuove modifiche non rompano funzionalità esistenti
\end{itemize}

\section{Continuous Integration e Continuous Deployment}

\subsection{Workflow CI/CD}
Il progetto utilizza un workflow CI/CD completo per automatizzare il processo di integrazione e deployment:

```mermaid
flowchart LR
    DeveloperChange["Developer Change"] --> GitCommit["Git Commit"]
    GitCommit --> AutomatedTests["Automated Tests"]
    AutomatedTests --> CodeReview["Code Review"]
    CodeReview --> MergeToMainBranch["Merge to Main Branch"]
    MergeToMainBranch --> StagingDeployment["Staging Deployment"]
    StagingDeployment --> LiveTesting["Live Testing"]
    LiveTesting --> ProductionDeployment["Production Deployment"]
```

\subsection{Configurazione CI/CD}
\begin{itemize}
    \item \textbf{Jenkins}: Per l'automazione dei processi CI/CD
    \item \textbf{GitHub Actions}: Per l'integrazione con il repository
    \item \textbf{Docker}: Per la creazione di ambienti di test e deployment consistenti
    \item \textbf{Automated Testing}: Esecuzione automatica di tutti i test ad ogni commit
    \item \textbf{Deployment Automatico}: Deployment automatico agli ambienti di staging e produzione dopo il passaggio dei test
\end{itemize}

\section{Gestione delle Release}

\subsection{Versionamento}
Il progetto segue il versionamento semantico (SemVer):

\begin{itemize}
    \item \textbf{Major}: Cambiamenti incompatibili con le versioni precedenti
    \item \textbf{Minor}: Nuove funzionalità compatibili con le versioni precedenti
    \item \textbf{Patch}: Correzioni di bug compatibili con le versioni precedenti
\end{itemize}

\subsection{Processo di Release}
\begin{enumerate}
    \item Creazione di un branch \texttt{release/X.Y.Z}
    \item Esecuzione di test completi sul branch di release
    \item Aggiornamento di changelog e documentazione
    \item Merge nel branch \texttt{main} e tagging con la versione
    \item Deployment in produzione
    \item Comunicazione della release agli stakeholder
\end{enumerate}

\subsection{Hotfix}
Per correzioni urgenti in produzione:
\begin{enumerate}
    \item Creazione di un branch \texttt{hotfix/X.Y.Z+1}
    \item Implementazione e test della correzione
    \item Merge nei branch \texttt{main} e \texttt{develop}
    \item Deployment in produzione
\end{enumerate}

\section{Ambienti di Deployment}

\subsection{Ambienti Disponibili}
\begin{itemize}
    \item \textbf{Development}: Ambiente di sviluppo locale per ogni sviluppatore
    \item \textbf{Integration}: Ambiente condiviso per l'integrazione delle modifiche
    \item \textbf{Staging}: Ambiente di pre-produzione per test finali
    \item \textbf{Production}: Ambiente di produzione visibile agli utenti finali
\end{itemize}

\subsection{Requisiti Infrastrutturali}
\begin{table}[H]
\centering
\begin{tabular}{|l|l|l|l|}
\hline
\textbf{Risorsa} & \textbf{Sviluppo} & \textbf{Staging} & \textbf{Produzione} \\
\hline
Server Web & Apache/Nginx & Nginx & Nginx \\
\hline
PHP & 7.4+ & 7.4+ & 7.4+ \\
\hline
CPU & 2 core & 4 core & 8+ core \\
\hline
RAM & 4 GB & 8 GB & 16+ GB \\
\hline
Storage & 20 GB SSD & 50 GB SSD & 100+ GB SSD \\
\hline
Caching & Opzionale & Redis & Redis \\
\hline
CDN & No & Opzionale & Sì \\
\hline
SSL & Opzionale & Sì & Sì \\
\hline
\end{tabular}
\caption{Requisiti infrastrutturali per ambiente}
\label{table:infrastructure-requirements}
\end{table}

\subsection{Configurazione di Deployment}

```mermaid
flowchart TD
    subgraph ProductionDeployment["Production Deployment"]
        Frontend["Frontend (Nginx)"] <--> Backend["Backend (PHP-FPM)"]
        Backend <--> RedisCache["Redis Cache"]
        
        Frontend --> CDNAssets["CDN (Assets)"]
        Backend --> ExternalAPIs["External APIs"]
    end
```

\subsection{Gestione delle Configurazioni}
\begin{itemize}
    \item \textbf{Environment Variables}: Configurazioni specifiche per ambiente
    \item \textbf{Configuration Files}: File di configurazione per impostazioni non sensibili
    \item \textbf{Secret Management}: Gestione sicura di chiavi API e credenziali
    \item \textbf{Feature Flags}: Abilita/disabilita funzionalità per ambiente
\end{itemize}

\section{Monitoraggio e Logging}

\subsection{Sistema di Logging}
\begin{itemize}
    \item \textbf{Application Logs}: Log delle attività dell'applicazione
    \item \textbf{Error Logs}: Log degli errori e delle eccezioni
    \item \textbf{Access Logs}: Log degli accessi al sistema
    \item \textbf{Performance Logs}: Log delle metriche di performance
    \item \textbf{API Logs}: Log delle chiamate alle API esterne
\end{itemize}

\subsection{Monitoraggio}
\begin{itemize}
    \item \textbf{Uptime Monitoring}: Controllo della disponibilità del servizio
    \item \textbf{Performance Monitoring}: Monitoraggio delle prestazioni
    \item \textbf{Error Tracking}: Rilevamento e tracciamento degli errori
    \item \textbf{Resource Monitoring}: Monitoraggio dell'utilizzo delle risorse
    \item \textbf{User Analytics}: Analisi del comportamento degli utenti
\end{itemize}

\subsection{Alerting}
\begin{itemize}
    \item \textbf{Threshold Alerts}: Avvisi basati su soglie predefinite
    \item \textbf{Anomaly Detection}: Rilevamento di comportamenti anomali
    \item \textbf{Incident Management}: Processo di gestione degli incidenti
    \item \textbf{On-Call Rotation}: Rotazione del personale di supporto
\end{itemize}

\section{Documentazione}

\subsection{Tipi di Documentazione}
\begin{itemize}
    \item \textbf{Documentazione Tecnica}: Descrizione dettagliata dell'architettura e dei componenti
    \item \textbf{API Documentation}: Documentazione delle API esposte
    \item \textbf{Code Documentation}: Documentazione del codice sorgente
    \item \textbf{User Documentation}: Guide per gli utenti finali
    \item \textbf{Operations Documentation}: Procedure operative e di manutenzione
\end{itemize}

\subsection{Gestione della Documentazione}
\begin{itemize}
    \item \textbf{Documentation as Code}: La documentazione è trattata come codice e versionata
    \item \textbf{Automated Generation}: Generazione automatica della documentazione dall'API e dal codice
    \item \textbf{Review Process}: Processo di revisione per la documentazione
    \item \textbf{Update on Release}: Aggiornamento della documentazione ad ogni release
\end{itemize}

\section{Backup e Disaster Recovery}

\subsection{Strategia di Backup}
\begin{itemize}
    \item \textbf{Code Repository}: Il codice è versionato e disponibile nel repository Git
    \item \textbf{Configuration Backup}: Backup regolari delle configurazioni
    \item \textbf{Data Backup}: Backup dei dati generati dall'applicazione
    \item \textbf{Log Backup}: Archiviazione dei log per analisi future
\end{itemize}

\subsection{Disaster Recovery}
\begin{itemize}
    \item \textbf{Recovery Plan}: Piano dettagliato per il ripristino del servizio
    \item \textbf{Recovery Time Objective (RTO)}: Tempo massimo accettabile per il ripristino
    \item \textbf{Recovery Point Objective (RPO)}: Massima perdita di dati accettabile
    \item \textbf{Failover Mechanism}: Meccanismi di failover per garantire la continuità del servizio
    \item \textbf{Regular Testing}: Test regolari del piano di disaster recovery
\end{itemize}

\section{Manutenzione e Supporto}

\subsection{Manutenzione Regolare}
\begin{itemize}
    \item \textbf{Security Updates}: Aggiornamenti regolari per questioni di sicurezza
    \item \textbf{Dependency Updates}: Aggiornamento delle dipendenze
    \item \textbf{Performance Optimization}: Ottimizzazione continua delle performance
    \item \textbf{Code Refactoring}: Refactoring del codice per migliorarne la qualità
\end{itemize}

\subsection{Supporto}
\begin{itemize}
    \item \textbf{Issue Tracking}: Sistema di tracciamento dei problemi
    \item \textbf{Support Tiers}: Livelli di supporto in base alla criticità
    \item \textbf{Response Time}: Tempi di risposta garantiti per ciascun livello
    \item \textbf{User Feedback}: Raccolta e analisi del feedback degli utenti
\end{itemize}
\chapter{Sicurezza}

\section{Panoramica della Sicurezza}
La sicurezza è un aspetto fondamentale del sistema Site War, implementata a diversi livelli per proteggere sia i dati degli utenti che le risorse del sistema. Questo capitolo descrive in dettaglio le misure di sicurezza adottate, le potenziali vulnerabilità e le strategie di mitigazione implementate.

L'approccio alla sicurezza del sistema segue il principio di "Security by Design", integrando considerazioni sulla sicurezza fin dalle prime fasi di progettazione dell'architettura. Questo approccio proattivo permette di identificare e mitigare potenziali rischi prima che possano essere sfruttati.

\section{Modello di Sicurezza}
Il modello di sicurezza di Site War è strutturato in diversi livelli, ciascuno con specifiche misure di protezione:

```mermaid
flowchart TD
    subgraph SecurityArchitecture["Security Architecture"]
        ClientSideSecurity["Client-side Security"]
        APILayerSecurity["API Layer Security"]
        ServerSideSecurity["Server-side Security"]
    end
    
    ClientSideSecurity --> InputValidation["Input Valid.\nXSS Prevention\nCSP Implement.\nObfuscation"]
    APILayerSecurity --> RequestValidation["Request Valid.\nCSRF Protection\nRate Limiting\nAuthentication"]
    ServerSideSecurity --> APIKeyProtection["API Key Protect\nInput Sanitiz.\nError Handling\nHTTPS Enforce."]
```

\subsection{Principi di Sicurezza}
Il sistema Site War si basa sui seguenti principi di sicurezza:

\begin{itemize}
    \item \textbf{Defense in Depth}: Implementazione di multiple barriere di protezione
    \item \textbf{Principle of Least Privilege}: Concessione dei minimi privilegi necessari
    \item \textbf{Secure by Default}: Configurazioni sicure predefinite
    \item \textbf{Fail Securely}: In caso di errore, il sistema deve fallire in modo sicuro
    \item \textbf{Security Through Transparency}: Trasparenza nelle pratiche di sicurezza
\end{itemize}

\section{Sicurezza Client-Side}

\subsection{Validazione degli Input}
Tutti gli input dell'utente sono validati sia lato client che lato server:

\begin{itemize}
    \item \textbf{Validazione Sintattica}: Verifica del formato e della lunghezza
    \item \textbf{Validazione Semantica}: Verifica della coerenza e della logica
    \item \textbf{Validazione Contestuale}: Verifica basata sul contesto di utilizzo
    \item \textbf{Sanitizzazione}: Rimozione o escape di caratteri pericolosi
\end{itemize}

\subsection{Protezione da Cross-Site Scripting (XSS)}
\begin{itemize}
    \item \textbf{Output Encoding}: Codifica dell'output per prevenire l'esecuzione di script
    \item \textbf{Content Security Policy (CSP)}: Restrizione delle fonti di contenuto eseguibile
    \item \textbf{HTTP-only Cookies}: Prevenzione dell'accesso ai cookie tramite JavaScript
    \item \textbf{X-XSS-Protection}: Header HTTP per abilitare la protezione XSS dei browser
\end{itemize}

\subsection{Obfuscation del Codice JavaScript}
\begin{itemize}
    \item \textbf{Minificazione}: Rimozione di spazi, commenti e altri elementi non necessari
    \item \textbf{Compressione}: Riduzione delle dimensioni del codice
    \item \textbf{Offuscamento}: Trasformazione del codice per renderlo meno leggibile
    \item \textbf{Protezione Algoritmi}: Nascondere la logica di business critica
\end{itemize}

\section{Sicurezza API Layer}

\subsection{Autenticazione e Autorizzazione}
\begin{itemize}
    \item \textbf{Validazione della Richiesta}: Verifica dell'integrità e della validità della richiesta
    \item \textbf{Origin Validation}: Verifica dell'origine della richiesta
    \item \textbf{Rate Limiting}: Limitazione del numero di richieste per prevenire abusi
    \item \textbf{API Key Validation}: Verifica delle chiavi API per le richieste autenticate
\end{itemize}

\subsection{Protezione da Cross-Site Request Forgery (CSRF)}
\begin{itemize}
    \item \textbf{CSRF Tokens}: Utilizzo di token anti-CSRF per le richieste sensibili
    \item \textbf{Same-Site Cookies}: Utilizzo di cookie same-site per limitare le richieste cross-site
    \item \textbf{Referrer Policy}: Controllo delle informazioni di referrer inviate nelle richieste
\end{itemize}

\subsection{Sicurezza delle Comunicazioni}
\begin{itemize}
    \item \textbf{HTTPS}: Utilizzo di HTTPS per tutte le comunicazioni
    \item \textbf{HTTP Strict Transport Security (HSTS)}: Forzatura dell'uso di HTTPS
    \item \textbf{Certificate Pinning}: Verifica del certificato SSL/TLS
    \item \textbf{TLS 1.3}: Utilizzo della versione più recente e sicura di TLS
\end{itemize}

\section{Sicurezza Server-Side}

\subsection{Protezione delle Chiavi API}
Le chiavi API utilizzate per accedere ai servizi esterni sono protette con le seguenti misure:

\begin{itemize}
    \item \textbf{Storage Sicuro}: Le chiavi sono memorizzate in file di configurazione protetti
    \item \textbf{Path Traversal Protection}: Prevenzione dell'accesso a file di configurazione
    \item \textbf{Proxy Service}: Le richieste API vengono inoltrate tramite un servizio proxy
    \item \textbf{Key Rotation}: Rotazione periodica delle chiavi API
\end{itemize}

\subsection{Protezione da Injection}
\begin{itemize}
    \item \textbf{Prepared Statements}: Utilizzo di statement preparati per le query al database
    \item \textbf{Input Sanitization}: Sanitizzazione degli input per prevenire injection
    \item \textbf{Parameter Validation}: Validazione dei parametri delle richieste
    \item \textbf{Safe API}: Utilizzo di API sicure che prevengono l'injection
\end{itemize}

\subsection{Sicurezza dei File}
\begin{itemize}
    \item \textbf{File Type Validation}: Validazione del tipo di file
    \item \textbf{File Size Limits}: Limitazione della dimensione dei file
    \item \textbf{File Storage Isolation}: Isolamento dello storage dei file
    \item \textbf{File Access Control}: Controllo dell'accesso ai file
\end{itemize}

\section{Protezione della Privacy dei Dati}

\subsection{Minimizzazione dei Dati}
\begin{itemize}
    \item \textbf{Raccolta Minimale}: Raccolta solo dei dati necessari
    \item \textbf{Eliminazione Tempestiva}: Eliminazione dei dati non più necessari
    \item \textbf{Storage Temporaneo}: Utilizzo di storage temporaneo per dati transitori
\end{itemize}

\subsection{Conformità GDPR}
\begin{itemize}
    \item \textbf{Consenso Esplicito}: Ottenere il consenso esplicito dell'utente
    \item \textbf{Diritto all'Oblio}: Possibilità di cancellare i propri dati
    \item \textbf{Portabilità dei Dati}: Possibilità di esportare i propri dati
    \item \textbf{Privacy by Design}: Privacy integrata nella progettazione
\end{itemize}

\section{Analisi delle Minacce}

\subsection{Modello STRIDE}
L'analisi delle minacce è stata condotta utilizzando il modello STRIDE:

\begin{table}[H]
\centering
\begin{tabular}{|l|l|l|}
\hline
\textbf{Tipo di Minaccia} & \textbf{Descrizione} & \textbf{Contromisure} \\
\hline
Spoofing & Impersonare un'altra entità & Validazione delle richieste \\
\hline
Tampering & Modificare dati & Controlli di integrità, HTTPS \\
\hline
Repudiation & Negare un'azione & Logging, audit trail \\
\hline
Information Disclosure & Accesso non autorizzato a dati & Encryption, controllo accessi \\
\hline
Denial of Service & Interruzione del servizio & Rate limiting, failover \\
\hline
Elevation of Privilege & Ottenere privilegi non autorizzati & Principle of least privilege \\
\hline
\end{tabular}
\caption{Analisi delle minacce con modello STRIDE}
\label{table:stride-threats}
\end{table}

\subsection{Attack Vectors}
\begin{itemize}
    \item \textbf{Client-Side Attacks}: XSS, CSRF, Clickjacking
    \item \textbf{Network Attacks}: Man-in-the-Middle, DNS Spoofing
    \item \textbf{Server-Side Attacks}: Injection, Path Traversal
    \item \textbf{API Abuse}: Rate Limiting Bypass, API Key Theft
\end{itemize}

\section{Security Headers}
La seguente configurazione di security headers è implementata per migliorare la sicurezza del browser:

\begin{table}[H]
\centering
\begin{tabular}{|l|l|}
\hline
\textbf{Header} & \textbf{Valore/Scopo} \\
\hline
Content-Security-Policy & Definisce le fonti affidabili per l'esecuzione di script, stili, ecc. \\
\hline
X-Content-Type-Options & nosniff; Previene il MIME type sniffing \\
\hline
X-Frame-Options & DENY; Previene il clickjacking \\
\hline
X-XSS-Protection & 1; mode=block; Abilita la protezione XSS del browser \\
\hline
Strict-Transport-Security & max-age=31536000; Forza l'uso di HTTPS \\
\hline
Referrer-Policy & strict-origin-when-cross-origin; Controlla le informazioni di referrer \\
\hline
Feature-Policy & Limita l'accesso a funzionalità del browser \\
\hline
\end{tabular}
\caption{Security Headers}
\label{table:security-headers}
\end{table}

\section{Rate Limiting e Protezione da DoS}

\subsection{Strategia di Rate Limiting}
Per prevenire abusi e attacchi di tipo Denial of Service, il sistema implementa una strategia di rate limiting:

\begin{itemize}
    \item \textbf{IP-based Rate Limiting}: Limitazione del numero di richieste per IP
    \item \textbf{API-based Rate Limiting}: Limitazione del numero di richieste per API
    \item \textbf{Graduated Rate Limiting}: Limitazione crescente in base all'utilizzo
    \item \textbf{Token Bucket Algorithm}: Algoritmo flessibile per il rate limiting
\end{itemize}

\subsection{Protezione da DoS/DDoS}
\begin{itemize}
    \item \textbf{Traffic Filtering}: Filtraggio del traffico sospetto
    \item \textbf{Resource Allocation}: Allocazione delle risorse in base alla priorità
    \item \textbf{Connection Throttling}: Limitazione delle connessioni simultanee
    \item \textbf{CDN}: Utilizzo di CDN per assorbire il traffico
\end{itemize}

\section{Secure Coding Practices}

\subsection{Linee Guida per la Sicurezza del Codice}
\begin{itemize}
    \item \textbf{Input Validation}: Validare tutti gli input dell'utente
    \item \textbf{Output Encoding}: Codificare tutto l'output in base al contesto
    \item \textbf{Error Handling}: Gestire gli errori in modo sicuro
    \item \textbf{Authentication}: Implementare meccanismi di autenticazione robusti
    \item \textbf{Session Management}: Gestire le sessioni in modo sicuro
    \item \textbf{Access Control}: Implementare controlli di accesso adeguati
\end{itemize}

\subsection{Code Review di Sicurezza}
\begin{itemize}
    \item \textbf{Security Code Review}: Revisione del codice specifica per la sicurezza
    \item \textbf{Static Analysis}: Analisi statica del codice
    \item \textbf{Dynamic Analysis}: Analisi dinamica dell'applicazione
    \item \textbf{Peer Review}: Revisione del codice da parte di altri sviluppatori
\end{itemize}

\section{Incident Response}

\subsection{Piano di Risposta agli Incidenti}
In caso di incidente di sicurezza, il sistema prevede un piano di risposta strutturato:

\begin{enumerate}
    \item \textbf{Rilevamento}: Identificazione dell'incidente
    \item \textbf{Contenimento}: Limitazione dell'impatto dell'incidente
    \item \textbf{Eradicazione}: Rimozione della causa dell'incidente
    \item \textbf{Ripristino}: Ripristino del normale funzionamento
    \item \textbf{Analisi Post-Incidente}: Analisi delle cause e delle lezioni apprese
    \item \textbf{Miglioramento}: Implementazione di miglioramenti per prevenire incidenti simili
\end{enumerate}

\subsection{Reporting e Comunicazione}
\begin{itemize}
    \item \textbf{Internal Reporting}: Reportistica interna sugli incidenti
    \item \textbf{External Reporting}: Comunicazione con gli utenti e le autorità
    \item \textbf{Disclosure Policy}: Politica di divulgazione degli incidenti
    \item \textbf{Communication Templates}: Template per la comunicazione in caso di incidente
\end{itemize}

\section{Testing di Sicurezza}

\subsection{Metodologie di Test}
\begin{itemize}
    \item \textbf{SAST (Static Application Security Testing)}: Analisi statica del codice
    \item \textbf{DAST (Dynamic Application Security Testing)}: Analisi dinamica dell'applicazione
    \item \textbf{Penetration Testing}: Test di penetrazione manuali
    \item \textbf{Vulnerability Scanning}: Scansione automatica delle vulnerabilità
    \item \textbf{Security Code Review}: Revisione manuale del codice per problemi di sicurezza
\end{itemize}

\subsection{Strumenti di Test}
\begin{itemize}
    \item \textbf{OWASP ZAP}: Per il test automatico delle vulnerabilità web
    \item \textbf{Burp Suite}: Per il test manuale delle vulnerabilità web
    \item \textbf{SonarQube}: Per l'analisi statica del codice
    \item \textbf{Nmap}: Per la scansione delle porte e dei servizi
    \item \textbf{Metasploit}: Per il test di penetrazione avanzato
\end{itemize}

\section{Gestione delle Dipendenze}

\subsection{Sicurezza delle Dipendenze}
\begin{itemize}
    \item \textbf{Dependency Scanning}: Scansione delle dipendenze per vulnerabilità note
    \item \textbf{Version Pinning}: Fissare le versioni delle dipendenze
    \item \textbf{Regular Updates}: Aggiornamenti regolari delle dipendenze
    \item \textbf{Minimal Dependencies}: Utilizzo del minimo numero di dipendenze necessarie
\end{itemize}

\subsection{Supply Chain Security}
\begin{itemize}
    \item \textbf{Vendor Verification}: Verifica dei fornitori di software
    \item \textbf{Integrity Verification}: Verifica dell'integrità dei pacchetti
    \item \textbf{Provenance Tracking}: Tracciamento della provenienza del software
    \item \textbf{Build Reproducibility}: Riproducibilità del processo di build
\end{itemize}

\section{Monitoraggio e Logging di Sicurezza}

\subsection{Event Logging}
\begin{itemize}
    \item \textbf{Security Events}: Registrazione degli eventi di sicurezza
    \item \textbf{Access Logging}: Registrazione degli accessi
    \item \textbf{Change Logging}: Registrazione delle modifiche
    \item \textbf{Error Logging}: Registrazione degli errori
\end{itemize}

\subsection{Log Management}
\begin{itemize}
    \item \textbf{Log Rotation}: Rotazione dei log per gestire lo spazio
    \item \textbf{Log Retention}: Conservazione dei log per il periodo necessario
    \item \textbf{Log Protection}: Protezione dei log da manipolazioni
    \item \textbf{Log Analysis}: Analisi dei log per identificare pattern sospetti
\end{itemize}

\section{Compliance e Standard di Sicurezza}

\subsection{Compliance Framework}
\begin{itemize}
    \item \textbf{GDPR}: Regolamento generale sulla protezione dei dati
    \item \textbf{PCI DSS}: Standard di sicurezza per i dati delle carte di pagamento
    \item \textbf{ISO 27001}: Standard per la gestione della sicurezza delle informazioni
    \item \textbf{OWASP Top 10}: Le 10 vulnerabilità web più critiche
\end{itemize}

\subsection{Security Assessment}
\begin{itemize}
    \item \textbf{Regular Audits}: Audit regolari della sicurezza
    \item \textbf{Compliance Checks}: Verifiche di conformità
    \item \textbf{Risk Assessment}: Valutazione dei rischi
    \item \textbf{Gap Analysis}: Analisi delle lacune di sicurezza
\end{itemize}
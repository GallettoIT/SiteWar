\chapter{Casi d'Uso}

\section{Panoramica dei Casi d'Uso}
In questo capitolo vengono descritti in dettaglio i casi d'uso principali del sistema Site War, illustrando le interazioni tra gli utenti e il sistema per le varie funzionalità offerte dall'applicazione. Attraverso questi casi d'uso, si ottiene una visione completa e dettagliata del funzionamento del sistema dal punto di vista dell'utente.

\section{Attori del Sistema}
Il sistema Site War interagisce con diversi tipi di utenti, ciascuno con esigenze e obiettivi specifici:

\begin{itemize}
    \item \textbf{Utente Generico}: Qualsiasi visitatore del sito Web War che desidera confrontare due siti web per curiosità o per scopi informativi.
    
    \item \textbf{Sviluppatore Web}: Utente che utilizza il sistema per confrontare il proprio sito con siti concorrenti, con l'obiettivo di identificare punti di forza e debolezza e migliorare il proprio lavoro.
    
    \item \textbf{Analista SEO/Performance}: Utente specializzato che utilizza il sistema per ottenere dati tecnici comparativi dettagliati su aspetti specifici dei siti web.
    
    \item \textbf{Penetration Tester}: Utente interessato agli aspetti di sicurezza dei siti web, che utilizza il sistema per identificare vulnerabilità e problemi di sicurezza.
\end{itemize}

\section{Diagramma dei Casi d'Uso}
Il diagramma seguente illustra i principali casi d'uso del sistema Site War e le loro relazioni:

```mermaid
flowchart TD
    UC1["UC1: Inserire URL"] --> UC2["UC2: Validare URL"]
    UC2 --> UC3["UC3: Avviare Analisi"]
    UC3 --> UC4["UC4: Visualizzare Animazioni"]
    UC4 --> UC5["UC5: Visualizzare Risultati"]
    UC5 --> UC6["UC6: Esplorare Dettagli"]
    UC6 --> UC7["UC7: Esportare Risultati"]
    UC5 --> UC8["UC8: Analizzare Vincitore"]
```

\section{Descrizioni Dettagliate dei Casi d'Uso}

\subsection{UC1: Inserire URL}

\subsubsection{Attori Principali}
Tutti gli utenti

\subsubsection{Pre-condizioni}
L'utente ha aperto il sito Site War

\subsubsection{Trigger}
L'utente desidera confrontare due siti web

\subsubsection{Scenario Principale di Successo}
\begin{enumerate}
    \item L'utente accede alla homepage di Site War
    \item Il sistema presenta un form con due campi per l'inserimento degli URL
    \item L'utente inserisce l'URL del primo sito nel campo ``Sito 1''
    \item L'utente inserisce l'URL del secondo sito nel campo ``Sito 2''
    \item L'utente fa clic sul pulsante ``Inizia la battaglia!''
    \item Il sistema procede con la validazione degli URL (UC2)
\end{enumerate}

\subsubsection{Estensioni}
\begin{itemize}
    \item 3-4a. L'utente inserisce un URL in formato non valido
    \begin{enumerate}
        \item Il sistema evidenzia il campo con errore
        \item Il sistema mostra un messaggio di errore specifico
        \item L'utente corregge l'input e riprova
    \end{enumerate}
    
    \item 5a. L'utente decide di cambiare uno degli URL inseriti
    \begin{enumerate}
        \item L'utente modifica uno o entrambi i campi URL
        \item L'utente fa clic sul pulsante ``Inizia la battaglia!''
    \end{enumerate}
\end{itemize}

\subsubsection{Requisiti Speciali}
\begin{itemize}
    \item Tempo di risposta: la validazione del formato URL deve essere immediata (client-side)
    \item I campi del form devono essere accessibili da tastiera
    \item Il form deve essere responsive per diversi dispositivi
\end{itemize}

\subsubsection{Frequenza}
Molto frequente (ogni analisi inizia da qui)

\subsection{UC2: Validare URL}

\subsubsection{Attori Principali}
Sistema

\subsubsection{Pre-condizioni}
L'utente ha inserito due URL e inviato il form

\subsubsection{Trigger}
Invio del form di input URL

\subsubsection{Scenario Principale di Successo}
\begin{enumerate}
    \item Il sistema verifica che entrambi gli URL siano in formato valido
    \item Il sistema verifica che entrambi i siti siano accessibili
    \item Il sistema utilizza l'AI per valutare la pertinenza del confronto tra i due siti
    \item L'AI conferma che i siti sono confrontabili
    \item Il sistema procede con l'avvio dell'analisi (UC3)
\end{enumerate}

\subsubsection{Estensioni}
\begin{itemize}
    \item 2a. Uno o entrambi i siti non sono accessibili
    \begin{enumerate}
        \item Il sistema mostra un messaggio di errore specifico
        \item L'utente viene invitato a verificare gli URL e riprovare
    \end{enumerate}
    
    \item 4a. L'AI determina che i siti non sono confrontabili
    \begin{enumerate}
        \item Il sistema mostra un messaggio che spiega perché i siti non sono confrontabili
        \item L'utente può scegliere di procedere comunque con l'analisi o modificare gli URL
    \end{enumerate}
    
    \item 4b. Il servizio AI non è disponibile
    \begin{enumerate}
        \item Il sistema procede con l'analisi senza la validazione di pertinenza
        \item Il sistema mostra un avviso che la validazione di pertinenza non è stata eseguita
    \end{enumerate}
\end{itemize}

\subsubsection{Requisiti Speciali}
\begin{itemize}
    \item Timeout: la validazione completa non deve superare i 5 secondi
    \item In caso di problemi con l'API AI, il sistema deve degradare in modo elegante
\end{itemize}

\subsubsection{Frequenza}
Molto frequente (ogni analisi richiede validazione)

\subsection{UC3: Avviare Analisi}

\subsubsection{Attori Principali}
Sistema

\subsubsection{Pre-condizioni}
Gli URL sono stati validati e sono confrontabili

\subsubsection{Trigger}
Completamento della validazione URL

\subsubsection{Scenario Principale di Successo}
\begin{enumerate}
    \item Il sistema mostra l'interfaccia di ``battaglia'' con i due siti
    \item Il sistema avvia le analisi lato client (DOM, Performance base, SEO base)
    \item Il sistema invia richieste al backend per le analisi più avanzate
    \item Il sistema aggiorna la visualizzazione dell'avanzamento dell'analisi
    \item Il backend elabora le richieste e comunica con API esterne quando necessario
    \item Il sistema riceve progressivamente i risultati e aggiorna l'interfaccia
    \item Al completamento di tutte le analisi, il sistema procede alla visualizzazione dei risultati (UC5)
\end{enumerate}

\subsubsection{Estensioni}
\begin{itemize}
    \item 3a. Errore di comunicazione con il backend
    \begin{enumerate}
        \item Il sistema mostra un messaggio di errore
        \item L'utente può riprovare l'analisi
    \end{enumerate}
    
    \item 5a. Errore di comunicazione con API esterne
    \begin{enumerate}
        \item Il sistema utilizza dati parziali o strategie di fallback
        \item Il sistema continua con analisi alternative disponibili
        \item I risultati saranno marcati come parziali
    \end{enumerate}
\end{itemize}

\subsubsection{Requisiti Speciali}
\begin{itemize}
    \item Performance: l'analisi completa deve essere completata entro 25 secondi
    \item Il sistema deve mostrare un indicatore di avanzamento accurato
    \item Le analisi devono avvenire in parallelo quando possibile
\end{itemize}

\subsubsection{Frequenza}
Molto frequente (ogni analisi passa per questa fase)

\subsection{UC4: Visualizzare Animazioni}

\subsubsection{Attori Principali}
Utente

\subsubsection{Pre-condizioni}
L'analisi è stata avviata

\subsubsection{Trigger}
Avanzamento dell'analisi

\subsubsection{Scenario Principale di Successo}
\begin{enumerate}
    \item Il sistema visualizza un'animazione iniziale che rappresenta i due siti come ``guerrieri''
    \item Man mano che l'analisi procede, il sistema aggiorna l'animazione con effetti visivi
    \item Le prestazioni relative dei siti influenzano l'animazione (es. il sito più veloce sembra ``attaccare'' l'altro)
    \item Al raggiungimento di soglie di progresso (25\%, 50\%, 75\%), l'animazione cambia fase
    \item Quando l'analisi è completa, l'animazione mostra l'effetto finale di ``vittoria''
    \item Il sistema passa alla visualizzazione dei risultati (UC5)
\end{enumerate}

\subsubsection{Estensioni}
\begin{itemize}
    \item 2a. L'utente usa un dispositivo a basse prestazioni
    \begin{enumerate}
        \item Il sistema rileva le capacità del dispositivo
        \item Il sistema mostra animazioni semplificate per garantire performance adeguate
    \end{enumerate}
\end{itemize}

\subsubsection{Requisiti Speciali}
\begin{itemize}
    \item Le animazioni devono essere fluide (60 fps)
    \item Il sistema deve adattare le animazioni in base alle capacità del dispositivo
    \item L'accessibilità deve essere garantita anche con animazioni attive
\end{itemize}

\subsubsection{Frequenza}
Molto frequente (ogni analisi include animazioni)

\subsection{UC5: Visualizzare Risultati}

\subsubsection{Attori Principali}
Utente

\subsubsection{Pre-condizioni}
L'analisi è stata completata

\subsubsection{Trigger}
Completamento dell'analisi

\subsubsection{Scenario Principale di Successo}
\begin{enumerate}
    \item Il sistema mostra una dashboard con la proclamazione del vincitore
    \item Il sistema visualizza i punteggi complessivi per entrambi i siti
    \item Il sistema mostra un confronto visivo delle principali categorie di analisi
    \item L'utente può visualizzare i dettagli di ogni categoria tramite tab
    \item L'utente può decidere di esplorare i dettagli specifici (UC6)
    \item L'utente può esportare i risultati (UC7)
\end{enumerate}

\subsubsection{Estensioni}
\begin{itemize}
    \item 1a. L'analisi ha prodotto risultati parziali
    \begin{enumerate}
        \item Il sistema mostra un avviso che alcuni dati potrebbero essere incompleti
        \item Il sistema indica quali analisi sono state completate con successo
    \end{enumerate}
    
    \item 4a. L'utente desidera confrontare una metrica specifica
    \begin{enumerate}
        \item L'utente seleziona la categoria desiderata
        \item Il sistema mostra una visualizzazione dettagliata per quella categoria
    \end{enumerate}
\end{itemize}

\subsubsection{Requisiti Speciali}
\begin{itemize}
    \item I risultati devono essere visualizzati in modo chiaro e intuitivo
    \item I grafici comparativi devono essere accessibili e comprensibili
    \item La dashboard deve essere responsive per diversi dispositivi
\end{itemize}

\subsubsection{Frequenza}
Molto frequente (ogni analisi completata)

\subsection{UC6: Esplorare Dettagli}

\subsubsection{Attori Principali}
Utente

\subsubsection{Pre-condizioni}
I risultati dell'analisi sono stati visualizzati

\subsubsection{Trigger}
L'utente desidera esplorare dettagli specifici

\subsubsection{Scenario Principale di Successo}
\begin{enumerate}
    \item L'utente fa clic su una categoria specifica (Performance, SEO, Sicurezza, Tecnica)
    \item Il sistema mostra una vista dettagliata con metriche specifiche per quella categoria
    \item Il sistema visualizza grafici comparativi per le metriche della categoria
    \item Il sistema evidenzia i punti di forza e debolezza di ciascun sito
    \item L'utente può navigare tra le diverse categorie utilizzando i tab
    \item L'utente può tornare alla vista generale dei risultati
\end{enumerate}

\subsubsection{Estensioni}
\begin{itemize}
    \item 2a. Dati insufficienti per la categoria selezionata
    \begin{enumerate}
        \item Il sistema mostra un messaggio che indica la mancanza di dati sufficienti
        \item Il sistema offre suggerimenti per analisi alternative
    \end{enumerate}
\end{itemize}

\subsubsection{Requisiti Speciali}
\begin{itemize}
    \item La navigazione tra categorie deve essere intuitiva
    \item I dettagli tecnici devono essere presentati in modo comprensibile
    \item Devono essere forniti suggerimenti per il miglioramento
\end{itemize}

\subsubsection{Frequenza}
Frequente (la maggior parte degli utenti esplora i dettagli)

\subsection{UC7: Esportare Risultati}

\subsubsection{Attori Principali}
Utente

\subsubsection{Pre-condizioni}
I risultati dell'analisi sono stati visualizzati

\subsubsection{Trigger}
L'utente desidera salvare o condividere i risultati

\subsubsection{Scenario Principale di Successo}
\begin{enumerate}
    \item L'utente fa clic sul pulsante ``Esporta risultati''
    \item Il sistema mostra un menu con opzioni di esportazione (CSV, PDF, Stampa)
    \item L'utente seleziona il formato desiderato
    \item Il sistema genera il file nel formato scelto
    \item Il browser avvia il download del file o apre l'anteprima di stampa
\end{enumerate}

\subsubsection{Estensioni}
\begin{itemize}
    \item 3a. L'utente sceglie l'opzione ``Stampa''
    \begin{enumerate}
        \item Il sistema prepara una versione ottimizzata per la stampa
        \item Il browser apre l'anteprima di stampa
    \end{enumerate}
    
    \item 4a. Errore nella generazione del file
    \begin{enumerate}
        \item Il sistema mostra un messaggio di errore
        \item L'utente può riprovare o scegliere un formato alternativo
    \end{enumerate}
\end{itemize}

\subsubsection{Requisiti Speciali}
\begin{itemize}
    \item I file esportati devono includere tutti i dati rilevanti
    \item I formati di esportazione devono essere standard e compatibili
    \item La versione stampabile deve essere ottimizzata per la carta
\end{itemize}

\subsubsection{Frequenza}
Occasionale (alcuni utenti esportano i risultati)

\subsection{UC8: Analizzare Vincitore}

\subsubsection{Attori Principali}
Utente

\subsubsection{Pre-condizioni}
I risultati dell'analisi sono stati visualizzati

\subsubsection{Trigger}
L'utente desidera comprendere i fattori che hanno determinato il vincitore

\subsubsection{Scenario Principale di Successo}
\begin{enumerate}
    \item L'utente fa clic sul badge del vincitore o su un pulsante ``Perché ha vinto?''
    \item Il sistema mostra una spiegazione dettagliata dei fattori chiave che hanno contribuito alla vittoria
    \item Il sistema evidenzia le principali differenze tra i due siti
    \item Il sistema fornisce suggerimenti su come il sito perdente potrebbe migliorare
    \item L'utente può navigare tra diverse aree di confronto
    \item L'utente può tornare alla vista principale dei risultati
\end{enumerate}

\subsubsection{Estensioni}
\begin{itemize}
    \item 2a. Il confronto è stato molto equilibrato
    \begin{enumerate}
        \item Il sistema spiega i fattori di desempate utilizzati
        \item Il sistema mostra quanto è stato ravvicinato il confronto
    \end{enumerate}
\end{itemize}

\subsubsection{Requisiti Speciali}
\begin{itemize}
    \item Le spiegazioni devono essere comprensibili anche per utenti non tecnici
    \item I suggerimenti di miglioramento devono essere pratici e attuabili
    \item La visualizzazione deve evidenziare chiaramente i punti di forza e debolezza
\end{itemize}

\subsubsection{Frequenza}
Frequente (molti utenti vogliono capire il risultato)

\section{Flussi di Interazione Principali}

\subsection{Flusso Base}
\begin{enumerate}
    \item L'utente inserisce gli URL dei due siti (UC1)
    \item Il sistema valida gli URL (UC2)
    \item Il sistema avvia l'analisi (UC3)
    \item L'utente visualizza le animazioni durante l'analisi (UC4)
    \item Il sistema mostra i risultati (UC5)
    \item L'utente esplora i dettagli (UC6)
    \item L'utente esporta i risultati (UC7)
\end{enumerate}

\subsection{Flusso Alternativo - Validazione Fallita}
\begin{enumerate}
    \item L'utente inserisce gli URL dei due siti (UC1)
    \item Il sistema determina che i siti non sono confrontabili (UC2)
    \item L'utente sceglie di procedere comunque
    \item Il sistema avvia l'analisi con avviso (UC3)
    \item Il flusso continua come nel flusso base
\end{enumerate}

\subsection{Flusso Alternativo - Analisi Parziale}
\begin{enumerate}
    \item L'utente inserisce gli URL dei due siti (UC1)
    \item Il sistema valida gli URL (UC2)
    \item Il sistema avvia l'analisi (UC3)
    \item Alcune analisi esterne falliscono
    \item Il sistema mostra risultati parziali con avviso (UC5)
    \item L'utente può esplorare i dati disponibili (UC6)
\end{enumerate}

\section{Requisiti Non Funzionali dei Casi d'Uso}

\subsection{Performance}
\begin{itemize}
    \item UC3 (Avviare Analisi): Completamento entro 25 secondi per l'analisi completa
    \item UC4 (Visualizzare Animazioni): 60 fps per le animazioni, degradando su dispositivi meno potenti
    \item UC2 (Validare URL): Validazione completa entro 5 secondi
\end{itemize}

\subsection{Usabilità}
\begin{itemize}
    \item UC1 (Inserire URL): Form semplice e intuitivo, accessibile da tastiera
    \item UC5 (Visualizzare Risultati): Visualizzazione chiara e comprensibile dei dati tecnici
    \item UC6 (Esplorare Dettagli): Navigazione intuitiva tra le categorie
\end{itemize}

\subsection{Sicurezza}
\begin{itemize}
    \item UC2 (Validare URL): Sanitizzazione degli input per prevenire attacchi
    \item UC3 (Avviare Analisi): Protezione delle chiavi API dai client
    \item UC7 (Esportare Risultati): Prevenzione di data leakage nei file esportati
\end{itemize}

\subsection{Scalabilità}
\begin{itemize}
    \item UC3 (Avviare Analisi): Gestione parallela di multiple richieste di analisi
    \item UC2 (Validare URL): Cache dei risultati di validazione per URL frequenti
\end{itemize}

\section{Diagramma di Stato del Processo di Analisi}
Il diagramma seguente illustra gli stati attraverso cui passa il sistema durante il processo di analisi:

```mermaid
stateDiagram-v2
    [*] --> Iniziale
    Iniziale --> URLInseriti: inserimento
    URLInseriti --> Validazione: invio
    Validazione --> AnalisiInCorso: validazione riuscita
    AnalisiInCorso --> RisultatiVisualizzati: analisi completata
    RisultatiVisualizzati --> NuovaAnalisi: reset
    NuovaAnalisi --> Iniziale: nuova operazione
    Validazione --> URLInseriti: errore validazione
```

Il diagramma di stato mostra come il sistema passa attraverso diversi stati durante il processo di analisi:

\begin{enumerate}
    \item \textbf{Iniziale}: Lo stato di partenza quando l'utente accede al sistema
    \item \textbf{URL Inseriti}: L'utente ha inserito gli URL ma non ha ancora avviato l'analisi
    \item \textbf{Validazione}: Il sistema sta verificando la validità e la pertinenza degli URL
    \item \textbf{Analisi In Corso}: Il sistema sta eseguendo le analisi sui siti web
    \item \textbf{Risultati Visualizzati}: Il sistema mostra i risultati all'utente
    \item \textbf{Nuova Analisi}: L'utente decide di avviare una nuova analisi
\end{enumerate}

\section{Diagramma di Attività - Analisi Completa}
Il diagramma seguente illustra il flusso di attività complete durante un processo di analisi:

```mermaid
flowchart TD
    InserireURL["Inserire URL"] --> ValidareURL["Validare URL"]
    ValidareURL --> VerificarePertinenza["Verificare Pertinenza"]
    VerificarePertinenza --> AvviareAnalisi["Avviare Analisi"]
    
    VerificarePertinenza --> EPertinente{"È Pertinente?"}
    EPertinente -- "No" --> MostrareAvviso["Mostrare Avviso"]
    EPertinente -- "Sì" --> EseguireAnalisiClient["Eseguire Analisi Lato Client"]
    
    MostrareAvviso --> ContinuaComunque["Continua Comunque"]
    ContinuaComunque --> EseguireAnalisiClient
    
    EseguireAnalisiClient --> EseguireAnalisiServer["Eseguire Analisi Lato Server"]
    EseguireAnalisiServer --> CombinareRisultati["Combinare Risultati"]
    CombinareRisultati --> DeterminareVincitore["Determinare Vincitore"]
    DeterminareVincitore --> MostrareRisultati["Mostrare Risultati"]
```

\section{Matrice Casi d'Uso-Requisiti}
La tabella seguente mappa i casi d'uso ai requisiti funzionali e non funzionali del sistema, mostrando quali requisiti sono soddisfatti da ciascun caso d'uso:

\begin{table}[H]
\centering
\begin{tabular}{|l|c|c|c|c|c|c|}
\hline
\textbf{Caso d'Uso} & \textbf{Performance} & \textbf{Usabilità} & \textbf{Accessibilità} & \textbf{Sicurezza} & \textbf{Accuratezza} & \textbf{Responsività} \\
\hline
UC1: Inserire URL & & \checkmark & \checkmark & \checkmark & & \checkmark \\
\hline
UC2: Validare URL & \checkmark & & & \checkmark & \checkmark & \\
\hline
UC3: Avviare Analisi & \checkmark & & & \checkmark & \checkmark & \\
\hline
UC4: Visualizzare Animazioni & \checkmark & \checkmark & \checkmark & & & \checkmark \\
\hline
UC5: Visualizzare Risultati & & \checkmark & \checkmark & & \checkmark & \checkmark \\
\hline
UC6: Esplorare Dettagli & & \checkmark & \checkmark & & \checkmark & \checkmark \\
\hline
UC7: Esportare Risultati & & \checkmark & & \checkmark & \checkmark & \\
\hline
UC8: Analizzare Vincitore & & \checkmark & \checkmark & & \checkmark & \\
\hline
\end{tabular}
\caption{Matrice Casi d'Uso-Requisiti}
\label{table:use-case-requirements}
\end{table}

\section{Analisi dei Rischi per i Casi d'Uso}
L'analisi dei rischi identifica i potenziali problemi che potrebbero verificarsi durante l'esecuzione dei casi d'uso e le strategie per mitigarli:

\begin{table}[H]
\centering
\begin{tabular}{|l|l|l|l|}
\hline
\textbf{Caso d'Uso} & \textbf{Rischio} & \textbf{Impatto} & \textbf{Strategia di Mitigazione} \\
\hline
UC2: Validare URL & API AI non disponibile & Medio & Procedere senza validazione di pertinenza \\
\hline
UC3: Avviare Analisi & Timeout API esterne & Alto & Implementare strategie di fallback e cache \\
\hline
UC3: Avviare Analisi & Rate limiting API & Alto & Implementare code e prioritizzazione \\
\hline
UC4: Visualizzare Animazioni & Performance browser limitata & Medio & Degrado graceful delle animazioni \\
\hline
UC5: Visualizzare Risultati & Dati incompleti & Alto & Mostrare avvisi e risultati parziali \\
\hline
UC7: Esportare Risultati & Errore generazione file & Basso & Offrire formati alternativi e retry \\
\hline
\end{tabular}
\caption{Analisi dei Rischi per i Casi d'Uso}
\label{table:use-case-risks}
\end{table}

\section{Considerazioni sull'Esperienza Utente}
I casi d'uso sono stati progettati per garantire un'esperienza utente ottimale, considerando diversi aspetti:

\begin{itemize}
    \item \textbf{Feedback Continuo}: Durante il processo di analisi, l'utente riceve feedback costante attraverso animazioni e indicatori di avanzamento.
    
    \item \textbf{Degradazione Elegante}: In caso di errori o limitazioni, il sistema offre alternative e continua a funzionare con capacità ridotte invece di fallire completamente.
    
    \item \textbf{Chiarezza dei Risultati}: I risultati sono presentati in modo chiaro e comprensibile, con diversi livelli di dettaglio per soddisfare utenti con diverse esigenze.
    
    \item \textbf{Accessibilità}: Tutti i casi d'uso considerano le esigenze di accessibilità, garantendo che l'applicazione sia utilizzabile da utenti con diverse capacità.
    
    \item \textbf{Apprendimento Progressivo}: L'interfaccia è progettata per consentire un apprendimento progressivo, con informazioni di base immediatamente comprensibili e dettagli tecnici accessibili su richiesta.
\end{itemize}
\chapter{Modello dei Dati}

\section{Panoramica del Modello dei Dati}
Il modello dei dati di Site War è progettato per gestire efficacemente i dati relativi all'analisi e al confronto dei siti web. Questo modello è strutturato per supportare l'acquisizione, l'elaborazione, la comparazione e la visualizzazione dei dati in modo coerente e performante.

Il sistema si basa su un modello orientato agli oggetti che rappresenta i risultati delle analisi attraverso classi ben definite, con una chiara gerarchia e relazioni tra le diverse entità. Questo approccio facilita l'estensibilità del sistema e la manutenzione del codice, consentendo l'aggiunta di nuove metriche e categorie di analisi senza impattare la struttura esistente.

\section{Diagramma delle Classi - Modello dei Dati}

Il diagramma seguente illustra la struttura principale del modello dei dati di Site War, mostrando le classi principali e le loro relazioni:

```mermaid
classDiagram
    class AnalysisResult {
        +site1: SiteAnalysis
        +site2: SiteAnalysis
        +winner: string
        +comparison: ComparisonResult
        +timestamp: number
        +determineWinner() string
        +getWinningCategories(site) string[]
        +toJSON() object
    }
    
    class SiteAnalysis {
        +url: string
        +performance: PerformanceMetrics
        +seo: SEOMetrics
        +security: SecurityMetrics
        +technical: TechnicalMetrics
        +finalScore: number
        +calculateFinalScore() number
        +getHighestScoringCategory() string
        +getLowestScoringCategory() string
    }
    
    class ComparisonResult {
        +performance: string
        +seo: string
        +security: string
        +technical: string
        +overallScore: string
        +scoreGap: number
        +getWinningCategories(site) string[]
        +isCloseDuel() boolean
    }
    
    class BaseMetrics {
        +score: number
        +metrics: object
        +details: object
        +getScore() number
        +getMetrics() object
        +getDetails() object
    }
    
    class PerformanceMetrics {
        +fcp: number
        +lcp: number
        +tti: number
        +cls: number
        +totalSize: number
        +getTiming() obj
    }
    
    class SEOMetrics {
        +title: string
        +meta: object
        +headings: object
        +links: object
        +images: object
        +getSEOIssues()
    }
    
    class SecurityMetrics {
        +ssl: object
        +headers: object
        +vulnerabilities
        +cookiesSecurity
        +csp: object
        +getVulnCount()
    }
    
    class TechnicalMetrics {
        +html: object
        +css: object
        +javascript: obj
        +responsive: bool
        +technologies: []
        +getTechStack()
    }
    
    class DomainInfo {
        +registrar: string
        +creationDate: date
        +expiryDate: date
        +nameservers: []
        +ipAddress: string
        +getAge() number
    }
    
    AnalysisResult --> SiteAnalysis
    AnalysisResult --> ComparisonResult
    BaseMetrics <|-- PerformanceMetrics
    BaseMetrics <|-- SEOMetrics
    BaseMetrics <|-- SecurityMetrics
    BaseMetrics <|-- TechnicalMetrics
```

\section{Struttura Dati - Formato JSON}

\subsection{Formato Completo}
Il sistema utilizza JSON come formato principale per la rappresentazione e lo scambio di dati. Di seguito è riportato un esempio di struttura JSON completa per i risultati di un'analisi:

\begin{asciiart}
{
  "site1": {
    "url": "https://example1.com",
    "performance": {
      "score": 85,
      "metrics": {
        "fcp": 1200,
        "lcp": 2500,
        "tti": 3500,
        "cls": 0.05,
        "totalSize": 1500000
      },
      "details": {
        "resources": {
          "js": 450000,
          "css": 120000,
          "images": 850000,
          "fonts": 80000
        },
        "resourceCount": 32,
        "renderBlocking": 3
      }
    },
    "seo": {
      "score": 78,
      "metrics": {
        "title": "Good",
        "meta": "Average",
        "headings": "Good",
        "images": "Average",
        "links": "Good"
      },
      "details": {
        "titleLength": 55,
        "metaDescription": 140,
        "hasCanonical": true,
        "imgAltTags": 85,
        "headingStructure": "Well structured",
        "linkQuality": "Good"
      }
    },
    "security": {
      "score": 92,
      "metrics": {
        "ssl": "A+",
        "headers": "Good",
        "vulnerabilities": 0,
        "cookies": "Secure",
        "csp": "Implemented"
      },
      "details": {
        "sslDetails": {
          "grade": "A+",
          "protocol": "TLS 1.3",
          "expiry": "2024-05-15"
        },
        "securityHeaders": {
          "strictTransportSecurity": true,
          "contentSecurityPolicy": true,
          "xFrameOptions": true,
          "xContentTypeOptions": true,
          "referrerPolicy": true
        },
        "cookiesDetails": {
          "secure": true,
          "httpOnly": true,
          "sameSite": "Strict"
        }
      }
    },
    "technical": {
      "score": 88,
      "metrics": {
        "html": "Valid",
        "css": "Valid",
        "javascript": "Modern",
        "responsive": true,
        "technologies": ["HTML5", "CSS3", "JavaScript", "jQuery"]
      },
      "details": {
        "htmlVersion": "HTML5",
        "cssVersion": "CSS3",
        "jsFeatures": ["ES6", "Modules"],
        "frameworks": ["jQuery 3.6.0"],
        "libraries": ["Animate.js", "Chart.js"],
        "serverInfo": {
          "server": "Nginx",
          "platform": "Linux"
        }
      }
    },
    "finalScore": 85.8
  },
  "site2": {
    // Struttura simile a site1
  },
  "winner": "site1",
  "comparison": {
    "performance": "site1",
    "seo": "site2",
    "security": "site1",
    "technical": "site1",
    "overallScore": "site1",
    "scoreGap": 7.3
  },
  "timestamp": 1633276850000
}
\end{asciiart}

\subsection{Formato Semplificato per UI}
Per ottimizzare la visualizzazione nell'interfaccia utente, viene utilizzata una versione semplificata del formato JSON:

\begin{asciiart}
{
  "site1": {
    "url": "https://example1.com",
    "scores": {
      "performance": 85,
      "seo": 78,
      "security": 92,
      "technical": 88,
      "final": 85.8
    },
    "highlights": {
      "strengths": ["Fast loading time", "Excellent security", "Modern technologies"],
      "weaknesses": ["Meta descriptions", "Image optimization"]
    }
  },
  "site2": {
    // Struttura simile a site1
  },
  "winner": {
    "site": "site1",
    "name": "example1.com",
    "margin": "Clear win (7.3 points)"
  },
  "categoryWinners": {
    "performance": "site1",
    "seo": "site2",
    "security": "site1",
    "technical": "site1"
  }
}
\end{asciiart}

\section{Diagramma delle Interazioni con i Dati}
Il diagramma seguente illustra il flusso di interazioni con i dati durante il processo di analisi:

```mermaid
flowchart LR
    Analyzers["Analyzers"] --> RawData["Raw Data"]
    RawData --> NormalizedData["Normalized Data"]
    NormalizedData --> CategoryScores["Category Scores"]
    CategoryScores --> ResultsProcessor["Results Processor"]
    ResultsProcessor --> UIDisplay["UI Display"]
```


\section{Diagramma ER per Risultati dell'Analisi}
Il diagramma Entity-Relationship (ER) seguente illustra le relazioni tra le entità del modello dei dati:

```mermaid
erDiagram
    AnalysisResult ||--|| ComparisonResult : has
    AnalysisResult ||--|{ SiteAnalysis : contains
    SiteAnalysis ||--|| DomainInfo : has
    DomainInfo ||--|{ CategoryMetrics : has
    CategoryMetrics ||--|| CategoryScore : scores
    CategoryMetrics ||--o{ PerformanceMetrics : "specializes to"
    CategoryMetrics ||--o{ SEOMetrics : "specializes to"
    CategoryMetrics ||--o{ SecurityMetrics : "specializes to"
    CategoryMetrics ||--o{ TechnicalMetrics : "specializes to"
    CategoryMetrics ||--o{ AccessibilityMetrics : "specializes to"
```

\section{Diagramma di Aggregazione dei Dati}
Il processo di aggregazione dei dati grezzi in risultati significativi è illustrato nel seguente diagramma:

```mermaid
flowchart TD
    RawMetrics["Raw Metrics"] -- "aggregated into" --> CategoryMetrics["Category Metrics"]
    CategoryMetrics -- "aggregated into" --> SiteAnalysis["Site Analysis"]
    SiteAnalysis -- "compared to create" --> AnalysisResult["Analysis Result"]
```

\section{Dettaglio delle Entità di Dati Principali}

\subsection{AnalysisResult}
\textbf{Responsabilità}: Rappresenta i risultati completi dell'analisi e del confronto tra due siti web.

\textbf{Attributi}:
\begin{itemize}
    \item \texttt{site1}: Risultati completi dell'analisi del primo sito
    \item \texttt{site2}: Risultati completi dell'analisi del secondo sito
    \item \texttt{winner}: Indicatore del sito vincitore (``site1'', ``site2'' o ``tie'')
    \item \texttt{comparison}: Risultati dettagliati del confronto
    \item \texttt{timestamp}: Data e ora dell'analisi
\end{itemize}

\textbf{Metodi}:
\begin{itemize}
    \item \texttt{determineWinner()}: Determina il vincitore in base ai punteggi finali
    \item \texttt{getWinningCategories(site)}: Ottiene le categorie in cui un sito ha ottenuto i punteggi più alti
    \item \texttt{toJSON()}: Serializza i risultati in formato JSON
\end{itemize}

\subsection{SiteAnalysis}
\textbf{Responsabilità}: Contiene i dati di analisi completi per un singolo sito web.

\textbf{Attributi}:
\begin{itemize}
    \item \texttt{url}: URL del sito analizzato
    \item \texttt{performance}: Metriche di performance
    \item \texttt{seo}: Metriche SEO
    \item \texttt{security}: Metriche di sicurezza
    \item \texttt{technical}: Metriche tecniche
    \item \texttt{finalScore}: Punteggio finale ponderato
\end{itemize}

\textbf{Metodi}:
\begin{itemize}
    \item \texttt{calculateFinalScore()}: Calcola il punteggio finale ponderato
    \item \texttt{getHighestScoringCategory()}: Restituisce la categoria con il punteggio più alto
    \item \texttt{getLowestScoringCategory()}: Restituisce la categoria con il punteggio più basso
\end{itemize}

\subsection{PerformanceMetrics}
\textbf{Responsabilità}: Contiene le metriche di performance di un sito web.

\textbf{Attributi}:
\begin{itemize}
    \item \texttt{score}: Punteggio complessivo di performance (0-100)
    \item \texttt{metrics}: Valori specifici delle metriche
    \begin{itemize}
        \item \texttt{fcp}: First Contentful Paint (ms)
        \item \texttt{lcp}: Largest Contentful Paint (ms)
        \item \texttt{tti}: Time to Interactive (ms)
        \item \texttt{cls}: Cumulative Layout Shift
        \item \texttt{totalSize}: Dimensione totale della pagina (bytes)
    \end{itemize}
    \item \texttt{details}: Dettagli aggiuntivi sulla performance
\end{itemize}

\textbf{Metodi}:
\begin{itemize}
    \item \texttt{getScore()}: Restituisce il punteggio complessivo
    \item \texttt{getMetrics()}: Restituisce le metriche specifiche
    \item \texttt{getTiming()}: Restituisce i tempi di caricamento aggregati
\end{itemize}

\subsection{SEOMetrics}
\textbf{Responsabilità}: Contiene le metriche SEO di un sito web.

\textbf{Attributi}:
\begin{itemize}
    \item \texttt{score}: Punteggio complessivo SEO (0-100)
    \item \texttt{metrics}: Valori specifici delle metriche
    \begin{itemize}
        \item \texttt{title}: Valutazione del titolo
        \item \texttt{meta}: Valutazione dei meta tag
        \item \texttt{headings}: Valutazione della struttura dei titoli
        \item \texttt{images}: Valutazione delle immagini
        \item \texttt{links}: Valutazione dei link
    \end{itemize}
    \item \texttt{details}: Dettagli aggiuntivi SEO
\end{itemize}

\textbf{Metodi}:
\begin{itemize}
    \item \texttt{getScore()}: Restituisce il punteggio complessivo
    \item \texttt{getMetrics()}: Restituisce le metriche specifiche
    \item \texttt{getSEOIssues()}: Restituisce i problemi SEO rilevati
\end{itemize}

\subsection{SecurityMetrics}
\textbf{Responsabilità}: Contiene le metriche di sicurezza di un sito web.

\textbf{Attributi}:
\begin{itemize}
    \item \texttt{score}: Punteggio complessivo di sicurezza (0-100)
    \item \texttt{metrics}: Valori specifici delle metriche
    \begin{itemize}
        \item \texttt{ssl}: Valutazione SSL/TLS
        \item \texttt{headers}: Valutazione degli header di sicurezza
        \item \texttt{vulnerabilities}: Numero di vulnerabilità rilevate
        \item \texttt{cookies}: Valutazione della sicurezza dei cookies
        \item \texttt{csp}: Valutazione della Content Security Policy
    \end{itemize}
    \item \texttt{details}: Dettagli aggiuntivi sulla sicurezza
\end{itemize}

\textbf{Metodi}:
\begin{itemize}
    \item \texttt{getScore()}: Restituisce il punteggio complessivo
    \item \texttt{getMetrics()}: Restituisce le metriche specifiche
    \item \texttt{getVulnCount()}: Restituisce il numero di vulnerabilità
\end{itemize}

\subsection{TechnicalMetrics}
\textbf{Responsabilità}: Contiene le metriche tecniche di un sito web.

\textbf{Attributi}:
\begin{itemize}
    \item \texttt{score}: Punteggio complessivo tecnico (0-100)
    \item \texttt{metrics}: Valori specifici delle metriche
    \begin{itemize}
        \item \texttt{html}: Valutazione HTML
        \item \texttt{css}: Valutazione CSS
        \item \texttt{javascript}: Valutazione JavaScript
        \item \texttt{responsive}: Flag di responsività
        \item \texttt{technologies}: Array di tecnologie rilevate
    \end{itemize}
    \item \texttt{details}: Dettagli aggiuntivi tecnici
\end{itemize}

\textbf{Metodi}:
\begin{itemize}
    \item \texttt{getScore()}: Restituisce il punteggio complessivo
    \item \texttt{getMetrics()}: Restituisce le metriche specifiche
    \item \texttt{getTechStack()}: Restituisce lo stack tecnologico completo
\end{itemize}

\subsection{ComparisonResult}
\textbf{Responsabilità}: Contiene i risultati del confronto tra i due siti web.

\textbf{Attributi}:
\begin{itemize}
    \item \texttt{performance}: Vincitore per la categoria performance
    \item \texttt{seo}: Vincitore per la categoria SEO
    \item \texttt{security}: Vincitore per la categoria sicurezza
    \item \texttt{technical}: Vincitore per la categoria tecnica
    \item \texttt{overallScore}: Vincitore complessivo
    \item \texttt{scoreGap}: Differenza tra i punteggi finali
\end{itemize}

\textbf{Metodi}:
\begin{itemize}
    \item \texttt{getWinningCategories(site)}: Restituisce le categorie vinte da un sito
    \item \texttt{isCloseDuel()}: Determina se il confronto è stato equilibrato
\end{itemize}

\section{Diagramma di Ereditarietà - Metriche}
Il sistema utilizza l'ereditarietà per organizzare le diverse metriche in una gerarchia coerente:

```mermaid
classDiagram
    BaseMetrics <|-- PerformanceMetrics
    BaseMetrics <|-- SEOMetrics
    BaseMetrics <|-- SecurityMetrics
    BaseMetrics <|-- TechnicalMetrics
    BaseMetrics <|-- AccessibilityMetrics
    
    class BaseMetrics {
        +score: number
        +metrics: object
        +details: object
        +getScore() number
        +getMetrics() object
        +getDetails() object
    }
```

\section{Diagramma dei Punteggi}
Il calcolo dei punteggi finali si basa su una ponderazione delle diverse categorie di analisi:

```mermaid
flowchart TD
    Performance["Performance Score"] -- "30%" --> FinalScore["Final Score"]
    SEO["SEO Score"] -- "25%" --> FinalScore
    Security["Security Score"] -- "25%" --> FinalScore
    Technical["Technical Score"] -- "20%" --> FinalScore
    Additional["Additional Metrics"] --> FinalScore
```

\section{Diagramma di Persistenza}
I dati generati dal sistema possono essere persistiti in diversi formati:

```mermaid
flowchart TD
    AnalysisResults["Analysis Results"] -- "stored as" --> JSONFiles["JSON Files"]
    JSONFiles <-- "cached in" --> MemoryCache["Memory Cache"]
    JSONFiles -- "exported as" --> CSVReports["CSV Reports"]
    JSONFiles -- "exported as" --> PDFReports["PDF Reports"]
```

\section{Specifica delle Interfacce dei Dati}

\subsection{Interfaccia AnalysisResult}
\begin{asciiart}
interface AnalysisResult {
  site1: SiteAnalysis;
  site2: SiteAnalysis;
  winner: 'site1' | 'site2' | 'tie';
  comparison: ComparisonResult;
  timestamp: number;
}
\end{asciiart}

\subsection{Interfaccia SiteAnalysis}
\begin{asciiart}
interface SiteAnalysis {
  url: string;
  performance: PerformanceMetrics;
  seo: SEOMetrics;
  security: SecurityMetrics;
  technical: TechnicalMetrics;
  finalScore: number;
}
\end{asciiart}

\subsection{Interfaccia BaseMetrics}
\begin{asciiart}
interface BaseMetrics {
  score: number;
  metrics: Record<string, any>;
  details: Record<string, any>;
}
\end{asciiart}

\subsection{Interfaccia PerformanceMetrics}
\begin{asciiart}
interface PerformanceMetrics extends BaseMetrics {
  metrics: {
    fcp: number;
    lcp: number;
    tti: number;
    cls: number;
    totalSize: number;
  };
  details: {
    resources: {
      js: number;
      css: number;
      images: number;
      fonts: number;
    };
    resourceCount: number;
    renderBlocking: number;
  };
}
\end{asciiart}

\subsection{Interfaccia SEOMetrics}
\begin{asciiart}
interface SEOMetrics extends BaseMetrics {
  metrics: {
    title: string;
    meta: string;
    headings: string;
    images: string;
    links: string;
  };
  details: {
    titleLength: number;
    metaDescription: number;
    hasCanonical: boolean;
    imgAltTags: number;
    headingStructure: string;
    linkQuality: string;
  };
}
\end{asciiart}

\subsection{Interfaccia SecurityMetrics}
\begin{asciiart}
interface SecurityMetrics extends BaseMetrics {
  metrics: {
    ssl: string;
    headers: string;
    vulnerabilities: number;
    cookies: string;
    csp: string;
  };
  details: {
    sslDetails: {
      grade: string;
      protocol: string;
      expiry: string;
    };
    securityHeaders: {
      strictTransportSecurity: boolean;
      contentSecurityPolicy: boolean;
      xFrameOptions: boolean;
      xContentTypeOptions: boolean;
      referrerPolicy: boolean;
    };
    cookiesDetails: {
      secure: boolean;
      httpOnly: boolean;
      sameSite: string;
    };
  };
}
\end{asciiart}

\subsection{Interfaccia TechnicalMetrics}
\begin{asciiart}
interface TechnicalMetrics extends BaseMetrics {
  metrics: {
    html: string;
    css: string;
    javascript: string;
    responsive: boolean;
    technologies: string[];
  };
  details: {
    htmlVersion: string;
    cssVersion: string;
    jsFeatures: string[];
    frameworks: string[];
    libraries: string[];
    serverInfo: {
      server: string;
      platform: string;
    };
  };
}
\end{asciiart}

\subsection{Interfaccia ComparisonResult}
\begin{asciiart}
interface ComparisonResult {
  performance: 'site1' | 'site2' | 'tie';
  seo: 'site1' | 'site2' | 'tie';
  security: 'site1' | 'site2' | 'tie';
  technical: 'site1' | 'site2' | 'tie';
  overallScore: 'site1' | 'site2' | 'tie';
  scoreGap: number;
}
\end{asciiart}

\section{Normalizzazione dei Dati}

\subsection{Diagramma di Normalizzazione}

```mermaid
flowchart LR
    RawData["Raw Data"] --> Validation["Validation"]
    Validation --> TypeConversion["Type Conversion"]
    TypeConversion --> ScaleAdjustment["Scale Adjustment"]
    ScaleAdjustment --> FinalProcessing["Final Processing"]
    FinalProcessing --> NormalizedData["Normalized Data"]
```

\subsection{Regole di Normalizzazione per Metriche Chiave}
\begin{table}[H]
\centering
\begin{tabular}{|l|l|l|c|}
\hline
\textbf{Metrica} & \textbf{Valore Raw} & \textbf{Valutazione} & \textbf{Punteggio} \\
\hline
FCP & < 1000ms & Eccellente & 90-100 \\
\hline
FCP & 1000-2500ms & Buono & 60-89 \\
\hline
FCP & 2500-4000ms & Migliorabile & 30-59 \\
\hline
FCP & > 4000ms & Scarso & 0-29 \\
\hline
LCP & < 2500ms & Eccellente & 90-100 \\
\hline
LCP & 2500-4000ms & Buono & 60-89 \\
\hline
LCP & 4000-6000ms & Migliorabile & 30-59 \\
\hline
LCP & > 6000ms & Scarso & 0-29 \\
\hline
CLS & < 0.1 & Eccellente & 90-100 \\
\hline
CLS & 0.1-0.25 & Buono & 60-89 \\
\hline
CLS & 0.25-0.4 & Migliorabile & 30-59 \\
\hline
CLS & > 0.4 & Scarso & 0-29 \\
\hline
SSL Grade & A+ & Eccellente & 90-100 \\
\hline
SSL Grade & A & Molto Buono & 80-89 \\
\hline
SSL Grade & B & Buono & 70-79 \\
\hline
SSL Grade & C & Migliorabile & 50-69 \\
\hline
SSL Grade & F & Scarso & 0-49 \\
\hline
\end{tabular}
\caption{Normalizzazione delle metriche principali}
\label{table:normalization-rules}
\end{table}

\section{Diagramma di Relazioni tra Metriche}

```mermaid
flowchart TD
    Performance["Performance"] -- "affects" --> UserExperience["User Experience"]
    Technical["Technical"] --> UserExperience
    Accessibility["Accessibility"] --> UserExperience
    
    UserExperience -- "contributes to" --> OverallQuality["Overall Quality"]
    Security["Security"] --> OverallQuality
    SEO["SEO"] --> OverallQuality
```

\section{Diagramma della Rappresentazione Visiva dei Dati}

```mermaid
flowchart TD
    AnalysisResults["Analysis Results"] -- "format for charts" --> ChartData["Chart Data"]
    ChartData -- "render as" --> RadarChart["Radar Chart"]
    
    AnalysisResults -- "format for tables" --> TableData["Table Data"]
    TableData -- "render as" --> ComparisonTable["Comparison Table"]
    TableData -- "render as" --> ScoreGauges["Score Gauges"]
    
    ComparisonTable --> DetailedMetricsView["Detailed Metrics View"]
    ScoreGauges --> DetailedMetricsView
    
    AnalysisResults -- "format for details" --> DetailedData["Detailed Data"]
    DetailedData --> DetailedMetricsView
```